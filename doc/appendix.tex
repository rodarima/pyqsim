\chapter{Notación}

En la siguiente tabla se recoge a modo de resumen la notación empleada a lo 
largo del proyecto.

\begin{table}[h]
	\centering
	\begin{tabular}{rl}
		\toprule
		Símbolo & Significado \\
		\midrule
		$A,B,C\ldots$
		& Matrices y operadores\\

		$A \otimes B$
		& Producto tensorial de $A$ y $B$\\

		$A^{\otimes n}$
		& Potencia tensorial de $A$:
		$A^{\otimes n} = A \otimes \ldots \otimes A$ ($n$ veces)\\

		$\V x, \V y, \V z\ldots$
		&Vectores columna\\

		$\V x = 010$
		&Vector en forma de cadena binaria: $\V x = \mat{0 & 1 & 0}^T$\\

		$\V x_i$
		&Elemento del vector $\V x$ en la posición $i$\\

		$\V x[i]$
		&Elemento del vector $\V x$ en la posición $i+1$\\

		$A_{ij}$
		&Elemento de $A$ en la fila $i$, columna $j$\\

		$A[i,j]$
		&Elemento de $A$ en la fila $i+1$, columna $j+1$\\

		$A[i,:]$
		&Vector fila correspondiente a la fila $i+1$\\

		$A[:,j]$
		&Vector columna correspondiente a la columna $j+1$\\

		$\ket{\psi}$
		&Estado cuántico descrito como un vector columna\\

		$\ket{\V x}$
		&Estado cuántico puro de un registro:
		$\ket{\V x} = \ket{\V x_1} \otimes \ldots \otimes \ket{\V 
		x_n}$\\

		$\ket{i}$
		&Igual que $\ket{\V x}$, pero tomando la representación binaria 
		de $i$\\

		$\ket0, \ket1$
		&Estados cuáticos puros\\

		$\ket{01}, \ket{0,1}, \ket0 \ket1$
		&Producto tensorial $\ket0 \otimes\ket1$\\

		$A^T$
		&Matriz transpuesta: $A^T_{ji} = A_{ij}$\\

		$\V z^T$
		&Vector fila del vector columna $\V z$\\

		$z^*$
		&Número complejo conjugado de $z$: $(1+i)^* = (1-i)$\\

		$\V z^*$
		&Vector complejo conjugado de $\V z$: $(\V z^*)_i =
		(\V z_i)^*$\\

		$A^*$
		&Matriz compleja conjugada de $A$: $(A^*)_{ij} =
		(A_{ij})^*$\\

		$A^\dagger$
		&Matriz adjunta de $A$: $A^\dagger = (A^T)^*$\\

		$x \oplus y$
		&Operación XOR de los bits $x$ e $y$\\

		$\V z = \V x \oplus \V y$
		&Operación XOR bit a bit de $\V x$ e $\V y$:
		$\V z_i = \V x_i \oplus \V y_i$\\

		$z = \V x \cdot \V y$
		&Producto interno binario de $\V x$ e $\V y$:
		$z = \V x_1 \V y_1 \oplus \ldots \oplus \V x_n \V y_n$\\
		\bottomrule
	\end{tabular}
\end{table}


