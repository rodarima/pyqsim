\documentclass[12pt,a4paper]{report}
\usepackage{amsfonts}
\usepackage{amsmath}
\usepackage{amsthm}
\usepackage[utf8]{inputenc}
\usepackage{braket}

% Código fuente
%\usepackage{fontspec}
\usepackage{minted}
\newminted{py}{%
		linenos,
		fontsize=\small,
		tabsize=4
}


% Emplear "tabla" y no "cuadro" para las tablas.
% Romper líneas por sílabas en español
\usepackage[spanish,es-nosectiondot,es-tabla]{babel}

% Usar estilo de libro para las tablas
\usepackage{booktabs}

% Gráficas
\usepackage{pgfplots}

% Emplear un tamaño más pequeño para las etiquetas
\usepackage[font=footnotesize]{caption}

% Cargar tablas de csv
\usepackage{pgfplotstable}

% Citas bibliograficas
\usepackage{cite}

% Mensajes en español para las referencias
\usepackage{babelbib}

% Urls
\usepackage{url}

% Dedicatoria
\newenvironment{dedication}{
\clearpage           % we want a new page
\thispagestyle{empty}% no header and footer
\vspace*{\stretch{1}}% some space at the top 
\itshape             % the text is in italics
\raggedleft          % flush to the right margin
}
{\par % end the paragraph
\vspace{\stretch{3}} % space at bottom is three times that at the top
\clearpage           % finish off the page
}

% Dimensiones de los márgenes.
%\usepackage[margin=3cm]{geometry}

% Macros para ayudar a la redacción
% Vector
\newcommand*\mat[1]{ \begin{pmatrix} #1 \end{pmatrix}}
\newcommand*\arr[1]{ \begin{bmatrix} #1 \end{bmatrix}}
\newcommand*\V[1]{ \boldsymbol{#1}}
% Ket psi
\newcommand*\ketp[1]{\ket{\psi_{#1}}}
% Floor
\newcommand{\floor}[1]{\lfloor #1 \rfloor}
% Absolute value
\newcommand\abs[1]{\left|#1\right|}

% Probabilidad I y ¬I
\newcommand*\NO[1]{ \overline{#1}}
\newcommand*\NI[1]{ \overline{I^{#1}}\,}

% Producto pensorial
\newcommand\ox[2]{\ket{#1} \otimes \ket{#2}}

% Apartado de un ejemplo
\theoremstyle{definition}
\newtheorem{ejemplo}{Ejemplo}[section]

% Gráficos con tikz
\usepackage{tikz}

\usetikzlibrary{calc,fadings,decorations.pathreplacing}
\usetikzlibrary{backgrounds,fit}

\usetikzlibrary{shapes,arrows,chains}
\usetikzlibrary{decorations.markings}

\newcommand\pgfmathsinandcos[3]{%
  \pgfmathsetmacro#1{sin(#3)}%
  \pgfmathsetmacro#2{cos(#3)}%
}
\newcommand\LongitudePlane[3][current plane]{%
  \pgfmathsinandcos\sinEl\cosEl{#2} % elevation
  \pgfmathsinandcos\sint\cost{#3} % azimuth
  \tikzset{#1/.estyle={cm={\cost,\sint*\sinEl,0,\cosEl,(0,0)}}}
}
\newcommand\LatitudePlane[3][current plane]{%
  \pgfmathsinandcos\sinEl\cosEl{#2} % elevation
  \pgfmathsinandcos\sint\cost{#3} % latitude
  \pgfmathsetmacro\yshift{\cosEl*\sint}
  \tikzset{#1/.estyle={cm={\cost,0,0,\cost*\sinEl,(0,\yshift)}}} %
}
\newcommand\DrawLongitudeCircle[2][1]{
  \LongitudePlane{\angEl}{#2}
%  \tikzset{current plane/.prefix style={scale=#1}}
   % angle of "visibility"
  \pgfmathsetmacro\angVis{atan(sin(#2)*cos(\angEl)/sin(\angEl))} %
  \draw[current plane] (\angVis:1) arc (\angVis:\angVis+180:1);
  \draw[current plane,dashed] (\angVis-180:1) arc (\angVis-180:\angVis:1);
}
\newcommand\DrawLatitudeCircle[2][1]{
  \LatitudePlane{\angEl}{#2}
%	\tikzset{current plane/.prefix style={scale=#1}}
  \pgfmathsetmacro\sinVis{sin(#2)/cos(#2)*sin(\angEl)/cos(\angEl)}
  % angle of "visibility"
  \pgfmathsetmacro\angVis{asin(min(1,max(\sinVis,-1)))}
  \draw[current plane] (\angVis:1) arc (\angVis:-\angVis-180:1);
  \draw[current plane,dashed] (180-\angVis:1) arc (180-\angVis:\angVis:1);
}
