\chapter{Circuitos cuánticos}
El objetivo de los circuitos cuánticos es el de aprovechar las propiedades que 
otorga la computación cuántica, para resolver problemas de forma más rápida y 
eficiente que los algoritmos convencionales. El estudio de sus propiedades 
permite determinar como aprovecharlas.

\section{Paralelismo}
Una de las técnicas en las que se basan los algoritmos cuánticos para acelerar 
su ejecución, consiste en la posibilidad de ejecutar muchas acciones a la vez.  
Esta característica se debe a la propiedad que posee un estado cuántico de 
encontrarse en superposición. Por ejemplo el estado
%
$$ \ket{\psi} = \frac{1}{\sqrt 2}(\ket{0} + \ket{1}) $$
%
Para observar como se puede hacer uso de esta propiedad, se diseñará un problema 
de ejemplo \cite{nielsen00}. Sea $f(x): \{0,1\} \rightarrow \{0,1\}$ una función 
binaria como la puerta clásica NOT. Entonces, una forma de implementarla en un 
operador cuántico $U_f$ es
%
$$ U_f \ket{x}\ket{y} = \ket{x}\ket{y \oplus f(x)} $$
%
Al aplicar el operador a un registro de dos qubits, al segundo registro se le 
suma (módulo 2) el resultado de $f$ sobre el primero. Por ejemplo, partiendo de 
un estado $\ket{1}\ket{0}$, y aplicando el operador $U_f$, se obtiene
%
$$ U_f \ket{1}\ket{0} = \ket{1}\ket{f(1)} $$
%
Esta extraña forma de computar $f$, está causada por la restricción de que los 
operadores cuánticos (como $U_f$) deben ser unitarios, para permitir una 
computación reversible.

Una vez obtenido el operador $U_f$, que dependerá de la función en cuentión, 
(pero será una matriz de $4 \times 4$ igualmente) es posible emplear un estado 
superpuesto como $\ket{\psi}$ como entrada de la función, colocándolo en 
$\ket{x}$. El otro qubit $\ket{y}$ queda como ket cero $\ket{0}$.
El operador $U_f$ actuará sobre el estado $\ket{\psi}\ket{0}$, produciendo:
$$ \ketp 1 = U_f\ket{\psi}\ket{0} = U_f\frac{1}{\sqrt 2}(\ket{0}\ket{0} + 
\ket{1}\ket{0}) =
\frac{1}{\sqrt 2}(U_f\ket{0}\ket{0} + U_f\ket{1}\ket{0})$$
%
Lo que está ocurriendo es que el operador $U_f$ está siendo aplicado 
simultáneamente sobre dos estados, $\ket{0}\ket{0}$ y $\ket{1}\ket{0}$, 
obteniendo finalmente:
%
$$ \ketp 1 = \frac{1}{\sqrt 2}(\ket{0}\ket{f(0)} + \ket{1}\ket{f(1)}) $$
%
En este estado, se observa como la función $f$ se evalúa simultáneamente en 0 y 
en 1. De forma que en un sólo paso, se realizan dos evaluaciones de $f$. Esta 
propiedad se conoce con el nombre de \textit{paralelismo cuántico}.
Sin embargo, no es posible observar los estados entrelazados, ya que el proceso 
de medición hace que se colapsen a uno de ellos. Es necesario realizar más 
operaciones para poder extraer la información deseada.

\section{Interferencia}

Partiendo de un estado inicial de dos qubits $\ketp 0 = \ket{01}$, y del mismo 
operador $U_f$ anteriormente descrito, el objetivo ahora será determinar que 
ocurre cuando se ejecuta la operación:
$$ (H \otimes I)(U_f)(H \otimes H)\ketp 0 $$
Primero, el operador de Hadamard actúa sobre ambos qubits
$$ \ketp1 = (H \otimes H)(\ket{0} \otimes \ket{1}) =
(H \ket{0} \otimes H \ket{1})$$
Transformándolos a un estado superpuesto
$$H \ket 0 = \frac{1}{\sqrt 2}(\ket 0 + \ket 1), \quad
H \ket 1 = \frac{1}{\sqrt 2}(\ket 0 - \ket 1)$$
Expresado conjuntamente como
$$ \ketp1 = \left( \frac{1}{\sqrt 2}(\ket 0 + \ket 1) \right)
\otimes \left( \frac{1}{\sqrt 2}(\ket 0 - \ket 1) \right) $$
Que se puede simplificar a
$$ \ketp1 = \frac{1}{2} \, \Big( \ket{00} - \ket{01} + \ket{10} - \ket{11} \Big) 
$$
Finalmente el operador $U_f$, definido como $ U_f \ket{x}\ket{y} = \ket{x}\ket{y 
\oplus f(x)}$, se aplica sobre $\ketp1$, transformándolo en
\begin{equation*}
\begin{split}
\ketp2 = \frac{1}{2} \, \Big(
&\ket{0}\otimes\ket{0 \oplus f(0)} - \ket{0}\otimes\ket{1 \oplus f(0)} + \\
&\ket{1}\otimes\ket{1 \oplus f(1)} - \ket{1}\otimes\ket{1 \oplus f(1)} \Big)
\end{split}
\end{equation*}
Que se puede expresar como
\begin{equation*}
\begin{split}
\ketp2 = \frac{1}{2} \, \Big(
& \ket{0} \otimes ( \ket{0 \oplus f(0)} - \ket{1 \oplus f(0)}) + \\
& \ket{1} \otimes ( \ket{1 \oplus f(1)} - \ket{1 \oplus f(1)}) \Big)
\end{split}
\end{equation*}
Y observando el hecho de que
$$ \ket{0 \oplus a} - \ket{1 \oplus a} = (-1)^a (\ket 0 - \ket 1) $$
Para $a = f(0)$ y $a = f(1)$, se simplifica en
\begin{equation*}
\begin{split}
\ketp2 = \frac{1}{2} \, \Big(
& (-1)^{f(0)} \ket{0} \otimes (\ket 0 - \ket 1) + \\
& (-1)^{f(1)} \ket{1} \otimes (\ket 0 - \ket 1) \Big)
\end{split}
\end{equation*}
Y finalmente, en
$$
\ketp2 = \frac{1}{2} \, \Big((-1)^{f(0)} \ket{0} + (-1)^{f(1)}\ket 1 \Big) 
\otimes \Big(\ket 0 - \ket 1) \Big)
$$
Si $f(0) = f(1)$, entonces $(-1)^{f(0)} = (-1)^{f(1)}$, y $\ketp 2$ sería
\begin{equation*}
\begin{split}
\ket{\psi_2^=} & = \pm \frac{1}{2} \, \Big(\ket 0 + \ket 1 \Big) \otimes 
\Big(\ket 0 - \ket 1 \Big) \\
& = \pm\left(\frac{\ket 0 + \ket 1}{\sqrt 2 } \right)
\otimes \left( \frac{\ket 0 -\ket 1}{\sqrt 2 } \right)
\end{split}
\end{equation*}
Ahora, un operador de Hadamard se aplica sobre el primer qubit, obteniendo
\begin{equation*}
\begin{split}
\ket{\psi_3^=}& = \pm (H\otimes I) \left(\frac{\ket 0 + \ket 1}{\sqrt 2 } 
\right)
\otimes \left( \frac{\ket 0 -\ket 1}{\sqrt 2 } \right) \\
&= \pm \ket 0 \otimes \left( \frac{\ket 0 -\ket 1}{\sqrt 2 } \right)
\end{split}
\end{equation*}
Si por el contrario $f(0) \neq f(1)$, entonces $(-1)^{f(0)} = - (-1)^{f(1)}$, y
\begin{equation*}
\begin{split}
\ket{\psi_2^{\neq}} & = \pm \frac{1}{2} \, \Big(\ket 0 - \ket 1 \Big) \otimes 
\Big(\ket 0 - \ket 1 \Big) \\
& = \pm\left(\frac{\ket 0 - \ket 1}{\sqrt 2 } \right)
\otimes \left( \frac{\ket 0 -\ket 1}{\sqrt 2 } \right)
\end{split}
\end{equation*}
Y tras el operador de Hadamard sobre el primer qubit
\begin{equation*}
\begin{split}
\ket{\psi_3^{\neq}}& = \pm (H\otimes I) \left(\frac{\ket 0 - \ket 1}{\sqrt 2 } 
\right)
\otimes \left( \frac{\ket 0 -\ket 1}{\sqrt 2 } \right) \\
&= \pm \ket 1 \otimes \left( \frac{\ket 0 -\ket 1}{\sqrt 2 } \right)
\end{split}
\end{equation*}
Obteniendo para $\ketp2$ de forma esquematizada
$$
\ket{\psi_2} = 
\begin{cases}
\pm\left(\frac{\ket 0 + \ket 1}{\sqrt 2 } \right)
\otimes \left( \frac{\ket 0 -\ket 1}{\sqrt 2 } \right)
, & \text{si}\ f(0) = f(1) \\
\pm\left(\frac{\ket 0 - \ket 1}{\sqrt 2 } \right)
\otimes \left( \frac{\ket 0 -\ket 1}{\sqrt 2 } \right)
, & \text{si}\ f(0) \neq f(1)
\end{cases}
$$
Y para $\ketp3$ los estados
$$
\ketp 3 =
\begin{cases}
\pm \ket 0 \otimes \left( \frac{\ket 0 -\ket 1}{\sqrt 2 } \right)
, & \text{si}\ f(0) = f(1) \\
\pm \ket 1 \otimes \left( \frac{\ket 0 -\ket 1}{\sqrt 2 } \right)
, & \text{si}\ f(0) \neq f(1)
\end{cases}
$$

% TODO: Observar el efecto de la interferencia mirando los signos

