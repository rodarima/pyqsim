\chapter{Circuitos cuánticos}
El objetivo de los circuitos cuánticos es el de aprovechar las propiedades de la 
computación cuántica, para resolver problemas de forma más rápida y eficiente 
que los algoritmos convencionales. El estudio estas propiedades permite 
determinar como aprovecharlas.

\section{Técnicas fundamentales}
El paralelismo cuántico y la interferencia son dos técnicas básicas que emplean 
los circuitos cuánticos para lograr una mejora en la complejidad.

\subsection{Paralelismo cuántico}
Una de las técnicas en las que se basan los algoritmos cuánticos para acelerar 
su ejecución, consiste en la posibilidad de ejecutar muchas acciones a la vez.  
Esta característica se debe a la propiedad que posee un estado cuántico de 
encontrarse en superposición, por ejemplo el estado $\ket{\psi}$.
%
$$ \ket{\psi} = \frac{1}{\sqrt 2}(\ket{0} + \ket{1}) $$
%
Para observar como se puede hacer uso de esta propiedad, se diseñará un problema 
de ejemplo \cite{nielsen00}. Sea $f(x): \{0,1\} \rightarrow \{0,1\}$ una función 
binaria como la puerta clásica NOT. Entonces, una forma de implementarla en un 
operador cuántico $U_f$ es
%
$$ U_f \ket{x}\ket{y} = \ket{x}\ket{y \oplus f(x)} $$
%
Al aplicar el operador a un registro de dos qubits, al segundo registro se le 
suma (módulo 2) el resultado de $f$ sobre el primero. Por ejemplo, partiendo de 
un estado $\ket{1}\ket{0}$, y aplicando el operador $U_f$, se obtiene
%
$$ U_f \ket{1}\ket{0} = \ket{1}\ket{f(1)} $$
%
Esta extraña forma de computar $f$, está causada por la restricción de que los 
operadores cuánticos (como $U_f$) deben ser unitarios, para permitir una 
computación reversible.

Una vez obtenido el operador $U_f$, que dependerá de la función en cuentión, 
(pero será una matriz de $4 \times 4$ igualmente) es posible emplear un estado 
superpuesto como $\ket{\psi}$ como entrada de la función, colocándolo en 
$\ket{x}$. El otro qubit $\ket{y}$ queda como ket cero $\ket{0}$.
El operador $U_f$ actuará sobre el estado $\ket{\psi}\ket{0}$, produciendo:
$$ \ketp 1 = U_f\ket{\psi}\ket{0} = U_f\frac{1}{\sqrt 2}(\ket{0}\ket{0} + 
\ket{1}\ket{0}) =
\frac{1}{\sqrt 2}(U_f\ket{0}\ket{0} + U_f\ket{1}\ket{0})$$
%
Lo que está ocurriendo es que el operador $U_f$ está siendo aplicado 
simultáneamente sobre dos estados, $\ket{0}\ket{0}$ y $\ket{1}\ket{0}$, 
obteniendo finalmente:
%
$$ \ketp 1 = \frac{1}{\sqrt 2}(\ket{0}\ket{f(0)} + \ket{1}\ket{f(1)}) $$
%
En este estado, se observa como la función $f$ se evalúa simultáneamente en 0 y 
en 1. De forma que en un único paso, se realizan dos evaluaciones de $f$. Esta 
propiedad se conoce con el nombre de \textit{paralelismo cuántico}.
Sin embargo, no es posible observar los estados entrelazados, ya que el proceso 
de medición hace que se colapsen a uno de ellos. Es necesario realizar más 
operaciones para poder extraer la información deseada.

\subsection{Interferencia}
Otra técnica empleada en los circuitos cuánticos es la interferencia. El efecto 
de interferencia proviene de las ondas, pues dos ondas con la misma frecuencia 
con la misma fase se suman, pero con fases opuestas se destruyen.  Esta 
propiedad, presente en el comportamiento cuántico, permite aprovechar los 
estados superpuestos computados en paralelo, y transformarlos para poder obtener 
información útil de todos ellos.
Un ejemplo consiste en una función binaria $f:\{0,1\} \rightarrow \{0,1\}$, que 
se desea determinar si es biyectiva, $f(0) \neq f(1)$, o si no lo es, $f(0) = 
f(1)$. Una solución clásica es calcular $f(0)$, luego $f(1)$, y comparar los 
valores. Pero requiere evaluar \textit{secuencialmente} la función dos veces.  
Una alternativa es emplear un circuito cuántico.
Partiendo de un estado inicial de dos qubits $\ketp 0 = \ket{01}$, y del mismo 
operador $U_f$ anteriormente descrito, el objetivo ahora será determinar que 
ocurre cuando se ejecuta la operación:
$$ (H \otimes I)(U_f)(H \otimes H)\ketp 0 $$
Este circuito se llama algoritmo de Deutsch, y fue diseñado para mostrar las 
capacidades de la computación cuántica \cite{deutsch85}.
Primero, el operador de Hadamard actúa sobre ambos qubits
$$ \ketp1 = (H \otimes H)(\ket{0} \otimes \ket{1}) =
(H \ket{0} \otimes H \ket{1})$$
transformándolos a un estado superpuesto
$$H \ket 0 = \frac{1}{\sqrt 2}(\ket 0 + \ket 1), \quad
H \ket 1 = \frac{1}{\sqrt 2}(\ket 0 - \ket 1)$$
expresado conjuntamente como
$$ \ketp1 = \left( \frac{1}{\sqrt 2}(\ket 0 + \ket 1) \right)
\otimes \left( \frac{1}{\sqrt 2}(\ket 0 - \ket 1) \right) $$
que se puede expresar como
$$ \ketp1 = \frac{1}{2} \, \Big( \ox 0 0 - \ox 0 1 + \ox 1 0 - \ox 1 1 \Big) $$
Posteriormente el operador $U_f$, definido como $ U_f \ket{x}\ket{y} = 
\ket{x}\ket{y \oplus f(x)}$, se aplica sobre $\ketp1$, transformándolo en
\begin{equation*}
\begin{split}
\ketp2 = \frac{1}{2} \, \Big(
&\ox 0 {0 \oplus f(0)} - \ox 0 {1 \oplus f(0)} + \\
&\ox 1 {1 \oplus f(1)} - \ox 1 {1 \oplus f(1)} \Big)
\end{split}
\end{equation*}
y extrayendo factor común $\ket{0}$ y $\ket{1}$ se obtiene
\begin{equation*}
\begin{split}
\ketp2 = \frac{1}{2} \, \Big(
& \ket{0} \otimes ( \ket{0 \oplus f(0)} - \ket{1 \oplus f(0)} ) + \\
& \ket{1} \otimes ( \ket{1 \oplus f(1)} - \ket{1 \oplus f(1)} ) \Big)
\end{split}
\end{equation*}
Y observando el hecho de que
$$ \ket{0 \oplus k} - \ket{1 \oplus k} = (-1)^k (\ket 0 - \ket 1) $$
Para $k = f(0)$ y $k = f(1)$, se convierte en
\begin{equation*}
\begin{split}
\ketp2 = \frac{1}{2} \, \Big(
& (-1)^{f(0)} \ket{0} \otimes (\ket 0 - \ket 1) + \\
& (-1)^{f(1)} \ket{1} \otimes (\ket 0 - \ket 1) \Big)
\end{split}
\end{equation*}
Y finalmente, en
$$
\ketp2 = \frac{1}{2} \, \Big((-1)^{f(0)} \ket{0} + (-1)^{f(1)}\ket 1 \Big) 
\otimes \Big(\ket 0 - \ket 1) \Big)
$$
Sea $a = (-1)^{f(0)}$ y $b = (-1)^{f(1)}$, se reescribe
$$
\ketp2 = \frac{1}{2} \, \Big(a \ket{0} + b \ket 1 \Big) \otimes \Big(\ket 0 - 
\ket 1) \Big)
$$
A partir de aquí, solo se empleará el primer qubit, de modo que se divide el 
estado $\ketp2$ en sus dos qubits $\ket{\psi_2^1} \otimes \ket{\psi_2^2}$, que 
son
$$
\ket{\psi_2^1} = \frac {a \ket{0} + b \ket 1}{\sqrt 2}\, \quad
\ket{\psi_2^2} = \frac {\ket 0 - \ket 1}{\sqrt 2}
$$
%
Ahora el operador de Hadamard se aplica sobre el primer qubit, $\ket{\psi_3^1} = 
H \ket{\psi_2^1}$, obteniendo
$$ \ket{\psi_3^1} = \frac{a \ket 0 + a \ket 1 + b \ket 0 - b \ket 1}{2}
= \frac{a+b}{2} \ket 0 + \frac{a-b}{2} \ket 1$$
%
Si $a = b = \pm 1$, el coeficiente del ket $\ket 0$ se sumará, quedando $\pm 1$.  
Es el denominado efecto de inferferencia constructiva. El ket $\ket 1$ por el 
contrario sufrirá el efecto opuesto, llamado interferencia destructiva, pues 
$(a-b) = 0$. De forma que el estado se transformará en $\pm \ket 0$.

Por el otro lado, si $a = -b = \pm 1$, sucederá el proceso inverso. El 
coeficiente del ket $\ket 0$ se destruirá, $(a+b) = (a-a) = 0$, y el del ket 
$\ket 1$ se sumará constructivamente $(a-b) = (a+a) = \pm 1$, obteniéndose $\pm
\ket 1$. Resultando los dos casos:
$$
\ket{\psi_3^1} = \begin{cases}
\pm \ket 0, & \text{si}\ a = b \\
\pm \ket 1, & \text{si}\ a = -b \\
\end{cases}
$$
Ahora si se mide este qubit, y se obtiene un 0, se deduce que $a = b$. Si se 
obtiene un 1, que $a=-b$. Esta información permite determinar si $f(0) = f(1)$, 
cuando $a = b$, o si $f(0) \neq f(1)$ cuando $a=-b$. El paralelismo cuántico 
permite evaluar la función $f$ de forma simultánea, y el efecto de la 
interferencia, transformar ese estado superpuesto, de forma que se obtenga un 0 
o un 1 correspondiente a la información que se desea obtener de la función.

\section{Otras técnicas}

La resolución de problemas complejos, requiere de técnicas más sofisticadas, que 
permitan transformar el problema en otro más simple. Un ejemplo es el caso de la 
transformada de Fourier para la cual se tiene una versión cuántica, empleada en 
el algoritmo de factorización numérica de Shor. La complejidad de la 
transformada de Fourier rápida clásica, se sitúa en $O(n2^n)$ para números de 
$n$ bits.  Sin embargo, para la versión cuántica, la complejidad se reduce a 
$O(n^2)$.

Por otra parte, otros algoritmos como el de búsqueda de Grover, emplean una 
técnica de aproximación al resultado. De forma que, iterativamente se acercan a 
la solución. La probabilidad de encontrar la solución correcta al final del 
algoritmo se reduce a medida que se realizan más iteraciones.

El algoritmo de Simon, que se analizará de forma minuciosa posteriormente, 
emplea la técnica de iteración para obtener en cada ejecución nueva información 
que permite acercarse a la solución.

%\section{Implementación}
% Comentar las técnicas de implementación de qubits. Aqui?

