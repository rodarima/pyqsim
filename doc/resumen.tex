\documentclass[12pt,a4paper]{article}
\usepackage{amsfonts}
\usepackage{amsmath}
\usepackage{amsthm}
\usepackage[utf8]{inputenc}
\usepackage{braket}

% Romper líneas por sílabas en español
\usepackage[spanish,es-nosectiondot]{babel}

% Dimensiones de los márgenes.
%\usepackage[margin=2.5cm]{geometry}

% Vector
\newcommand*\mat[1]{ \begin{pmatrix} #1 \end{pmatrix}}
\newcommand*\arr[1]{ \begin{bmatrix} #1 \end{bmatrix}}

\newcommand*\V[1]{ \boldsymbol{#1}}

% Apartado de un ejemplo
\theoremstyle{definition}
\newtheorem{ejemplo}{Ejemplo}[section]

% Gráficos con tikz
\usepackage{tikz}

\usetikzlibrary{calc,fadings,decorations.pathreplacing}
\usetikzlibrary{backgrounds,fit}
\newcommand\pgfmathsinandcos[3]{%
  \pgfmathsetmacro#1{sin(#3)}%
  \pgfmathsetmacro#2{cos(#3)}%
}
\newcommand\LongitudePlane[3][current plane]{%
  \pgfmathsinandcos\sinEl\cosEl{#2} % elevation
  \pgfmathsinandcos\sint\cost{#3} % azimuth
  \tikzset{#1/.estyle={cm={\cost,\sint*\sinEl,0,\cosEl,(0,0)}}}
}
\newcommand\LatitudePlane[3][current plane]{%
  \pgfmathsinandcos\sinEl\cosEl{#2} % elevation
  \pgfmathsinandcos\sint\cost{#3} % latitude
  \pgfmathsetmacro\yshift{\cosEl*\sint}
  \tikzset{#1/.estyle={cm={\cost,0,0,\cost*\sinEl,(0,\yshift)}}} %
}
\newcommand\DrawLongitudeCircle[2][1]{
  \LongitudePlane{\angEl}{#2}
%  \tikzset{current plane/.prefix style={scale=#1}}
   % angle of "visibility"
  \pgfmathsetmacro\angVis{atan(sin(#2)*cos(\angEl)/sin(\angEl))} %
  \draw[current plane] (\angVis:1) arc (\angVis:\angVis+180:1);
  \draw[current plane,dashed] (\angVis-180:1) arc (\angVis-180:\angVis:1);
}
\newcommand\DrawLatitudeCircle[2][1]{
  \LatitudePlane{\angEl}{#2}
%	\tikzset{current plane/.prefix style={scale=#1}}
  \pgfmathsetmacro\sinVis{sin(#2)/cos(#2)*sin(\angEl)/cos(\angEl)}
  % angle of "visibility"
  \pgfmathsetmacro\angVis{asin(min(1,max(\sinVis,-1)))}
  \draw[current plane] (\angVis:1) arc (\angVis:-\angVis-180:1);
  \draw[current plane,dashed] (180-\angVis:1) arc (180-\angVis:\angVis:1);
}

\begin{document}

\thispagestyle{empty}
\newgeometry{margin=3cm, top=3cm, bottom=2cm}
\section*{Diseño, desarrollo, implementación y prueba de un simulador cuántico 
para el algoritmo de Simon}

\vspace{1cm}

\noindent
\textbf{Alumno:} Rodrigo Arias Mallo\\
\textbf{Director:} Vicente Moret Bonillo \\
\textbf{Convocatoria de defensa:} Febrero de 2016\\

\section*{Resumen}
\noindent
La computación cuántica estudia cómo aprovechar las características de la 
materia a escala atómica para realizar cómputos. Surgen nuevas posibilidades 
para la resolución de problemas, que son imposibles en los ordenadores actuales.  
Conceptos cómo superposición o entrelazamiento suponen la mayor diferencia a la 
hora de comprender como funcionan las nuevas soluciones.

Un simulador cuántico permite realizar pruebas en un ordenador convencional, y 
observar los resultados para contrastarlos con un modelo teórico. Además, sirve 
como herramienta didáctica, debido a que permite la observación minuciosa de la 
ejecución de un circuito cuántico. Pese a la complejidad exponencial requerida 
para simular el comportamiento cuántico, la capacidad actual es suficiente para 
realizar pruebas a pequeña escala.

En este caso, se propone el análisis del algoritmo de Simon, que soluciona un 
problema cuya solución clásica tiene una complejidad de $O(2^n/2)$, y se puede 
resolver con un algoritmo cuántico en $O(n)$.

Adicionalmente, se miden y analizan los resultados obtenidos tanto de forma 
experimental con la simulación, así como los teóricos, que son contrastados. El 
tiempo de cómputo y los recursos necesarios son minuciosamente examinados 
durante la ejecución.

%\clearpage
%\thispagestyle{empty}
%\cleardoublepage

%{\let\clearpage\relax \chapter*{Palabras clave}}
\section*{Palabras clave}
\noindent
Computación cuántica, circuito cuántico, algoritmo de Simon, qubit,
superposición, simulación.
\end{document}
