\chapter{Simulador cuántico}
% Que es un simulador. Como se realiza la simulacion

Un simulador de circuitos cuántico, es un programa ejecutado en una máquina 
clásica que permite efectuar una simulación de un circuito, de forma que los 
resultados obtenidos sean idénticos a los que se obtendrían en una máquina 
cuántica que ejecute el mismo circuito.

La principal desventaja en un simulador cuántico, es que la simulación es un 
proceso muy costoso, que crece en complejidad de forma exponencial con el número 
de qubits. Sin embargo, permite analizar el comportamiento de un algoritmo, y 
comprobar que el comportamiento simulado es el predicho de forma teórica.

El proceso de simulación se basa en dos partes. Primero se determinan todas las 
puertas del circuito, y se aplican al estado inicial de forma secuencial. De 
esta forma se obtiene el estado final. Esta es la parte de simulación cuántica.

Por otra parte, es necesario un proceso de medida y cómputo empleando bits 
clásicos. De modo que, partiendo del estado final, se determina el proceso de 
medición calculando las probabilidades asociadas a cada estado clásico. Con la 
distribución de probabilidad fijada, se efectúa la generación de una medida 
aleatoria, y a continuación los cálculos necesarios con las mediciones 
resultantes.

\section{Diseño del simulador}
% Comentar el software con todas las partes y estructuras de datos.
El simulador consta de tres partes bien diferenciadas. Cada una se encarga de 
realizar una tarea en concreto.
%
\begin{itemize}
\item Simulación del circuito cuántico.
\item Medición de estados.
\item Procesado clásico.
\end{itemize}

Primero, el circuito cuántico es simulado, produciendo un estado final cuántico.  
A continuación, un proceso de medición colapsa los registros cuánticos en bits 
clásicos. Finalmente, un última etapa de cálculo clásico, toma dicha información 
para ser procesada. La simulación se repite hasta que el problema se haya 
completado.
%
\begin{center}
% Define block styles
\tikzstyle{decision} = [diamond, draw, text width=4.5em, text badly centered, 
inner sep=0pt]
\tikzstyle{block} = [rectangle, draw, text width=5em, text centered]
\tikzstyle{line} = [draw, decoration={markings,mark=at position 
1 with {\arrow[scale=1.5]{latex'}}}, postaction={decorate}]
%
\begin{tikzpicture}[node distance = 3cm, auto]
	% Place nodes
	\node[block] (quantum) {Simulación cuántica};
	\node[block, right of=quantum] (measure) {Medición};
	\node[block, right of=measure] (classic) {Procesado clásico};
	\node[decision, right of=classic] (end) {Fin?};
	% Draw edges
	\draw [line] (quantum.west)+(-1cm,0) -- (quantum.west);
	\draw [line] (quantum) -> (measure);
	\draw [line] (measure) -- (classic);
	\draw [line] (classic) -- (end);
	\draw [line] (end.south) |-+(0,-1em)-| node [near start, above] {No} 
(quantum.south);
	\draw [line] (end.east) -- node [near start] {Sí} +(1cm, 0);

\end{tikzpicture}
\end{center}
%

\subsection{Diseño optimizado}
Realizar la simulación de un circuito cuántico es una operación muy costosa. Sin 
embargo, partiendo de las mismas entradas, un circuito produce un estado 
cuántico final que no varía entre simulaciones. Debido a que al realizar la 
medición en el simulador, no se destruye el estado cuántico, es posible 
almacenarlo y reutilizarlo posteriermente.

De este modo, tan sólo será necesario realizar la ejecución del circuito 
cuántico para obtener el estado final; y posteriormente, emplear dicho estado en 
las mediciones sucesivas. El esquema optimizado de la simulación, se puede 
apreciar en el siguiente diagrama.

\begin{center}
% Define block styles
\tikzstyle{decision} = [diamond, draw, text width=4.5em, text badly centered, 
inner sep=0pt]
\tikzstyle{block} = [rectangle, draw, text width=5em, text centered]
\tikzstyle{line} = [draw, decoration={markings,mark=at position 
1 with {\arrow[scale=1.5]{latex'}}}, postaction={decorate}]
%
\begin{tikzpicture}[node distance = 3cm, auto]
	% Place nodes
	\node[block] (quantum) {Simulación cuántica};
	\node[block, right of=quantum] (measure) {Medición};
	\node[block, right of=measure] (classic) {Procesado clásico};
	\node[decision, right of=classic] (end) {Fin?};
	% Draw edges
	\draw [line] (quantum.west)+(-1cm,0) -- (quantum.west);
	\draw [line] (quantum) -> (measure);
	\draw [line] (measure) -- (classic);
	\draw [line] (classic) -- (end);
	\draw [line] (end.south) |-+(0,-1em)-| node [near start, above] {No} 
(measure.south);
	\draw [line] (end.east) -- node [near start] {Sí} +(1cm, 0);

\end{tikzpicture}
\end{center}

\section{Entorno de simulación}
El simulador se ejecuta bajo unas condiciones restringidas tanto de tiempo de 
ejecución, como de memoria. Cuenta con 512 MB disponibles para almacenar estados 
y operadores. El tiempo de la simulación está limitado a cinco minutos de 
procesamiento. El equipo que realiza la simulación consta de un procesador AMD 
Sempron de 1 GHz con aproximadamente 300 millones de operaciones de punto 
flotante por segundo y se ejecuta bajo el sistema operativo GNU/Linux.

\section{Estructuras de datos}
A medida que crece el número de qubits $n$, crece el espacio necesario para 
almacenar los operadores y los estados.
Un estado de $n$ qubits, emplea un vector de $2^{n}$ amplitudes, y cada amplitud 
está representada por un número complejo.  Un operador que actúe sobre dicho 
estado, requiere de $2^{n} \cdot 2^{n} = 2^{2n}$ elementos.  Sea $\ket{\psi}$ un 
estado de $n$ qubits, $H$ un operador, y $m = 2^{n}-1$:
%
$$ \ket{\psi} = \mat{e_0 \\ e_1 \\ \vdots \\ e_m},\, H =
	\mat{ u_{00} & u_{01} & \ldots & u_{0m} \\
		u_{10} & u_{11} & \ldots & u_{1m}\\
		\vdots & \vdots & \ddots & \vdots \\
		u_{m0} & u_{m1} & \ldots & u_{mm}\\
	}
$$
%
De modo que aplicar el operador $H$ sobre el estado $\ket{\psi}$, equivale a una 
multiplicación de una matriz por un vector:
%
$$ H \ket{\psi} = \mat{ u_{00} & u_{01} & \ldots & u_{0m} \\
		u_{10} & u_{11} & \ldots & u_{1m}\\
		\vdots & \vdots & \ddots & \vdots \\
		u_{m0} & u_{m1} & \ldots & u_{mm}\\
	}
	\mat{e_0 \\ e_1 \\ \vdots \\ e_m}
=
	\mat{b_0 \\ b_1 \\ \vdots \\ b_m} = \ket{\psi'}
$$
%
Sin embargo, generalmente $\ket{\psi}$ contiene muchos elementos que son cero, 
$e_i = 0$ y no necesitan ser almacenados. Las matrices huecas aprovechan esta 
propiedad para reducir el espacio de almacenamiento.

\subsection{Matrices huecas}

Si la matriz $A$ contiene \textit{muchos} elementos nulos, se considera una 
matriz \textit{hueca}. Aprovechar esta propiedad permite: por una parte reducir 
el número de elementos que es necesario almacenar, y por otra, reducir el número 
de sumas y multiplicaciones al operar con la matriz.
$$ A = \mat{ a_{11} & \cdots & a_{1n} \\
		\vdots &   & \vdots \\
		a_{m1} & \cdots & a_{mn}\\
	}
$$

Sea $N_z$ el número de elementos no nulos de $A$. Entonces la \textit{densidad} 
de una matriz se define como dicho número entre el total, $N_z/mn$.

\subsubsection{Almacenamiento por coordenadas COO}

La matriz $A \in M^{m\times n}$ puede almacenarse en tres vectores $\V a$, $\V r$ y 
$\V c$; de forma que $\V a$ contiene todos los elementos no nulos junto con $\V r$ y 
$\V c$ que almacenan los índices de la fila y columna respectivamente. Este 
esquema se conoce como almacenamiento por coordenadas (COO).

Sea $S_e$ y $S_i$ el tamaño de almacenamiento de un elemento de la matriz y de 
un índice respectivamente, entonces el espacio necesario es
$$ S_{COO} = S_eN_z + 2S_iN_z$$
En comparación con la representación íntegra de la matriz, se conseguiría una 
reducción del espacio si
$$ N_z/mn < \frac{S_e}{S_e + 2S_i} $$

\begin{ejemplo}
La matriz $A$ de $5\times5$ elementos, contiene 10 no nulos, de un total de 25, 
que se almacenan en el vector $\V a$. A cada uno le corresponde un índice en $\V r$ 
que indica la fila, y otro en $\V c$ con la columna. El elemento 8 se encuenta en 
$\V a_7$.  Entonces $\V r_7$ indica la tercera fila en la matriz, y $\V c_7$ la 
tercera columna.
$$ A = \mat{
	1 & 0 & 0 & 2 & 0 \\
	3 & 4 & 0 & 5 & 0 \\
	9 & 0 & \textbf{8} & 0 & 0 \\
	0 & 0 & 6 & 5 & 0 \\
	0 & 0 & 0 & 0 & 4 }
\quad
\begin{aligned}
	\V a = \arr{ 1 & 2 & 3 & 4 & 5 & 9 & \textbf{8} & 6 & 5 & 4} \\
	\V r = \arr{ 1 & 1 & 2 & 2 & 2 & 3 & \textbf{3} & 4 & 4 & 5} \\
	\V c = \arr{ 1 & 4 & 1 & 2 & 4 & 1 & \textbf{3} & 3 & 4 & 5}
\end{aligned}
$$
\end{ejemplo}

\subsubsection{Almacenamiento por filas comprimidas (CSR)}

El almacenamiento por coordenadas contiene información redundante. Los elementos 
de la misma fila repiten el mismo índice en $\V r$. Para reducir el espacio, se 
puede emplear sólo el índice en $\V a$ del primer elemento de cada fila.

De este modo, $\V a$ almacena los valores no nulos, ordenados por filas. El vector 
$\V j$ mantendrá las posiciones de las columnas para cada elemento de $\V a$, y 
finalmente, $\V i$, el índice en $\V a$ del primer elemento de la fila.  
Adicionalmente $\V i$ termina con un elemento extra, que indica el final de la 
última fila, con valor $N_z + 1$.

El tamaño necesario para almacenar una matriz de $m$ filas en CSR es
$$ S_{CSR} = S_e N_z + S_i (N_z + m + 1) $$
El espacio necesario sería menor que en COO, si $m + 1 < N_z$.

\begin{ejemplo} En la misma matriz:

$$ A = \mat{
	1 & 0 & 0 & 2 & 0 \\
	3 & 4 & 0 & 5 & 0 \\
	9 & 0 & \textbf{8} & 0 & 0 \\
	0 & 0 & 6 & 5 & 0 \\
	0 & 0 & 0 & 0 & 4 }
\quad
\begin{aligned}
	\V a = \arr{ 1 & 2 & 3 & 4 & 5  & 9 & \textbf{8} & 6 & 5 & 4} \\
	\V j = \arr{ 1 & 4 & 1 & 2 & 4  & 1 & \textbf{3} & 3 & 4 & 5}\\
	\V i = \arr{ 1 & 3 & \textbf{6} & 8 & 10 & 11}
\end{aligned}
$$

Para acceder al elemento $A_{33}$, primero se calcula la posición en $\V a$ del 
elemento que comienza la tercera fila, este índice es $\V i_3 = 6$. La tercera 
fila se sitúa en $\V a$ desde $\V i_3 = 6$ hasta $\V i_{3+1}-1 = 7$.  Para determinar 
la columna se recorren los índices de $\V j$ desde 6 hasta 7, buscando la columna 
3, que se encuentra en $\V j_7 = 3$. De modo que el elemento buscado se encuentra 
en $\V a_7 = 8$.

\subsection{Operadores secuenciales}
Un circuito cuántico, que contiene varios operadores que se aplican de forma 
secuencial, es equivalente a un único operador que realiza todo el proceso. Este 
operador $C$ sería el resultado de multiplicar las matrices que conforman las 
operaciones en cada etapa, en el mismo orden. En el caso del algoritmo de Simon:
%
$$ C = (H^{\otimes n} \otimes I^{\otimes n}) U_f (H^{\otimes n} \otimes 
I^{\otimes n})$$
%
Calcular el producto de las matrices es computacionalmente muy costoso, incluso 
prohibitivo para un $n$ grande. Adicionamente, los operadores que sólo actúan 
sobre una línea, como $(H^{\otimes n} \otimes I^{\otimes n})$, y en general 
cualquier operador del tipo $(A \otimes I^{\otimes n})$ permiten una 
optimización en tiempo y espacio.

\subsubsection{Semi-operadores}
Un \textit{semi-operador} es un operador $D$ que sólo actúa en la mitad superior
de un circuito. De modo que se puede expresar como $D = (A \otimes I^{\otimes 
n})$.
Sea $B = I^{\otimes n}$, y tanto $A$ como $B$ sean matrices de $2^n\times 2^n$, 
los elementos de $D = A \otimes B$ se determinan como
\begin{equation}
\begin{split}
	D[i,j] &= A[i_H, j_H] B[i_L, j_L]\\
	i_H &= \floor{i/2^n} \\
	j_H &= \floor{j/2^n} \\
	i_L &= i \mod 2^n \\
	j_L &= j \mod 2^n \\
\end{split}
\end{equation}
Debido a que $B[i, j]$ es la identidad, cuando $i \neq j$, $B[i,j] = 0$ y los 
cálculos se simplifican.
%
El estado $\ket{\psi}$ de $2n$ qubits, con $m = 2^{2n}-1$, se puede descomponer 
en
$$ \ket{\psi} = \sum_{i = 0}^{m} a_i \ket{i} $$
Sea $J$ el conjunto de posiciones $i$ para las cuales $a_i \neq 0$, $ J = \{ i 
\mid a_i \neq 0\}$, entonces
$$ \ket{\psi} = \sum_{i = 0}^{m} a_i \ket{i} = \sum_{i \in J} a_i \ket{i} $$
Además, como $\ket{i}$ es un estado básico, se puede descomponer en $\ket{i} = 
\ket{i_H} \otimes \ket{i_L}$, con $i_H=\floor{i/2^n}$ y $i_L = (i \mod 2^n)$, 
obteniendo
$$ \ket{\psi} = \sum_{i \in J} a_i \ket{i_H} \otimes \ket{i_L}$$
De modo que aplicar el operador $D$ sobre $\ket{\psi}$ es
\begin{equation}
\begin{split}
D \ket{\psi} &= (A \otimes B) \ket{\psi} = (A \otimes B) \sum_{i \in J} a_i 
\ket{i_H} \otimes \ket{i_L} \\
	&= \sum_{i \in J} a_i (A \otimes B) (\ket{i_H} \otimes \ket{i_L})\\
	&= \sum_{i \in J} a_i (A \ket{i_H}) \otimes (B \ket{i_L})\\
\end{split}
\end{equation}
Dado que $B$ es el operador identidad, $B\ket{i_L} = \ket{i_L}$.
$$ D \ket{\psi} = \sum_{i \in J} a_i (A \ket{i_H}) \otimes (\ket{i_L}) $$
Pero como $\ket{i_H}$, es un estado básico, tendrá un 1 en la posición $i_H$ 
(comenzando en 0) y el resto de amplitudes serán 0. Quedando el producto $A 
\ket{i_H}$:
$$
A \mat{0 \\ \vdots \\ 1  \\ \vdots\\ 0} =
\mat{A[0,i_H] \\ \vdots \\ A[i_H,i_H] \\ \vdots \\ A[m, i_H]}
$$
Es decir, el producto será un vector con la columna $i_H$ de $A$. Representando 
dicho vector como $A[:, i_H]$,
$$
D \ket{\psi} = \sum_{i \in J} a_i (A[:, i_H] \otimes \ket{i_L})
$$
Con esta operación no se necesita calcular el producto tensorial $A\otimes B$, y 
almacenarlo en la memoria, empleando una matrix de $2^{2n} \times 2^{2n}$. Sólo 
requiere almacenar un estado cuántico de $2^{2n}$ elementos, e iterativamente 
calcular el sumatorio. Esto reduce drásticamente el espacio requerido. El 
algoritmo se detalla a continuación.
\\
\\
\begin{algorithm}[H]
\SetKw{In}{Entrada}
\SetKw{Out}{Salida}
\In{Operador $A$ y estado $\ket{\psi}$}\;
\Out{Estado $\ket{\psi'} = (A\otimes I^{\otimes n}) \ket{\psi}$}\;
$\ket{\psi'} \longleftarrow \mat{0 & 0 & \ldots & 0}^T$\;
\For{$i \in J$}
{
	$i_H \longleftarrow \floor{i/2^n}$\;
	$i_L \longleftarrow i \mod 2^n$\;
	$\ket{\psi'} \longleftarrow \ket{\psi'} + a_i (A[:, i_H] \otimes 	
	\ket{i_L}) $\;
}
\caption{Aplicar semi-operador $A\otimes I^{\otimes n}$ al estado $\ket{\psi}$.}
\label{alg:semiop}
\end{algorithm}

%Comenzar con $\ket{\psi'} := \mat{0 & 0 & \ldots & 0}^T$.
%
%Para cada $i \in J$, hacer: $\ket{\psi'} := \ket{\psi'} + a_i (A[:, i_H] 
%\otimes \ket{i_L})$
%\\
%\noindent
%Finalmente se obtiene $\ket{\psi'} = (A \otimes I^{\otimes n}) \ket{\psi}$

% Comentar que sólo es necesario computar H^n y U_f. Y que U_f es muy pequeño

\subsection{Operadores y estados}

El circuito de Simon requiere dos líneas de $n$ qubits, por lo tanto, un estado 
requiere $N=2n$ qubits, un total de $2^{2n}$ amplitudes. El entorno de 
simulación cuenta con una memoria limitada de 512 MB = $2^{29}$ bytes.

Sea $S_e = 16 = 2^4$ bytes, el tamaño de un número complejo. Un estado 
$\ket{\psi}$, empleará $2^{2n}\cdot S_e = 2^{2n+4}$ bytes si almacena todas las 
amplitudes.  Limitando el espacio de la simulación: $2^{2n+4} \leq 2^{29}$ 
entonces $2n+4 \leq 29 \implies n \leq 25/2 = 12.5$. Sólo sería posible 
almacenar estados de un tamaño de hasta $n=12$.
Al emplear las matrices huecas, el tamaño se reduce, como se muestra en la 
siguiente tabla.

\begin{table}[h]
	\centering
	\begin{tabular}{*{4}{r}}
		\toprule
		$n$ & $\log_2 \, S(\ket{\psi_1})$ & $\log_2 \, S(\ket{\psi_3})$ & $2n+4$ \\
		\midrule
		2 	& 6.64	& 6.64	& 8\\
		3 	& 8.34	& 8.60	& 10\\
		4 	& 10.17	& 10.59	& 12\\
		5 	& 12.09	& 12.59	& 14\\
		6 	& 14.04	& 14.59	& 16\\
		7 	& 16.02	& 16.59	& 18\\
		8 	& 18.01	& 18.58	& 20\\
		9 	& 20.01	& 20.58	& 22\\
		10 	& 22.00	& 22.58	& 24\\
		\bottomrule
	\end{tabular}
	\caption{Tamaño de los estados en escala logarítmica.}
\end{table}

Sin embargo, un operador requiere mucha más memoria. En el caso de una matriz 
densa de $2n \times 2n$ serán necesarios $2^{2n} \cdot 2^{2n} = 2^{4n}$ números 
complejos. De esta forma, se limita el tamaño a:
$$2^{4n+4} \leq 2^{29} \implies n \leq 25/4 = 6.25 $$
Un máximo de $n = 6$. Sin embargo, empleando la propiedad de que tanto lo 
operadores como los estado contienen un gran número de elementos nulos (son 
matrices huecas), es posible reducir en gran medida su tamaño, y ampliar $n$ 
hasta $11$.

\begin{table}[h]
	\centering
	\begin{tabular}{*{4}{r}}
		\toprule
		$n$ & $\log_2 \, S(H)$ & $\log_2 \, S(U_f)$ & $4n+4$ \\
		\midrule
		2 	& 7.21	& 7.61	& 12\\
		3 	& 9.10	& 9.59	& 16\\
		4 	& 11.05	& 11.59	& 20\\
		5 	& 13.02	& 13.59	& 24\\
		6 	& 15.01	& 15.59	& 28\\
		7 	& 17.01	& 17.58	& 32\\
		8 	& 19.00	& 19.58	& 36\\
		9 	& 21.00	& 21.58	& 40\\
		10 	& 23.00	& 23.58	& 44\\
		\bottomrule
	\end{tabular}
	\caption{Tamaño de los operadores en escala logarítmica.}
\end{table}



\begin{table}[h]
	\centering
	\begin{tabular}{*{10}{r}}
		\toprule
$n$ & 2	& 3	& 4	& 5	& 6	& 7	& 8	& 9	& 10 \\
$2n$& 4	& 6	& 8	& 10	& 12	& 14	& 16	& 18	& 20 \\
$\log_2 \, S_T$ & 9.09	& 10.99	& 12.95	& 14.93	& 16.92	& 18.91	& 20.91	& 22.91	
& 24.91 \\
		\bottomrule
	\end{tabular}
	\caption{Tamaño total de los operadores y los estados.}
\end{table}


\end{ejemplo}

\section{Implementación del simulador}

La simulación se realiza con el lenguaje de programación \texttt{python}, 
empleando paquetes externos que aportarán las partes comunes del simulador. De 
esta forma se reutiliza el software ya existente.

Para los cálculos numéricos se emplea \texttt{numpy} y \texttt{scipy}, en 
operaciones como multiplicación y suma de matrices, productos tensoriales y 
funciones matemáticas vectorizadas. Para el almacenamiento de operadores y 
estados cuánticos, se emplea el paquete \texttt{qutip}. La clase 
\texttt{qutip.Qobj} permite emplear de forma implícita las matrices huecas de 
\texttt{scipy.sparse} con el formato de almacenamiento CSR o COO.

%\section{Almacenamiento}
%El paquete \texttt{qutip} permite representar los estados cuánticos, así como 
%los operadores. Esta representación, se realiza mediante matrices huecas.
%
%De esta forma, un estado $\ket{\psi} = \ket{0000} $ se representa como una 
%matriz hueca, de $2^4 \times 1$. Un operador como Hadamard, se calcula 
%realizando las potencias tensoriales del operador de Hadamard para un qubit.  
%Almacenándose en una matriz hueca.

\subsection{Construcción de operadores}
Para la simulación del circuito cuántico, serán necesarios varios operadores.
Por una parte el semi-operador $\opn H n \otimes \opn I n$ que se empleará en 
dos ocasiones, y por otra el operador $U_f$.

\subsubsection{Operador $\opn H n \otimes \opn I n $}
El operador $\opn H n \otimes \opn I n$ es un semi-operador, debido a que no 
modifica la línea inferior. No es necesario construir íntegramente el operador 
completo, el algoritmo~\ref{alg:semiop} permite calcular el resultado partiendo 
de $\opn H n$.
%
Para obtener $\opn H n$ el paquete \texttt{qutip} incluye la operación 
\texttt{hadamard\_transform} que computa las potencias tensoriales de $H$. La 
implementación del operador $\opn H n \otimes \opn I n $ empleando el 
algoritmo~\ref{alg:semiop} se muestra a continuación.
%
%\begin{samepage}
\begin{pycode}
def HI(self, state):
	bits = self.bits

	nz = state.data.nonzero()[0]
	ai = state.data[nz,0].T.toarray()[0]
	iH = nz >> bits
	iL = nz % 2**bits

	tmp = qutip.zero_ket(2**(2*bits))
	for i in range(len(nz)):
		a = ai[i]
		ketH = self.H * ket(iH[i], bits)
		ketL = ket(iL[i], bits)
		tmp += a * tensor(ketH, ketL)

	return tmp
\end{pycode}
%
En la línea 4, se obtienen los índices \texttt{nz} de los elementos no nulos del 
estado \texttt{state} al que se aplicará el operador. Para cada índice, en la 
línea 5, se almacena en \texttt{ai} la amplitud correspondiende. A continuación, 
se calcula $i_H$ e $i_L$ para cada índice, y se inicializa el estado resultante 
a un vector de ceros con \texttt{qutip.zero\_ket}. Finalmente en las líneas de 9 
a 14 se realiza el bucle que iterativamente calcula el estado resultante  en 
\texttt{tmp}, tal y como se describe en el algoritmo~\ref{alg:semiop}.
%\end{samepage}

\subsubsection{Operador $U_f$}
\noindent
La operación $U_f$ se define, a partir de una función $f$, como
$$ U_f \ket{\V x} \ket{\V y} = \ket{\V x} \ket{\V y \oplus f(\V x)} $$
Esta operación, puede describirse como una matriz de $2^{2n} \times 2^{2n}$,
aplicada sobre el estado $\ket{\V x} \ket{\V y}$, que es un vector de $2^{2n}$ 
componentes, siendo $n$ el número de qubits de $\ket{\V x}$ y también de $\ket{
\V y}$.  Sea $m = 2^{2n} -1$, se tiene
$$ U_f \ket{\V x} \ket{\V y} = \mat{
	u_{00} & \ldots & u_{0m} \\
	\vdots &        & \vdots \\
	u_{m0} & \ldots & u_{mm} \\
}
\mat{
	a_{0} \\ \vdots \\ a_{m} \\
}
=
\mat{
	b_{0} \\ \vdots \\ b_{m} \\
}
$$
El estado $\ket{\V x} \ket{\V y}$ se puede escribir también como
%
$$ \ket{\V x} \ket{\V y} = \sum_{k=0}^m a_k \ket{k_H}\ket{k_L} $$
%
% TODO Indicar que \ket{110} se puede escrbir como \ket{k} con k = 6
%
Siendo $k_H = \floor{k/2^n}$ y $k_L = k \mod 2^n$. De modo que al aplicar $U_f$,
%
$$ U_f \ket{\V x} \ket{\V y} = \sum_{k=0}^m a_k\ U_f \ket{k_H}\ket{k_L} =
\sum_{k=0}^m a_k \ket{k_H}\ket{k_L \oplus f(k_H)} $$
%
El estado $\ket{k_H} \ket{k_L}$ es un vector de $2^{2n}$ componentes, con un 1 
en $k = k_H \cdot 2^n + k_L$ y con el resto 0; el estado $\ket{k_H}\ket{k_L 
\oplus f(k_H)}$ en $k' = k_H \cdot 2^n + k_L \oplus f(k_H)$.  Se observa que la 
operación $U_f$ toma la componente $a_k$ en la posición $k$ y la coloca en la 
posición $k'$.  Es decir, se trata de una matriz de permutación. Para realizar 
esta transformación, basta con colocar un 1 en la columna $k$ y en la fila $k'$.  
De esta forma se define $U_f$ como:
%
\begin{equation*}
U_f[i,j] =
\begin{cases}
1, & \text{si}\ i = k',\ j = k \\
0, & \text{en otro caso}
\end{cases}
\end{equation*}
Para $0 \leq i,j \leq m$ con $k = j$ y $k' = k_H \cdot 2^n + k_L \otimes 
f(k_H)$.

%
%\begin{samepage}
%
% TODO: Revisar condiciones del tipo 0 < i < n, ya que deben ser 0 <= i < n
\noindent
El algoritmo de construcción del operador $U_f$ se detalla a continuación
%
\begin{pycode}
def build_U(self, f):
	n = self.bits
	m = 2 ** (2 * n) - 1
	k = np.arange(0, m + 1)
	kH = k >> n
	kL = k % 2 ** n
	_kH = kH
	_kL = f[kH] ^ kL
	_k = _kH << n | _kL
	data = [1]*n
	U = coo.coo_matrix((data, (_k, k)), shape=[n, n])
	return qutip.Qobj(U)
\end{pycode}
%
La operación de desplazamiento de bits $k_H = k \gg n$ de la línea 5, equivale a 
una división entera $k_H = \floor{k/2^n}$.

El funcionamiento del algoritmo, consiste en calcular las posiciones $k$ y $k'$, 
para $0 \leq k \leq m$, como se observa en la línea 4. Luego se calcula $k'$ 
aplicando la función $f$ y la suma $\oplus$, en las líneas 8 y 9. Finalmente en 
la línea 11 se construye la matriz $U_f$ indicando sólo las posiciones en las 
que se encuentran los unos, las filas $k'$ y columnas $k$. El resto de la matriz 
contiene elementos nulos, que no es necesario almacenar, ahorrando un espacio 
considerable.
%\end{samepage}

\subsection{Circuito cuántico}
Tras haber construido el operador $U_f$, y empleando el 
algoritmo~\ref{alg:semiop} para calcular $\opn H n \otimes \opn I n$, el código 
que ejecuta la simulación se muestra a continuación
%$$(\opn H n \otimes \opn I n)\ U_f\ (\opn H n \otimes \opn I n)$$
\begin{pycode}
def simon(self, bits, f):
	x = ket(0, bits)
	y = ket(0, bits)

	self.U = self.build_U(f)

	phi0 = tensor(x, y)
	phi1 = self.HI(phi0)
	phi2 = self.U * phi1
	phi3 = self.HI(phi2)

	return phi3
\end{pycode}

Los estados $\ket{\V x}$ y $\ket{\V y}$ se inicializan a cero en las líneas 2 y 
3. En la línea 5 se calcula el operador $U_f$ partiendo de la función $f$ dada.  
Luego se obtiene el producto tensorial de $\ket{\V x}$ y $\ket{\V y}$ que será 
el estado inicial $\ketp 0 $ en la línea 7. Finalmente se aplican los operadores 
$\opn H n \otimes \opn I n$, $U_f$ y otra vez $\opn H n \otimes \opn I n$, tal y 
como describe el circuito de Simon. El estado final $\ketp3$ es devuelto para 
ser almacenado.
