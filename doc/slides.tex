\input{preamble2}
% Propiedades
\title{Diseño, desarrollo, implementación y prueba de un simulador cuántico para 
el algoritmo de Simon}

\author{Rodrigo Arias Mallo (rodrigo.arias@udc.es)}

\begin{document}

%\maketitle

%%%%%%%%%%%%%%%%%%%%%%%%%%%%%%%%%%%%%%%%%%%%%%%%%%%%%%%%%%%%%%%%%%%%%%%%%%%%%%%
%\clearpage 

%\tableofcontents

%\clearpage 

% Diapositiva inicial
\frame{\titlepage}

%%%%%%%%%%%%%%%%%%%%%%%%%%%%%%%%%%%%%%%%%%%%%%%%%%%%%%%%%%%%%%%%%%%%%%%%%%%%%%%
\begin{frame}
\frametitle{Introducción}

\begin{itemize}
\item Que es esto?
\item Por que?
\end{itemize}

\end{frame}
%%%%%%%%%%%%%%%%%%%%%%%%%%%%%%%%%%%%%%%%%%%%%%%%%%%%%%%%%%%%%%%%%%%%%%%%%%%%%%%
\begin{frame}
\frametitle{Problema de Simon}
Sea $f$ una función binaria
$$f:\{0,1\}^n \rightarrow \{0,1\}^n$$
que cumple la propiedad
$$ f(\V x) = f(\V y) \iff \V x = \V y \oplus \V s$$
con $\V x, \V y, \V s \in \{0,1\}^n$ y el período $\V s \neq \V 0$.

\vspace{1cm}

\textbf{Objetivo: }Encontrar $\V s$ partiendo de la función $f$.


\end{frame}
%%%%%%%%%%%%%%%%%%%%%%%%%%%%%%%%%%%%%%%%%%%%%%%%%%%%%%%%%%%%%%%%%%%%%%%%%%%%%%%
\begin{frame}
\frametitle{Función $f$ de ejemplo}

Ejemplo de 2 bits, y un período $\V s = 01$

\begin{center}
\begin{tabular}{cccc}
	\toprule
	$\V x$ & $f(\V x)$ & $\V x \oplus \V s$ & $f(\V x \oplus \V s)$\\
	\midrule
	00 & 00 & 01 & 00\\
	01 & 00 & 00 & 00\\
	10 & 01 & 11 & 01\\
	11 & 01 & 10 & 01\\
	\bottomrule
\end{tabular}
\end{center}

$$ f(\V x) = f(\V y) \iff \V x = \V y \oplus \V s$$

\end{frame}
%%%%%%%%%%%%%%%%%%%%%%%%%%%%%%%%%%%%%%%%%%%%%%%%%%%%%%%%%%%%%%%%%%%%%%%%%%%%%%%
\begin{frame}
\frametitle{Solución clásica}
\begin{itemize}
\item Una solución sencilla consiste en probar entradas hasta obtener una salida 
repetida.

\item Hay $2^n/2$ salidas diferentes.

\item Es necesario probar al menos $2^n/2 + 1$ entradas: $O(2^n)$.
\end{itemize}
\end{frame}
%%%%%%%%%%%%%%%%%%%%%%%%%%%%%%%%%%%%%%%%%%%%%%%%%%%%%%%%%%%%%%%%%%%%%%%%%%%%%%%
\begin{frame}
\frametitle{Solución cuántica: $O(2^n) \rightarrow O(n)$}
\begin{itemize}
\item Para solucionar este problema, Daniel R. Simon propuso una solución 
empleando la \textbf{computación cuántica}.
\item Soluciona el problema en $O(n)$.
\end{itemize}

\end{frame}
%%%%%%%%%%%%%%%%%%%%%%%%%%%%%%%%%%%%%%%%%%%%%%%%%%%%%%%%%%%%%%%%%%%%%%%%%%%%%%%
\begin{frame}
\frametitle{Introducción a la computación cuántica}
\begin{itemize}
\item La computación cuántica emplea las propiedades de la materia a escala 
atómica.
\item El comportamiento de la materia se describe mediante los postulados de la 
mecánica cuántica
\end{itemize}
\end{frame}
%%%%%%%%%%%%%%%%%%%%%%%%%%%%%%%%%%%%%%%%%%%%%%%%%%%%%%%%%%%%%%%%%%%%%%%%%%%%%%%
\begin{frame}
\frametitle{El bit}
Para almacenar la información clásica se emplea el bit, que puede representarse 
mediante un vector $\V x$.
$$ \V x = \mat{p \\ q} $$
Con  $p, q \in \{0,1\}$ y cumpliendo la restricción $p \oplus q = 1$.

Se encuentra en uno de los dos estados posibles
$$ \text{bit 0} = \mat{1 \\ 0},\quad \text{bit 1} = \mat{0 \\ 1} $$



\end{frame}
%%%%%%%%%%%%%%%%%%%%%%%%%%%%%%%%%%%%%%%%%%%%%%%%%%%%%%%%%%%%%%%%%%%%%%%%%%%%%%%
\begin{frame}
\frametitle{El bit probabilístico}
%
El estado de moneda perfecta al ser lanzada al aire, se puede describir mediante 
un vector $\V y$.
$$ \V y = \mat{1/2 \\ 1/2} = \mat{a \\ b} $$
Con $a,b \in \mathbb R$ y cumpliendo la restricción $a + b = 1$.
Siendo $a$ la probabilidad de que salga cara, y $b$ la de que salga cruz.  Al 
caer:
$$ \V y_{\text{cara}} = \mat{1 \\ 0},\quad
\V y_{\text{cruz}} = \mat{0 \\ 1}$$
\end{frame}
%%%%%%%%%%%%%%%%%%%%%%%%%%%%%%%%%%%%%%%%%%%%%%%%%%%%%%%%%%%%%%%%%%%%%%%%%%%%%%%
\begin{frame}
\frametitle{El bit cuántico o qubit}
El estado de un sistema cuántico también se puede representar mediante un vector 
$\V z$.
$$ \V z = \mat{\alpha \\ \beta} $$
Con $\alpha, \beta \in \mathbb C$ y cumpliendo la restricción $|\alpha|^2 + 
|\beta|^2 = 1$. La probabilidad de que salga 0 es $|\alpha|^2$ y de que salga 1 
es $|\beta|^2$. Al medir:
$$ \V z_{\text{cero}} = \mat{1 \\ 0},\quad
\V z_{\text{uno}} = \mat{0 \\ 1}$$


\end{frame}
%%%%%%%%%%%%%%%%%%%%%%%%%%%%%%%%%%%%%%%%%%%%%%%%%%%%%%%%%%%%%%%%%%%%%%%%%%%%%%%
\begin{frame}
\frametitle{Notación de Dirac}

\end{frame}
%%%%%%%%%%%%%%%%%%%%%%%%%%%%%%%%%%%%%%%%%%%%%%%%%%%%%%%%%%%%%%%%%%%%%%%%%%%%%%%

\end{document}
