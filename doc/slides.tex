\documentclass{beamer}

% Fórmulas matemáticas con fuente adecuada
\usefonttheme[onlymath]{serif}

%\documentclass[12pt,a4paper]{report}
\usepackage{amsfonts}
\usepackage{amsmath}
\usepackage{amsthm}
\usepackage[utf8]{inputenc}
\usepackage{braket}

% Código fuente
%\usepackage{fontspec}
\usepackage{minted}
\newminted{py}{%
		linenos,
		fontsize=\small,
		tabsize=4,
		mathescape,
}

% Tamaño de los números de la las líneas en el código fuente
\renewcommand\theFancyVerbLine{\footnotesize\arabic{FancyVerbLine}}

% Emplear "tabla" y no "cuadro" para las tablas.
% Romper líneas por sílabas en español
\usepackage[spanish,es-nosectiondot,es-tabla]{babel}

% Usar estilo de libro para las tablas
\usepackage{booktabs}

% Gráficas
\usepackage{pgfplots}

% Emplear un tamaño más pequeño para las etiquetas
\usepackage[font=footnotesize]{caption}

% Cargar tablas de csv
\usepackage{pgfplotstable}

% Urls
\usepackage{url}

% Algoritmos
\usepackage[spanish, ruled]{algorithm2e}

% Imágenes vectoriales en SVG
\usepackage{svg}

% Dedicatoria
\newenvironment{dedication}{
\thispagestyle{empty}% no header and footer
\cleardoublepage           % we want a new page
\thispagestyle{empty}% no header and footer
\vspace*{\stretch{1}}% some space at the top 
\itshape             % the text is in italics
\raggedleft          % flush to the right margin
}
{\par % end the paragraph
\vspace{\stretch{3}} % space at bottom is three times that at the top
\clearpage           % finish off the page
%\cleardoublepage
}

% Dimensiones de los márgenes.
%\usepackage[margin=3cm]{geometry}
%\usepackage[top=3cm, bottom=3cm, left=3.5cm, right=2.5cm]{geometry}

% Macros para ayudar a la redacción
% Vector
\newcommand*\mat[1]{ \begin{pmatrix} #1 \end{pmatrix}}
\newcommand*\arr[1]{ \begin{bmatrix} #1 \end{bmatrix}}
\newcommand*\V[1]{ \boldsymbol{#1}}
% Ket psi
\newcommand*\ketp[1]{\ket{\psi_{#1}}}
% Floor
\newcommand{\floor}[1]{\lfloor #1 \rfloor}
% Absolute value
\newcommand\abs[1]{\left|#1\right|}

% Probabilidad I y ¬I
\newcommand*\NO[1]{ \overline{#1}}
\newcommand*\NI[1]{ \overline{I^{#1}}\,}

% Operador identidad I^{\otimes n}
\newcommand*\opn[2]{{#1}^{\otimes #2}}

% Producto pensorial
\newcommand\ox[2]{\ket{#1} \otimes \ket{#2}}

% Colores
\newcommand*\cred[1]{{\color{red} {#1}}}
\newcommand*\cgre[1]{{\color{green} {#1}}}

% Apartado de un ejemplo
\theoremstyle{definition}
\newtheorem{ejemplo}{Ejemplo}[section]

% Gráficos con tikz
\usepackage{tikz}

\usetikzlibrary{calc,fadings,decorations.pathreplacing}
\usetikzlibrary{backgrounds,fit}

\usetikzlibrary{shapes,arrows,chains}
\usetikzlibrary{decorations.markings}

\newcommand\pgfmathsinandcos[3]{%
  \pgfmathsetmacro#1{sin(#3)}%
  \pgfmathsetmacro#2{cos(#3)}%
}
\newcommand\LongitudePlane[3][current plane]{%
  \pgfmathsinandcos\sinEl\cosEl{#2} % elevation
  \pgfmathsinandcos\sint\cost{#3} % azimuth
  \tikzset{#1/.estyle={cm={\cost,\sint*\sinEl,0,\cosEl,(0,0)}}}
}
\newcommand\LatitudePlane[3][current plane]{%
  \pgfmathsinandcos\sinEl\cosEl{#2} % elevation
  \pgfmathsinandcos\sint\cost{#3} % latitude
  \pgfmathsetmacro\yshift{\cosEl*\sint}
  \tikzset{#1/.estyle={cm={\cost,0,0,\cost*\sinEl,(0,\yshift)}}} %
}
\newcommand\DrawLongitudeCircle[2][1]{
  \LongitudePlane{\angEl}{#2}
%  \tikzset{current plane/.prefix style={scale=#1}}
   % angle of "visibility"
  \pgfmathsetmacro\angVis{atan(sin(#2)*cos(\angEl)/sin(\angEl))} %
  \draw[current plane] (\angVis:1) arc (\angVis:\angVis+180:1);
  \draw[current plane,dashed] (\angVis-180:1) arc (\angVis-180:\angVis:1);
}
\newcommand\DrawLatitudeCircle[2][1]{
  \LatitudePlane{\angEl}{#2}
%	\tikzset{current plane/.prefix style={scale=#1}}
  \pgfmathsetmacro\sinVis{sin(#2)/cos(#2)*sin(\angEl)/cos(\angEl)}
  % angle of "visibility"
  \pgfmathsetmacro\angVis{asin(min(1,max(\sinVis,-1)))}
  \draw[current plane] (\angVis:1) arc (\angVis:-\angVis-180:1);
  \draw[current plane,dashed] (180-\angVis:1) arc (180-\angVis:\angVis:1);
}

% Numerar las diapositivas
\beamertemplatenavigationsymbolsempty

\setbeamerfont{page number in head/foot}{size=\large}
\setbeamertemplate{footline}[frame number]


% Propiedades
\title{Diseño, desarrollo, implementación y prueba de un simulador cuántico para 
el algoritmo de Simon}

\author{\textbf{Autor:} Rodrigo Arias Mallo \\
\textbf{Director:} Vicente Moret Bonillo}

\begin{document}

%\maketitle

%%%%%%%%%%%%%%%%%%%%%%%%%%%%%%%%%%%%%%%%%%%%%%%%%%%%%%%%%%%%%%%%%%%%%%%%%%%%%%%
%\clearpage 

%\tableofcontents

%\clearpage 

% Diapositiva inicial
\frame{\titlepage}

%%%%%%%%%%%%%%%%%%%%%%%%%%%%%%%%%%%%%%%%%%%%%%%%%%%%%%%%%%%%%%%%%%%%%%%%%%%%%%%

\begin{frame}
\frametitle{Introducción}

\textbf{Resumen:} Resolución de un problema mediante un algoritmo cuántico.  
Simulación y comprobación de los resultados.

\tableofcontents
\begin{itemize}
\item Presentación del problema
\item Solución clásica
\item Solución cuántica
\item Complejidad teórica
\item Simulador
\item Resultados experimentales
\item Conclusiones y trabajo futuro
\end{itemize}

\end{frame}

%%%%%%%%%%%%%%%%%%%%%%%%%%%%%%%%%%%%%%%%%%%%%%%%%%%%%%%%%%%%%%%%%%%%%%%%%%%%%%%

\begin{frame}
\frametitle{Problema de Simon}

Sea $f$ una función binaria
$$f:\{0,1\}^n \rightarrow \{0,1\}^n$$
que cumple la propiedad
$$ f(\V x) = f(\V y) \iff \V y = \V x \oplus \V s$$
con $\V x, \V y, \V s \in \{0,1\}^n$ y el período $\V s \neq \V 0$.

\vspace{1cm}

\textbf{Objetivo: }Encontrar $\V s$ partiendo de la función $f$.


\end{frame}
%%%%%%%%%%%%%%%%%%%%%%%%%%%%%%%%%%%%%%%%%%%%%%%%%%%%%%%%%%%%%%%%%%%%%%%%%%%%%%%
\begin{frame}
\frametitle{Función $f$ de ejemplo}

Ejemplo de 2 bits, y un período $\V s = 01$

\begin{center}
\begin{tabular}{cccc}
	\toprule
	$\V x$ & $f(\V x)$ & $\V x \oplus \V s$ & $f(\V x \oplus \V s)$\\
	\midrule
	00 & 00 & 01 & 00\\
	01 & 00 & 00 & 00\\
	10 & 01 & 11 & 01\\
	11 & 01 & 10 & 01\\
	\bottomrule
\end{tabular}
\end{center}

$$ f(\V x) = f(\V y) \iff \V y = \V x \oplus \V s$$

\end{frame}
%%%%%%%%%%%%%%%%%%%%%%%%%%%%%%%%%%%%%%%%%%%%%%%%%%%%%%%%%%%%%%%%%%%%%%%%%%%%%%%
\begin{frame}
\frametitle{Solución clásica}
\begin{itemize}
\item Una solución sencilla consiste en probar entradas hasta obtener una salida 
repetida.

\item Hay $2^n/2$ salidas diferentes.

\item Es necesario probar al menos $2^n/2 + 1$ entradas: $O(2^n)$.
\end{itemize}
\end{frame}
%%%%%%%%%%%%%%%%%%%%%%%%%%%%%%%%%%%%%%%%%%%%%%%%%%%%%%%%%%%%%%%%%%%%%%%%%%%%%%%
\begin{frame}
\frametitle{Solución cuántica: $O(2^n) \rightarrow O(n)$}
\begin{itemize}
\item Para solucionar este problema, Daniel R. Simon propuso una solución 
empleando la \textbf{computación cuántica}.
\item Soluciona el problema en $O(n)$.
\end{itemize}

\begin{center}
	\begin{tikzpicture}%[thick]
	% `operator' will only be used by Hadamard (H) gates here.
	\tikzstyle{operator} = [draw,fill=white,minimum size=1.5em] 
	%
	\matrix[row sep=0.4cm, column sep=1cm] (circuit) {
		% First row
		\node (q1) {$\ket{00}$}; &
		\node[operator] (H11) {$H^{\otimes 2}$}; &
		\node[operator] (P13) {}; &
		\node[operator] (H11) {$H^{\otimes 2}$}; &
		\node[operator] (M11) {$M$};&
		\coordinate (end1); \\
		% Second row.
		\node (q2) {$\ket{00}$}; & & \node[operator] (P23) {}; & & & \coordinate 
(end2);\\
		% Third row
		\node (q31) {}; & \node (q32) {}; & \node (q33) {}; &
		\node (q34) {}; & \node (q35) {}; & \node (q36) {}; \\
		\node (q41) {}; & \node (q42) {}; & \node (q43) {}; &
		\node (q44) {}; & \node (q45) {}; & \node (q46) {}; \\
	};
	\node[operator] (Uf) [fit = (P13) (P23), minimum width=1cm] {$U_f$};

	\node (arr0) [fit = (q31) (q32)] {$\uparrow$};
	\node (arr1) [fit = (q32) (q33)] {$\uparrow$};
	\node (arr2) [fit = (q33) (q34)] {$\uparrow$};
	\node (arr3) [fit = (q34) (q35)] {$\uparrow$};

	\node (psi0) [fit = (q41) (q42)] {$\ket{\psi_0}$};
	\node (psi1) [fit = (q42) (q43)] {$\ket{\psi_1}$};
	\node (psi2) [fit = (q43) (q44)] {$\ket{\psi_2}$};
	\node (psi3) [fit = (q44) (q45)] {$\ket{\psi_3}$};

	\node[fill=white, fit=(end1) (end2)] (cover) {};

	\begin{pgfonlayer}{background}
		% Draw lines.
		\draw[thick] (q1) -- (M11)  (q2) -- (end2);
		\draw[thick, double, double distance=2pt] (M11) -- (end1);
	\end{pgfonlayer}
	%
\end{tikzpicture}
\end{center}


\end{frame}
%%%%%%%%%%%%%%%%%%%%%%%%%%%%%%%%%%%%%%%%%%%%%%%%%%%%%%%%%%%%%%%%%%%%%%%%%%%%%%%
\begin{frame}
\frametitle{Funcionamiento}
\begin{center}
	\begin{tikzpicture}%[thick]
	% `operator' will only be used by Hadamard (H) gates here.
	\tikzstyle{operator} = [draw,fill=white,minimum size=1.5em] 
	%
	\matrix[row sep=0.4cm, column sep=1cm] (circuit) {
		% First row
		\node (q1) {$\ket{00}$}; &
		\node[operator] (H11) {$H^{\otimes 2}$}; &
		\node[operator] (P13) {}; &
		\node[operator] (H11) {$H^{\otimes 2}$}; &
		\node[operator] (M11) {$M$};&
		\coordinate (end1); \\
		% Second row.
		\node (q2) {$\ket{00}$}; & & \node[operator] (P23) {}; & & & \coordinate 
(end2);\\
		% Third row
		\node (q31) {}; & \node (q32) {}; & \node (q33) {}; &
		\node (q34) {}; & \node (q35) {}; & \node (q36) {}; \\
		\node (q41) {}; & \node (q42) {}; & \node (q43) {}; &
		\node (q44) {}; & \node (q45) {}; & \node (q46) {}; \\
	};
	\node[operator] (Uf) [fit = (P13) (P23), minimum width=1cm] {$U_f$};

	\node (arr0) [fit = (q31) (q32)] {$\uparrow$};
	\node (arr1) [fit = (q32) (q33)] {$\uparrow$};
	\node (arr2) [fit = (q33) (q34)] {$\uparrow$};
	\node (arr3) [fit = (q34) (q35)] {$\uparrow$};

	\node (psi0) [fit = (q41) (q42)] {$\ket{\psi_0}$};
	\node (psi1) [fit = (q42) (q43)] {$\ket{\psi_1}$};
	\node (psi2) [fit = (q43) (q44)] {$\ket{\psi_2}$};
	\node (psi3) [fit = (q44) (q45)] {$\ket{\psi_3}$};

	\node[fill=white, fit=(end1) (end2)] (cover) {};

	\begin{pgfonlayer}{background}
		% Draw lines.
		\draw[thick] (q1) -- (M11)  (q2) -- (end2);
		\draw[thick, double, double distance=2pt] (M11) -- (end1);
	\end{pgfonlayer}
	%
\end{tikzpicture}
\end{center}

Primero se inicializa el circuito con ceros.
$$ \ket{\psi_0} = \ket{00,00} $$
\end{frame}
%%%%%%%%%%%%%%%%%%%%%%%%%%%%%%%%%%%%%%%%%%%%%%%%%%%%%%%%%%%%%%%%%%%%%%%%%%%%%%%
\begin{frame}
\frametitle{Funcionamiento}
\begin{center}
	\begin{tikzpicture}%[thick]
	% `operator' will only be used by Hadamard (H) gates here.
	\tikzstyle{operator} = [draw,fill=white,minimum size=1.5em] 
	%
	\matrix[row sep=0.4cm, column sep=1cm] (circuit) {
		% First row
		\node (q1) {$\ket{00}$}; &
		\node[operator] (H11) {$H^{\otimes 2}$}; &
		\node[operator] (P13) {}; &
		\node[operator] (H11) {$H^{\otimes 2}$}; &
		\node[operator] (M11) {$M$};&
		\coordinate (end1); \\
		% Second row.
		\node (q2) {$\ket{00}$}; & & \node[operator] (P23) {}; & & & \coordinate 
(end2);\\
		% Third row
		\node (q31) {}; & \node (q32) {}; & \node (q33) {}; &
		\node (q34) {}; & \node (q35) {}; & \node (q36) {}; \\
		\node (q41) {}; & \node (q42) {}; & \node (q43) {}; &
		\node (q44) {}; & \node (q45) {}; & \node (q46) {}; \\
	};
	\node[operator] (Uf) [fit = (P13) (P23), minimum width=1cm] {$U_f$};

	\node (arr0) [fit = (q31) (q32)] {$\uparrow$};
	\node (arr1) [fit = (q32) (q33)] {$\uparrow$};
	\node (arr2) [fit = (q33) (q34)] {$\uparrow$};
	\node (arr3) [fit = (q34) (q35)] {$\uparrow$};

	\node (psi0) [fit = (q41) (q42)] {$\ket{\psi_0}$};
	\node (psi1) [fit = (q42) (q43)] {$\ket{\psi_1}$};
	\node (psi2) [fit = (q43) (q44)] {$\ket{\psi_2}$};
	\node (psi3) [fit = (q44) (q45)] {$\ket{\psi_3}$};

	\node[fill=white, fit=(end1) (end2)] (cover) {};

	\begin{pgfonlayer}{background}
		% Draw lines.
		\draw[thick] (q1) -- (M11)  (q2) -- (end2);
		\draw[thick, double, double distance=2pt] (M11) -- (end1);
	\end{pgfonlayer}
	%
\end{tikzpicture}
\end{center}

Se aplica el operador de Hadamard sobre la línea superior, obteniendo un estado 
entrelazado.
$$
\ket{\psi_1} = \frac{1}{2} \Big(\ket{00,00} + \ket{01,00} + \ket{10,00} + 
\ket{11,00} \Big)
$$
\end{frame}
%%%%%%%%%%%%%%%%%%%%%%%%%%%%%%%%%%%%%%%%%%%%%%%%%%%%%%%%%%%%%%%%%%%%%%%%%%%%%%%
\begin{frame}
\frametitle{Funcionamiento}
\begin{center}
	\begin{tikzpicture}%[thick]
	% `operator' will only be used by Hadamard (H) gates here.
	\tikzstyle{operator} = [draw,fill=white,minimum size=1.5em] 
	%
	\matrix[row sep=0.4cm, column sep=1cm] (circuit) {
		% First row
		\node (q1) {$\ket{00}$}; &
		\node[operator] (H11) {$H^{\otimes 2}$}; &
		\node[operator] (P13) {}; &
		\node[operator] (H11) {$H^{\otimes 2}$}; &
		\node[operator] (M11) {$M$};&
		\coordinate (end1); \\
		% Second row.
		\node (q2) {$\ket{00}$}; & & \node[operator] (P23) {}; & & & \coordinate 
(end2);\\
		% Third row
		\node (q31) {}; & \node (q32) {}; & \node (q33) {}; &
		\node (q34) {}; & \node (q35) {}; & \node (q36) {}; \\
		\node (q41) {}; & \node (q42) {}; & \node (q43) {}; &
		\node (q44) {}; & \node (q45) {}; & \node (q46) {}; \\
	};
	\node[operator] (Uf) [fit = (P13) (P23), minimum width=1cm] {$U_f$};

	\node (arr0) [fit = (q31) (q32)] {$\uparrow$};
	\node (arr1) [fit = (q32) (q33)] {$\uparrow$};
	\node (arr2) [fit = (q33) (q34)] {$\uparrow$};
	\node (arr3) [fit = (q34) (q35)] {$\uparrow$};

	\node (psi0) [fit = (q41) (q42)] {$\ket{\psi_0}$};
	\node (psi1) [fit = (q42) (q43)] {$\ket{\psi_1}$};
	\node (psi2) [fit = (q43) (q44)] {$\ket{\psi_2}$};
	\node (psi3) [fit = (q44) (q45)] {$\ket{\psi_3}$};

	\node[fill=white, fit=(end1) (end2)] (cover) {};

	\begin{pgfonlayer}{background}
		% Draw lines.
		\draw[thick] (q1) -- (M11)  (q2) -- (end2);
		\draw[thick, double, double distance=2pt] (M11) -- (end1);
	\end{pgfonlayer}
	%
\end{tikzpicture}
\end{center}

El operador $U_f$ realiza el cómputo de $f$ en paralelo.
$$
	\ket{\psi_2} = \frac{1}{2} \Big(\ket{00,f(00)} + \ket{01,f(01)} + 
\ket{10,f(10)} + \ket{11,f(11)}\Big)
$$
\end{frame}
%%%%%%%%%%%%%%%%%%%%%%%%%%%%%%%%%%%%%%%%%%%%%%%%%%%%%%%%%%%%%%%%%%%%%%%%%%%%%%%
\begin{frame}
\frametitle{Funcionamiento}
\begin{center}
	\begin{tikzpicture}%[thick]
	% `operator' will only be used by Hadamard (H) gates here.
	\tikzstyle{operator} = [draw,fill=white,minimum size=1.5em] 
	%
	\matrix[row sep=0.4cm, column sep=1cm] (circuit) {
		% First row
		\node (q1) {$\ket{00}$}; &
		\node[operator] (H11) {$H^{\otimes 2}$}; &
		\node[operator] (P13) {}; &
		\node[operator] (H11) {$H^{\otimes 2}$}; &
		\node[operator] (M11) {$M$};&
		\coordinate (end1); \\
		% Second row.
		\node (q2) {$\ket{00}$}; & & \node[operator] (P23) {}; & & & \coordinate 
(end2);\\
		% Third row
		\node (q31) {}; & \node (q32) {}; & \node (q33) {}; &
		\node (q34) {}; & \node (q35) {}; & \node (q36) {}; \\
		\node (q41) {}; & \node (q42) {}; & \node (q43) {}; &
		\node (q44) {}; & \node (q45) {}; & \node (q46) {}; \\
	};
	\node[operator] (Uf) [fit = (P13) (P23), minimum width=1cm] {$U_f$};

	\node (arr0) [fit = (q31) (q32)] {$\uparrow$};
	\node (arr1) [fit = (q32) (q33)] {$\uparrow$};
	\node (arr2) [fit = (q33) (q34)] {$\uparrow$};
	\node (arr3) [fit = (q34) (q35)] {$\uparrow$};

	\node (psi0) [fit = (q41) (q42)] {$\ket{\psi_0}$};
	\node (psi1) [fit = (q42) (q43)] {$\ket{\psi_1}$};
	\node (psi2) [fit = (q43) (q44)] {$\ket{\psi_2}$};
	\node (psi3) [fit = (q44) (q45)] {$\ket{\psi_3}$};

	\node[fill=white, fit=(end1) (end2)] (cover) {};

	\begin{pgfonlayer}{background}
		% Draw lines.
		\draw[thick] (q1) -- (M11)  (q2) -- (end2);
		\draw[thick, double, double distance=2pt] (M11) -- (end1);
	\end{pgfonlayer}
	%
\end{tikzpicture}
\end{center}

Finalmente el operador de Hadamard se aplica de nuevo, produciendo el efecto de 
interferencia.
\begin{equation*}
\begin{split}
\ket{\psi_3} = 1/4 \big( &
		+ \ket{00,00} + \cred{\ket{01,00}}+ \ket{10,00} + \cred{\ket{11,00}}\\
	& + \ket{00,00} - \cred{\ket{01,00}}+ \ket{10,00} - \cred{\ket{11,00}}\\
	& + \ket{00,01} + \cred{\ket{01,01}}- \ket{10,01} - \cred{\ket{11,01}}\\
	& + \ket{00,01} - \cred{\ket{01,01}}- \ket{10,01} + \cred{\ket{11,01}}
	\big)
\end{split}
\end{equation*}
\end{frame}
%%%%%%%%%%%%%%%%%%%%%%%%%%%%%%%%%%%%%%%%%%%%%%%%%%%%%%%%%%%%%%%%%%%%%%%%%%%%%%%
\begin{frame}
\frametitle{Funcionamiento}
\begin{center}
	\begin{tikzpicture}%[thick]
	% `operator' will only be used by Hadamard (H) gates here.
	\tikzstyle{operator} = [draw,fill=white,minimum size=1.5em] 
	%
	\matrix[row sep=0.4cm, column sep=1cm] (circuit) {
		% First row
		\node (q1) {$\ket{00}$}; &
		\node[operator] (H11) {$H^{\otimes 2}$}; &
		\node[operator] (P13) {}; &
		\node[operator] (H11) {$H^{\otimes 2}$}; &
		\node[operator] (M11) {$M$};&
		\coordinate (end1); \\
		% Second row.
		\node (q2) {$\ket{00}$}; & & \node[operator] (P23) {}; & & & \coordinate 
(end2);\\
		% Third row
		\node (q31) {}; & \node (q32) {}; & \node (q33) {}; &
		\node (q34) {}; & \node (q35) {}; & \node (q36) {}; \\
		\node (q41) {}; & \node (q42) {}; & \node (q43) {}; &
		\node (q44) {}; & \node (q45) {}; & \node (q46) {}; \\
	};
	\node[operator] (Uf) [fit = (P13) (P23), minimum width=1cm] {$U_f$};

	\node (arr0) [fit = (q31) (q32)] {$\uparrow$};
	\node (arr1) [fit = (q32) (q33)] {$\uparrow$};
	\node (arr2) [fit = (q33) (q34)] {$\uparrow$};
	\node (arr3) [fit = (q34) (q35)] {$\uparrow$};

	\node (psi0) [fit = (q41) (q42)] {$\ket{\psi_0}$};
	\node (psi1) [fit = (q42) (q43)] {$\ket{\psi_1}$};
	\node (psi2) [fit = (q43) (q44)] {$\ket{\psi_2}$};
	\node (psi3) [fit = (q44) (q45)] {$\ket{\psi_3}$};

	\node[fill=white, fit=(end1) (end2)] (cover) {};

	\begin{pgfonlayer}{background}
		% Draw lines.
		\draw[thick] (q1) -- (M11)  (q2) -- (end2);
		\draw[thick, double, double distance=2pt] (M11) -- (end1);
	\end{pgfonlayer}
	%
\end{tikzpicture}
\end{center}

Obteniéndose
\begin{equation*}
	\ket{\psi_3} = 1/2 \left( \ket{\mathbf{00},00} + \ket{\mathbf{00},01} + 
\ket{\mathbf{10},00} - \ket{\mathbf{10},01} \right)
\end{equation*}
Al medir la línea superior se obtienen vectores $\V x$ tal que $\V x \cdot \V s 
= 0$, con igual probabilidad.
$$ \V x \in \{00, 10\} $$
\end{frame}
%%%%%%%%%%%%%%%%%%%%%%%%%%%%%%%%%%%%%%%%%%%%%%%%%%%%%%%%%%%%%%%%%%%%%%%%%%%%%%%
\begin{frame}
\frametitle{Complejidad teórica}
Una vez obtenidos $n-1$ vectores $\V x$ linealmente independientes, se puede 
resolver el sistema de ecuaciones y calcular $\V s$.
\begin{equation*}
\begin{cases}
	& \V x^{(1)} \cdot \V s = 0 \\
	& \V x^{(2)} \cdot \V s = 0 \\
	& \vdots \\
	& \V x^{(n-1)} \cdot \V s = 0 \\
\end{cases}
\end{equation*}
El número de ejecuciones promedio $E[R]$ será:
$$ E[R] = \sum^{\infty}_{j=1} j \cdot p(R=j) $$
Siendo $p(R=j)$ la probabilidad de terminar en $j$ iteraciones.

\end{frame}
%%%%%%%%%%%%%%%%%%%%%%%%%%%%%%%%%%%%%%%%%%%%%%%%%%%%%%%%%%%%%%%%%%%%%%%%%%%%%%%
\begin{frame}
\frametitle{Complejidad teórica}

Construyendo una recurrencia, se calcula $E[R]$
$$ E[R] = \prod^{n-2}_{j=0} (1 - 2^{j-n+1}) \cdot \sum_{p=1}^\infty p \cdot 
2^{(p-n+1)(-n+1)} \cdot {p-1 \choose n-2}_{q=2} $$
Obteniendose computacionalmente los valores:
\begin{center}
	\small
	\begin{tabular}{c*{9}{c}}
		\toprule
		$n$      & 2 & 3 & 4 & 5 & 6 & 7 & 8 & 9 & 10\\
		\midrule
		$E[R]$   &
2.00& 3.33& 4.47& 5.54& 6.57& 7.59& 8.59& 9.60& 10.60
\\
		$\frac{E[R]}{n}$ &
1.00& 1.11& 1.12& 1.11& 1.10& 1.08& 1.08& 1.07& 1.06\\
		\bottomrule
	\end{tabular}
\end{center}

Se observa una complejidad lineal: $O(n)$

\end{frame}
%%%%%%%%%%%%%%%%%%%%%%%%%%%%%%%%%%%%%%%%%%%%%%%%%%%%%%%%%%%%%%%%%%%%%%%%%%%%%%%
\begin{frame}
\frametitle{Simulador cuántico}

Realiza la simulación del circuito, la medición y el procesado clásico final, 
imitando el comportamiento de un ordenador cuántico.

\begin{center}
\resizebox{\linewidth}{!}{
% Define block styles
\tikzstyle{decision} = [diamond, draw, text width=4.5em, text badly centered, 
inner sep=0pt]
\tikzstyle{block} = [rectangle, draw, text width=5em, text centered]
\tikzstyle{line} = [draw, decoration={markings,mark=at position 
1 with {\arrow[scale=1.5]{latex'}}}, postaction={decorate}]
%
\begin{tikzpicture}[node distance = 3cm, auto]%, transform canvas={scale=0.7}]
	% Place nodes
	\node[block] (quantum) {Simulación cuántica};
	\node[block, right of=quantum] (measure) {Medición};
	\node[block, right of=measure] (classic) {Procesado clásico};
	\node[decision, right of=classic] (end) {Fin?};
	% Draw edges
	\draw [line] (quantum.west)+(-1cm,0) -- (quantum.west);
	\draw [line] (quantum) -> (measure);
	\draw [line] (measure) -- (classic);
	\draw [line] (classic) -- (end);
	\draw [line] (end.south) |-+(0,-1em)-| node [near start, above] {No} 
(quantum.south);
	\draw [line] (end.east) -- node [near start] {Sí} +(1cm, 0);

\end{tikzpicture}
}
\end{center}
%

\textbf{Problema:} La simulación cuántica es muy costosa.

\end{frame}
%%%%%%%%%%%%%%%%%%%%%%%%%%%%%%%%%%%%%%%%%%%%%%%%%%%%%%%%%%%%%%%%%%%%%%%%%%%%%%%
\begin{frame}
\frametitle{Diseño del simulador}

Permite reutilizar el estado final $\ketp3$ tras la simulación cuántica para 
realizar las mediciones.

\begin{center}
\resizebox{\linewidth}{!}{

% Define block styles
\tikzstyle{decision} = [diamond, draw, text width=4.5em, text badly centered, 
inner sep=0pt]
\tikzstyle{block} = [rectangle, draw, text width=5em, text centered]
\tikzstyle{line} = [draw, decoration={markings,mark=at position 
1 with {\arrow[scale=1.5]{latex'}}}, postaction={decorate}]
%
\begin{tikzpicture}[node distance = 3cm, auto]
	% Place nodes
	\node[block] (quantum) {Simulación cuántica};
	\node[block, right of=quantum] (measure) {Medición};
	\node[block, right of=measure] (classic) {Procesado clásico};
	\node[decision, right of=classic] (end) {Fin?};
	% Draw edges
	\draw [line] (quantum.west)+(-1cm,0) -- (quantum.west);
	\draw [line] (quantum) -> (measure);
	\draw [line] (measure) -- (classic);
	\draw [line] (classic) -- (end);
	\draw [line] (end.south) |-+(0,-1em)-| node [near start, above] {No} 
(measure.south);
	\draw [line] (end.east) -- node [near start] {Sí} +(1cm, 0);

\end{tikzpicture}

}
\end{center}
%
\end{frame}
%%%%%%%%%%%%%%%%%%%%%%%%%%%%%%%%%%%%%%%%%%%%%%%%%%%%%%%%%%%%%%%%%%%%%%%%%%%%%%%
\begin{frame}
\frametitle{Estructuras de datos}
\begin{itemize}
\item Para almacenar los estados y operadores se emplean \textbf{matrices 
huecas}.
\item Permiten ahorrar espacio y reducir el número de operaciones.
\item Implementados en paquetes de cálculo como \texttt{scipy}
\item Varios formatos (COO y CSR).
\end{itemize}

\vspace{1cm}

El simulador está implementado en \texttt{python} empleando los paquetes 
\texttt{qutip}, \texttt{scipy} y \texttt{numpy}.


\end{frame}
%%%%%%%%%%%%%%%%%%%%%%%%%%%%%%%%%%%%%%%%%%%%%%%%%%%%%%%%%%%%%%%%%%%%%%%%%%%%%%%
\begin{frame}
\frametitle{Complejidad de la simulación}

Toda la simulación es analizada para determinar su rendimiento, midiendo el 
tiempo y el espacio empleados.

\end{frame}
%%%%%%%%%%%%%%%%%%%%%%%%%%%%%%%%%%%%%%%%%%%%%%%%%%%%%%%%%%%%%%%%%%%%%%%%%%%%%%%
\begin{frame}
\frametitle{Tiempo de la simulación}

%
\begin{figure}[!htb]
\centering
\footnotesize
\begin{tikzpicture}
\begin{semilogyaxis}[
	name=cpu,
	width=0.9\linewidth,
	height=7cm,
%	no marks,
	mark options={mark size=1},
	thick,
	grid=both,
	xtick={4,...,20},
	xmin=3.5,xmax=20.5,
	legend style={at={(0.0,1.0)},anchor=north west},
	xlabel={Número de qubits $N$},
	ylabel={Tiempo empleado (segundos)},
]
\addplot table [x=N, y=qc0-mean, col sep=comma] {csv/table_cpu.csv};
\addplot table [x=N, y=qcf-mean, col sep=comma] {csv/table_cpu.csv};
\addplot table [x=N, y=m-mean, col sep=comma] {csv/table_cpu.csv};
\addplot table [x=N, y=cc-mean, col sep=comma] {csv/table_cpu.csv};
\addplot [dashed] table [x=N, y=all-mean, col sep=comma] {csv/table_cpu.csv};
\legend{
	$T_\mu(QC_0)$,
	$T_\mu(QC_f)$,
	$T_\mu(M)$,
	$T_\mu(CC)$,
	$T_\mu(SIM)$
};
\end{semilogyaxis}
\end{tikzpicture}
\caption{Tiempo de simulación en escala logarítmica.}
\label{fig:tiempo-qc}
\end{figure}
%

\end{frame}
%%%%%%%%%%%%%%%%%%%%%%%%%%%%%%%%%%%%%%%%%%%%%%%%%%%%%%%%%%%%%%%%%%%%%%%%%%%%%%%
\begin{frame}
\frametitle{Espacio de la simulación}
%
\begin{figure}[!htb]
\footnotesize
\centering
\begin{tikzpicture}
\begin{axis}[
	name=qc0,
	width=0.9\linewidth,
	height=7cm,
%	no marks,
	mark options={mark size=1},
	thick,
	grid=both,
	xtick={4,...,20},
	xmin=3.5,xmax=20.5,
	legend style={at={(1.0,0.0)},anchor=south east},
	xlabel={Número de qubits $N$},
	ylabel={Espacio ocupado $\log_2 S$},
]
\addplot [red] coordinates {(0,29) (20.5,29)};
\addplot [blue,mark=*] table [x=N, y=log2_all_qcf, col sep=comma] 
{csv/table_qcf.csv};
\addplot [black,mark=*] table [x=N, y=log2_all_qc0, col sep=comma] 
{csv/table_qc0.csv};
\addplot [blue, dashed] table [x=N, y=log2_approx_qcf, col sep=comma] 
{csv/table_qcf.csv};
\addplot [black, dashed] table [x=N, y=log2_approx_qc0, col sep=comma] 
{csv/table_qc0.csv};
\addplot [purple, dashed, domain=4:6.25]{(4*x+4)};
\legend{
	$S_{MAX}$,
	$S_T (QC_f)$,
	$S_T (QC_0)$,
	$S_T' (QC_f)$,
	$S_T' (QC_0)$,
	$S_D$};
\end{axis}
\end{tikzpicture}
\caption{Espacio necesario para la simulación en bytes (escala logarítmica). El 
espacio requerido sin emplear matrices huecas, usando matrices densas se muestra 
como $S_D$. La memoria disponible para la simulación es $S_{MAX} = 2^{29}$.}
\label{fig:espacio-qc}
\end{figure}
%
\end{frame}
%%%%%%%%%%%%%%%%%%%%%%%%%%%%%%%%%%%%%%%%%%%%%%%%%%%%%%%%%%%%%%%%%%%%%%%%%%%%%%%
\begin{frame}
\frametitle{Complejidad experimental del algoritmo de Simon}
%
\begin{figure}[!htb]
\footnotesize
\centering
\begin{tikzpicture}
\begin{axis}[
	name=qc0,
	width=0.9\linewidth,
	height=7cm,
%	no marks,
	mark options={mark size=1},
	thick,
	grid=both,
	xtick={4,...,20},
	ytick={2,3,...,20},
	xmin=3.5,xmax=20.5,
	legend style={at={(1.0,0.0)},anchor=south east},
	xlabel={Número de qubits $N$},
	ylabel={Número esperado de ejecuciones R},
]
\addplot [red,mark=*,thick] table [x=N, y=R-mean, col sep=comma] 
{csv/com_qc.csv};
\addplot [black, dashed, mark=none, very thick] table [x=N, y=RT-mean, col 
sep=comma] {csv/com_qc.csv};
\legend{
	$R_\mu$,
	$R_\mu'$};
\end{axis}
\end{tikzpicture}
\caption{Número esperado de iteraciones a medida que aumentan los qubits. Se 
observa el valor experimental $R_\mu$ comparado con el teórico $R_\mu'$.}
\label{fig:comp-qc}
\end{figure}
%
\end{frame}
%%%%%%%%%%%%%%%%%%%%%%%%%%%%%%%%%%%%%%%%%%%%%%%%%%%%%%%%%%%%%%%%%%%%%%%%%%%%%%%
\begin{frame}
\frametitle{Conclusiones}

\begin{itemize}
\item La computación cuántica puede resolver problemas \textbf{mucho} más rápido 
que la computación clásica.
\item Realizar la simulación de un circuito cuántico es muy costoso.
\item La simulación de circuitos permite comprobar los resultados teóricos.
\end{itemize}

\end{frame}
%%%%%%%%%%%%%%%%%%%%%%%%%%%%%%%%%%%%%%%%%%%%%%%%%%%%%%%%%%%%%%%%%%%%%%%%%%%%%%%
\begin{frame}
\frametitle{Trabajo futuro}
\begin{itemize}
\item Probar otros algoritmos cuánticos.
\item Realizar el análisis de un algoritmo cuántico de forma automática, para 
determinar su complejidad.
\end{itemize}

\end{frame}
%%%%%%%%%%%%%%%%%%%%%%%%%%%%%%%%%%%%%%%%%%%%%%%%%%%%%%%%%%%%%%%%%%%%%%%%%%%%%%%
\begin{frame}
\frametitle{Preguntas}
¿Preguntas o comentarios?

\end{frame}
%%%%%%%%%%%%%%%%%%%%%%%%%%%%%%%%%%%%%%%%%%%%%%%%%%%%%%%%%%%%%%%%%%%%%%%%%%%%%%%

\end{document}
