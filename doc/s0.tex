\begin{dedication}
A los rebeldes.
\end{dedication}

\thispagestyle{empty}
\cleardoublepage
\chapter*{Agradecimientos}

Este proyecto ha sido posible debido a una serie de seminarios de introducción a 
la computación cuántica, impartidos por Vicente Moret, que a su vez, ha dirigido 
este trabajo, y a quien le estoy muy agradecido.

Gracias a todos los que han aportado ideas y discutido los ejemplos 
seleccionados para ser fáciles de comprender.

Pero sobre todo, muchas gracias a la biblioteca de la FIC, que debido a su 
programa de libros expurgados, me ha causado un interés sin precedentes por la 
lectura de temas desconocidos, y que ahora, forman parte de este proyecto.

\clearpage
\thispagestyle{empty}
\cleardoublepage

\chapter*{Resumen}
\noindent
La computación cuántica estudia como aprovechar las características de la 
materia a escala subatómica para realizar cómputos. Surgen nuevas posibilidades 
para la resolución de problemas, que son imposibles en los ordenadores actuales.  
Conceptos como superposición o entrelazamiento suponen la mayor diferencia a la 
hora de comprender como funcionan las nuevas soluciones.

Un simulador cuántico permite realizar pruebas en un ordenador convencional, y 
observar los resultados para contrastarlos con un modelo teórico. Además, sirve 
como herramienta didáctica, debido a que permite la observación minuciosa del 
proceso. Pese a la complejidad exponencial requerida para simular el 
comportamiento cuántico, la capacidad actual es suficiente para realizar pruebas 
a pequeña escala.

En este caso, se propone el análisis del algoritmo de Simon, que soluciona un 
problema cuya solución clásica tiene una complejidad de $O(2^n)$, y se puede 
resolver con un algoritmo cuántico en $O(n)$.

Adicionalmente, se miden y analizan los resultados obtenidos tanto de forma 
experimental con la simulación, así como los teóricos, que son contrastados. El 
tiempo de cómputo y los recursos necesarios son minuciosamente examinados 
durante la ejecución.

\clearpage
\thispagestyle{empty}
\cleardoublepage

\chapter*{Palabras clave}
\noindent
Computación cuántica, circuito cuántico, algoritmo de Simon, qubit,
superposición, simulación.
