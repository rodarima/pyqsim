\documentclass{beamer}

% Fórmulas matemáticas con fuente adecuada
\usefonttheme[onlymath]{serif}

%\documentclass[12pt,a4paper]{report}
\usepackage{amsfonts}
\usepackage{amsmath}
\usepackage{amsthm}
\usepackage[utf8]{inputenc}
\usepackage{braket}

% Código fuente
%\usepackage{fontspec}
\usepackage{minted}
\newminted{py}{%
		linenos,
		fontsize=\small,
		tabsize=4,
		mathescape,
}

% Tamaño de los números de la las líneas en el código fuente
\renewcommand\theFancyVerbLine{\footnotesize\arabic{FancyVerbLine}}

% Emplear "tabla" y no "cuadro" para las tablas.
% Romper líneas por sílabas en español
\usepackage[spanish,es-nosectiondot,es-tabla]{babel}

% Usar estilo de libro para las tablas
\usepackage{booktabs}

% Gráficas
\usepackage{pgfplots}

% Emplear un tamaño más pequeño para las etiquetas
\usepackage[font=footnotesize]{caption}

% Cargar tablas de csv
\usepackage{pgfplotstable}

% Urls
\usepackage{url}

% Algoritmos
\usepackage[spanish, ruled]{algorithm2e}

% Imágenes vectoriales en SVG
\usepackage{svg}

% Dedicatoria
\newenvironment{dedication}{
\thispagestyle{empty}% no header and footer
\cleardoublepage           % we want a new page
\thispagestyle{empty}% no header and footer
\vspace*{\stretch{1}}% some space at the top 
\itshape             % the text is in italics
\raggedleft          % flush to the right margin
}
{\par % end the paragraph
\vspace{\stretch{3}} % space at bottom is three times that at the top
\clearpage           % finish off the page
%\cleardoublepage
}

% Dimensiones de los márgenes.
%\usepackage[margin=3cm]{geometry}
%\usepackage[top=3cm, bottom=3cm, left=3.5cm, right=2.5cm]{geometry}

% Macros para ayudar a la redacción
% Vector
\newcommand*\mat[1]{ \begin{pmatrix} #1 \end{pmatrix}}
\newcommand*\arr[1]{ \begin{bmatrix} #1 \end{bmatrix}}
\newcommand*\V[1]{ \boldsymbol{#1}}
% Ket psi
\newcommand*\ketp[1]{\ket{\psi_{#1}}}
% Floor
\newcommand{\floor}[1]{\lfloor #1 \rfloor}
% Absolute value
\newcommand\abs[1]{\left|#1\right|}

% Probabilidad I y ¬I
\newcommand*\NO[1]{ \overline{#1}}
\newcommand*\NI[1]{ \overline{I^{#1}}\,}

% Operador identidad I^{\otimes n}
\newcommand*\opn[2]{{#1}^{\otimes #2}}

% Producto pensorial
\newcommand\ox[2]{\ket{#1} \otimes \ket{#2}}

% Colores
\newcommand*\cred[1]{{\color{red} {#1}}}
\newcommand*\cgre[1]{{\color{green} {#1}}}

% Apartado de un ejemplo
\theoremstyle{definition}
\newtheorem{ejemplo}{Ejemplo}[section]

% Gráficos con tikz
\usepackage{tikz}

\usetikzlibrary{calc,fadings,decorations.pathreplacing}
\usetikzlibrary{backgrounds,fit}

\usetikzlibrary{shapes,arrows,chains}
\usetikzlibrary{decorations.markings}

\newcommand\pgfmathsinandcos[3]{%
  \pgfmathsetmacro#1{sin(#3)}%
  \pgfmathsetmacro#2{cos(#3)}%
}
\newcommand\LongitudePlane[3][current plane]{%
  \pgfmathsinandcos\sinEl\cosEl{#2} % elevation
  \pgfmathsinandcos\sint\cost{#3} % azimuth
  \tikzset{#1/.estyle={cm={\cost,\sint*\sinEl,0,\cosEl,(0,0)}}}
}
\newcommand\LatitudePlane[3][current plane]{%
  \pgfmathsinandcos\sinEl\cosEl{#2} % elevation
  \pgfmathsinandcos\sint\cost{#3} % latitude
  \pgfmathsetmacro\yshift{\cosEl*\sint}
  \tikzset{#1/.estyle={cm={\cost,0,0,\cost*\sinEl,(0,\yshift)}}} %
}
\newcommand\DrawLongitudeCircle[2][1]{
  \LongitudePlane{\angEl}{#2}
%  \tikzset{current plane/.prefix style={scale=#1}}
   % angle of "visibility"
  \pgfmathsetmacro\angVis{atan(sin(#2)*cos(\angEl)/sin(\angEl))} %
  \draw[current plane] (\angVis:1) arc (\angVis:\angVis+180:1);
  \draw[current plane,dashed] (\angVis-180:1) arc (\angVis-180:\angVis:1);
}
\newcommand\DrawLatitudeCircle[2][1]{
  \LatitudePlane{\angEl}{#2}
%	\tikzset{current plane/.prefix style={scale=#1}}
  \pgfmathsetmacro\sinVis{sin(#2)/cos(#2)*sin(\angEl)/cos(\angEl)}
  % angle of "visibility"
  \pgfmathsetmacro\angVis{asin(min(1,max(\sinVis,-1)))}
  \draw[current plane] (\angVis:1) arc (\angVis:-\angVis-180:1);
  \draw[current plane,dashed] (180-\angVis:1) arc (180-\angVis:\angVis:1);
}

% Numerar las diapositivas
\beamertemplatenavigationsymbolsempty

\setbeamerfont{page number in head/foot}{size=\large}
\setbeamertemplate{footline}[frame number]


% Propiedades
\title{Diseño, desarrollo, implementación y prueba de un simulador cuántico para 
el algoritmo de Simon}

\author{Rodrigo Arias Mallo (rodrigo.arias@udc.es)}

\begin{document}

%\maketitle

%%%%%%%%%%%%%%%%%%%%%%%%%%%%%%%%%%%%%%%%%%%%%%%%%%%%%%%%%%%%%%%%%%%%%%%%%%%%%%%
%\clearpage 

%\tableofcontents

%\clearpage 

% Diapositiva inicial
\frame{\titlepage}

%%%%%%%%%%%%%%%%%%%%%%%%%%%%%%%%%%%%%%%%%%%%%%%%%%%%%%%%%%%%%%%%%%%%%%%%%%%%%%%
\begin{frame}
\frametitle{Introducción}

\begin{itemize}
\item Que es esto?
\item Por que?
\end{itemize}

\end{frame}
%%%%%%%%%%%%%%%%%%%%%%%%%%%%%%%%%%%%%%%%%%%%%%%%%%%%%%%%%%%%%%%%%%%%%%%%%%%%%%%
\begin{frame}
\frametitle{Problema de Simon}
Sea $f$ una función binaria
$$f:\{0,1\}^n \rightarrow \{0,1\}^n$$
que cumple la propiedad
$$ f(\V x) = f(\V y) \iff \V x = \V y \oplus \V s$$
con $\V x, \V y, \V s \in \{0,1\}^n$ y el período $\V s \neq \V 0$.

\vspace{1cm}

\textbf{Objetivo: }Encontrar $\V s$ partiendo de la función $f$.


\end{frame}
%%%%%%%%%%%%%%%%%%%%%%%%%%%%%%%%%%%%%%%%%%%%%%%%%%%%%%%%%%%%%%%%%%%%%%%%%%%%%%%
\begin{frame}
\frametitle{Función $f$ de ejemplo}

Ejemplo de 2 bits, y un período $\V s = 01$

\begin{center}
\begin{tabular}{cccc}
	\toprule
	$\V x$ & $f(\V x)$ & $\V x \oplus \V s$ & $f(\V x \oplus \V s)$\\
	\midrule
	00 & 00 & 01 & 00\\
	01 & 00 & 00 & 00\\
	10 & 01 & 11 & 01\\
	11 & 01 & 10 & 01\\
	\bottomrule
\end{tabular}
\end{center}

$$ f(\V x) = f(\V y) \iff \V x = \V y \oplus \V s$$

\end{frame}
%%%%%%%%%%%%%%%%%%%%%%%%%%%%%%%%%%%%%%%%%%%%%%%%%%%%%%%%%%%%%%%%%%%%%%%%%%%%%%%
\begin{frame}
\frametitle{Solución clásica}
\begin{itemize}
\item Una solución sencilla consiste en probar entradas hasta obtener una salida 
repetida.

\item Hay $2^n/2$ salidas diferentes.

\item Es necesario probar al menos $2^n/2 + 1$ entradas: $O(2^n)$.
\end{itemize}
\end{frame}
%%%%%%%%%%%%%%%%%%%%%%%%%%%%%%%%%%%%%%%%%%%%%%%%%%%%%%%%%%%%%%%%%%%%%%%%%%%%%%%
\begin{frame}
\frametitle{Solución cuántica: $O(2^n) \rightarrow O(n)$}
\begin{itemize}
\item Para solucionar este problema, Daniel R. Simon propuso una solución 
empleando la \textbf{computación cuántica}.
\item Soluciona el problema en $O(n)$.
\end{itemize}

\end{frame}
%%%%%%%%%%%%%%%%%%%%%%%%%%%%%%%%%%%%%%%%%%%%%%%%%%%%%%%%%%%%%%%%%%%%%%%%%%%%%%%
\begin{frame}
\frametitle{Introducción a la computación cuántica}
\begin{itemize}
\item La computación cuántica emplea las propiedades de la materia a escala 
atómica.
\item El comportamiento de la materia se describe mediante los postulados de la 
mecánica cuántica
\end{itemize}
\end{frame}
%%%%%%%%%%%%%%%%%%%%%%%%%%%%%%%%%%%%%%%%%%%%%%%%%%%%%%%%%%%%%%%%%%%%%%%%%%%%%%%
\begin{frame}
\frametitle{El bit}
Para almacenar la información clásica se emplea el bit, que puede representarse 
mediante un vector $\V x$.
$$ \V x = \mat{p \\ q} $$
Con  $p, q \in \{0,1\}$ y cumpliendo la restricción $p \oplus q = 1$.

Se encuentra en uno de los dos estados posibles
$$ \text{bit 0} = \mat{1 \\ 0},\quad \text{bit 1} = \mat{0 \\ 1} $$



\end{frame}
%%%%%%%%%%%%%%%%%%%%%%%%%%%%%%%%%%%%%%%%%%%%%%%%%%%%%%%%%%%%%%%%%%%%%%%%%%%%%%%
\begin{frame}
\frametitle{El bit probabilístico}
%
El estado de moneda perfecta al ser lanzada al aire, se puede describir mediante 
un vector $\V y$.
$$ \V y = \mat{1/2 \\ 1/2} = \mat{a \\ b} $$
Con $a,b \in \mathbb R$ y cumpliendo la restricción $a + b = 1$.
Siendo $a$ la probabilidad de que salga cara, y $b$ la de que salga cruz.  Al 
caer:
$$ \V y_{\text{cara}} = \mat{1 \\ 0},\quad
\V y_{\text{cruz}} = \mat{0 \\ 1}$$
\end{frame}
%%%%%%%%%%%%%%%%%%%%%%%%%%%%%%%%%%%%%%%%%%%%%%%%%%%%%%%%%%%%%%%%%%%%%%%%%%%%%%%
\begin{frame}
\frametitle{El bit cuántico o qubit}
El estado de un sistema cuántico también se puede representar mediante un vector 
$\V z$.
$$ \V z = \mat{\alpha \\ \beta} $$
Con $\alpha, \beta \in \mathbb C$ y cumpliendo la restricción $|\alpha|^2 + 
|\beta|^2 = 1$. La probabilidad de que salga 0 es $|\alpha|^2$ y de que salga 1 
es $|\beta|^2$. Al medir:
$$ \V z_{\text{cero}} = \mat{1 \\ 0},\quad
\V z_{\text{uno}} = \mat{0 \\ 1}$$


\end{frame}
%%%%%%%%%%%%%%%%%%%%%%%%%%%%%%%%%%%%%%%%%%%%%%%%%%%%%%%%%%%%%%%%%%%%%%%%%%%%%%%
\begin{frame}
\frametitle{Notación de Dirac}

\end{frame}
%%%%%%%%%%%%%%%%%%%%%%%%%%%%%%%%%%%%%%%%%%%%%%%%%%%%%%%%%%%%%%%%%%%%%%%%%%%%%%%

\end{document}
