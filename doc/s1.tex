\documentclass[12pt,a4paper]{article}
\usepackage{amsfonts}
\usepackage{amsmath}
\usepackage{amsthm}
\usepackage[utf8]{inputenc}
\usepackage{braket}

% Romper líneas por sílabas en español
\usepackage[spanish,es-nosectiondot]{babel}

% Dimensiones de los márgenes.
%\usepackage[margin=2.5cm]{geometry}

% Vector
\newcommand*\mat[1]{ \begin{pmatrix} #1 \end{pmatrix}}
\newcommand*\arr[1]{ \begin{bmatrix} #1 \end{bmatrix}}

\newcommand*\V[1]{ \boldsymbol{#1}}

% Apartado de un ejemplo
\theoremstyle{definition}
\newtheorem{ejemplo}{Ejemplo}[section]

% Gráficos con tikz
\usepackage{tikz}

\usetikzlibrary{calc,fadings,decorations.pathreplacing}
\usetikzlibrary{backgrounds,fit}
\newcommand\pgfmathsinandcos[3]{%
  \pgfmathsetmacro#1{sin(#3)}%
  \pgfmathsetmacro#2{cos(#3)}%
}
\newcommand\LongitudePlane[3][current plane]{%
  \pgfmathsinandcos\sinEl\cosEl{#2} % elevation
  \pgfmathsinandcos\sint\cost{#3} % azimuth
  \tikzset{#1/.estyle={cm={\cost,\sint*\sinEl,0,\cosEl,(0,0)}}}
}
\newcommand\LatitudePlane[3][current plane]{%
  \pgfmathsinandcos\sinEl\cosEl{#2} % elevation
  \pgfmathsinandcos\sint\cost{#3} % latitude
  \pgfmathsetmacro\yshift{\cosEl*\sint}
  \tikzset{#1/.estyle={cm={\cost,0,0,\cost*\sinEl,(0,\yshift)}}} %
}
\newcommand\DrawLongitudeCircle[2][1]{
  \LongitudePlane{\angEl}{#2}
%  \tikzset{current plane/.prefix style={scale=#1}}
   % angle of "visibility"
  \pgfmathsetmacro\angVis{atan(sin(#2)*cos(\angEl)/sin(\angEl))} %
  \draw[current plane] (\angVis:1) arc (\angVis:\angVis+180:1);
  \draw[current plane,dashed] (\angVis-180:1) arc (\angVis-180:\angVis:1);
}
\newcommand\DrawLatitudeCircle[2][1]{
  \LatitudePlane{\angEl}{#2}
%	\tikzset{current plane/.prefix style={scale=#1}}
  \pgfmathsetmacro\sinVis{sin(#2)/cos(#2)*sin(\angEl)/cos(\angEl)}
  % angle of "visibility"
  \pgfmathsetmacro\angVis{asin(min(1,max(\sinVis,-1)))}
  \draw[current plane] (\angVis:1) arc (\angVis:-\angVis-180:1);
  \draw[current plane,dashed] (180-\angVis:1) arc (180-\angVis:\angVis:1);
}

\begin{document}

\section{Introducción a la computación cuántica}
\subsection{Preliminares}
La computación cuántica es un nuevo paradigma de computación, que busca la
resolución de problemas empleando las propiedades de la mecánica cuántica.

Fundamenta su utilidad en el hecho de que algunos problemas que tienen una alta
complejidad en los ordenadores clásicos, y son intratables, pasan a ser
tratables en un ordenador cuántico.

\subsection{El bit y el qubit}

\subsubsection{Bit clásico}
%Explicar lo que es un bit, y diferenciar entre el estado interno y lo que 
%representa
En los ordenadores actuales, la unidad básica de representación de información 
es el bit. La palabra bit significa dígito binario, que puede ser 0 o 1. Es el 
nexo de unión entre los sistemas físicos que implementan un bit, y una lógica 
binaria, que se abstrae de su implementación.

La forma en la que se implementa un bit depende de la arquitectura: la posición 
de una leva mecánica en la máquina, la existencia de una presilla en una cinta 
de vídeo (o también de cinta adhesiva), el estado de un interruptor, la 
presencia o ausencia de un agujero en una tarjeta perforada, anillos de ferrita 
que se magnetizan en un sentido o en otro, dos niveles de voltaje diferentes...

Todas estas representaciones, diferentes en su naturaleza física, comparten dos 
propiedades en común; es posible representar dos estados diferentes, y además es 
posible modificar y leer el estado en el que se encuentra el sistema.

De esta forma es importante distinguir entre la \textit{representación física} y 
el \textit{significado lógico}. En el caso de la tarjeta perforada, esta 
relación puede darse de la siguiente forma: Si hay un agujero, entonces 
representa un 0; en caso contrario, un 1. Para representar un bit, se empleará 
un estado $x \in \{0, 1\}$.


\subsubsection{Bit probabilístico}
%Introducir la analogía de una moneda para diferenciar entre el estado de la 
%moneda que es 0.5/0.5 mientras no se mide, y una vez "colapsa". Expresar la 
%probabilidad en forma de vector

Una moneda perfecta que se lanza al aire, y que cae sobre una superficie lisa, 
saca cara con igual probabilidad que cruz. Mientras se encuentra en el aire 
dando vueltas, su estado no es ni cara ni cruz. Es un estado diferente. Este 
estado puede representarse mediante un bit probabilístico, de forma que la cara 
es el 0, la cruz es el 1, y ambos tienen una probabilidad de 0.5 de ocurrir.  
Escribiendo las probabilidades en un vector, se define
%
$$ y = \mat{0.5 \\ 0.5} $$
%
Comenzando la numeración en 0, el primer elemento $y[0]$ representa la 
probabilidad de que el sistema saque un 0. Y $y[1]$ de que saque un 1. El hecho 
de que el índice coincida con el estado que representa, será útil en el futuro.

Además, debido a que se trata de un sistema probabilístico, ambas probabilidades 
han de sumar la unidad.
%
$$ y[0] + y[1] = 1 $$
%
Una vez que la moneda cae, su estado deja de ser un bit probabilístico $y$, y se 
convierte en un bit clásico $x$. Esta operación es la \textit{medición} de un 
sistema probabilístico.

\subsubsection{Bit cuántico o qubit}
%Extender el ejemplo de la moneda con la amplitud en vez de la probabilidad, y 
%explicar la notación en forma de vector

Hasta ahora, la clase de sistemas que se han descrito, son conocidos por 
experiencias en la vida cotidiana o profesional, y sirven de analogías. Sin 
embargo, el funcionamiento del sistema que se describirá a continuación no tiene 
un ejemplo conocido, tan sólo la imaginación será capaz de construir dicho 
sistema.

Un sistema cuántico puede encontrarse en un estado $z$, definido como
%
$$ z = \mat{\alpha \\ \beta}$$
%
Además, tanto $\alpha$ como $\beta$ son números complejos. Este estado se 
denomina bit cuántico o qubit.
%
Tras medir un qubit, el sistema se convertirá en un bit clásico $x$, con una 
probabilidad de que salga 0 de $|\alpha|^2$ y de que salga 1 de $|\beta|^2$.
Es decir, que se comportará como un bit probabilístico con las probabilidades
%
$$ y = \mat{|\alpha|^2 \\ |\beta|^2}$$
%
Y de igual modo, cumple la restricción de que ambas probabilidades suman la 
unidad:
%
$$ |\alpha|^2 + |\beta|^2 = 1 $$
%
Los números $\alpha$ y $\beta$ se denominan \textit{amplitudes}, y tienen 
asociada una probabilidad que es $|\alpha|^2$ y $|\beta|^2$ respectivamente.

\subsection{Múltiples qubits}
% Combinar dos bits para representar 4 estados, dos monedas para 4 estados
% probabilísticos, y 2 qubits para un 2-qubit.

\subsection{Operador}

Un sistema clásico puede contener varios bits, de forma que el número de 
posibles estados en los que se puede encontrar es $2^n$ para $n$ bits. Por 
ejemplo, en una tarjeta perforada del telar de Jacquard de $8$ filas por 26 
columnas, con un total de 208 celdas, existen $2^{208} \approx 
4.11 \cdot 10^{62}$ posibles combinaciones.

Si se emplean 108 monedas, el número de posibles combinaciones de cara y cruz 
sería el mismo. Sin embargo, justo antes de caer, el sistema se encuentra en un 
estado en el que cada posible combinación tiene la misma probabilidad de 
ocurrir. Del mismo modo que para un bit probabilístico, para $n$ se puede 
emplear un vector $y$ que describa la probabilidad para cada estado posible
$$
	y = \mat{y_0 & \cdots & y_{2^n-1}}^T
$$

\subsection{Registros cuánticos}
Para almacenar más de un qbit se emplean los registros cuánticos, que son 
análogos a los registros clásicos de $n$ bits.

Sin embargo difieren de los registros clásicos de bits. En uno clásico, el 
registro se encuentra siempre en un estado determinado. Para $n$ bits, existen 
$2^n$ estados posibles, y el registro se encuentra en uno de ellos. Sin embargo 
en uno cuántico, no sólo puede encontrarse en todos esos estados, si no también 
en superposición. Esto es, que se encuentra en una mezcla de varios estados. O 
incluso de todos ellos.

\subsubsection{Superposición}


\subsubsection{Producto tensorial}
\label{sss:producto-tensorial}

La operación $\otimes$ se define como el producto tensorial o producto de 
Kronecker. Si A y B son dos matrices de $n \times m$ y $k \times l$ 
respectivamente:
%
$$
\begin{array}{c c}
	A=\mat{
		a_{11} & \cdots & a_{1m} \\
		\vdots &        & \vdots \\
		a_{n1} & \cdots & a_{nm}
	}
	,\quad
	&
	B=\mat{
		b_{11} & \cdots & b_{1l} \\
		\vdots &        & \vdots \\
		b_{k1} & \cdots & b_{kl}
	}
\end{array}
$$
%
Entonces $C = A \otimes B$ es la matriz $C$ de $nk \times ml$, definida como
%
$$
C = A \otimes B=\mat{
	a_{11} B & \cdots & a_{1m} B \\
	\vdots   &        & \vdots   \\
	a_{n1} B & \cdots & a_{nm} B
}
$$
%
El producto tensorial cumple varias propiedades interesantes.
%
\begin{itemize}
\item Es asociativo $(A \otimes B) \otimes C = A \otimes (B \otimes C)$

\item Es distributivo respecto a la suma $A \otimes (B + C) = (A \otimes B) + (A
\otimes C) $ y $(A + B) \otimes C = (A \otimes C) + (B \otimes C)$

\item Para un escalar $k$, cumple $(kA) \otimes B = A \otimes (kB) = k (A 
\otimes B)$
\end{itemize}
%
Pero en general, no es conmutativo: $A \otimes B \neq B \otimes A$.





Para representar $N$ bits, es suficiente con un vector $\boldsymbol{b} = 
\mat{b_0 & b_1 & \cdots & b_{N-1} }^T$. De forma que $b_i \in \{0,1\}$. Por 
ejemplo un byte, que está formado de 8 bits, es:
$$\mat{b_0 & b_1 & b_2 & b_3 & b_4 & b_5 & b_6 & b_7}^T$$
Simplificando la notación, representando solo los bits:
$$\boldsymbol{b} = b_0 \, b_1 \, b_2 \, b_3 \, b_4 \, b_5 \, b_6 \, b_7$$
Por lo que $\boldsymbol{b} \in \{0,1\}^N$, y numerando cada vector con:
$$d = \sum_i 2^i b_i$$
Por ejemplo $d=14$ es $\boldsymbol{b} = 1110$ con $N=4$. $1110_b = 14_d$
Por ejemplo $11000001_b = 193_d$


\subsection{Operaciones}



Para realizar una operación sobre un estado cuántico, se emplea un operador 
expresado por una matriz.

Por ejemplo, la puerta NOT que invierte el estado de un bit, tiene una operación 
análoga en la computación cuántica.
%
$$ NOT = \mat{0 & 1 \\ 1 & 0} $$
%
De modo que al aplicarla sobre un estado $\ket{\psi} = \mat{\alpha & \beta}^T$, 
se obtiene:
%
$$ NOT \ket{\psi} = \mat{0 & 1 \\ 1 & 0} \mat{\alpha \\ \beta} = \mat{\beta \\ 
\alpha} $$
%
Las amplitudes se han invertido, de forma análoga a un estado clásico. Si 
$\ket{\psi} = \ket{0}$, entonces $NOT\ket{0} = NOT \mat{1 & 0}^T = \mat{0 & 
1}^T = \ket{1}$. Y de igual forma para $NOT \ket{1} = \ket{0}$.

\newpage

\end{document}
