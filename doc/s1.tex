\chapter{Introducción}

Los ordenadores son máquinas complejas que realizan cálculos. Se basan en la 
lógica binaria, para almacenar toda la información en forma de ceros y unos.

La computación cuántica abre las puertas a una nueva forma de construir 
ordenadores, y de operar con ellos. Aunque por ahora la construcción de tales 
máquinas es compleja, su simulación es posible en los ordenadores clásicos.

En este trabajo se propone un simulador cuántico, que permite ejecutar circuitos 
de la misma forma que realizaría un ordenador cuántico.

\section{Motivación y objetivos}
% En que consiste 

Entre los objetivos que se buscan con un simulador de circuitos, se encuentra la 
capacidad para ejecutar un circuito cuántico, con el objetivo de contrastar los 
resultados experimentales con los valores teóricos. Debe permitir conseguir los 
mismos resultados que el sistema real, y en un tiempo de simulación razonable.

%

\section{Organización}

En el capítulo 2 se realiza una introducción a la computación cuántica, para 
adquirir los conceptos previos en los que se basan los circuitos cuánticos, así 
como la notación empleada. En el capítulo 3, se examinan las técnicas 
fundamentales que usan los circuitos cuánticos para explotar las nuevas 
posibilidades de cómputo. El capítulo 4, introduce un ejemplo concreto, el 
algoritmo de Simon, que es analizado de forma teórica. El capítulo 5 presenta el 
diseño del simulador, así como la implementación con el circuito del algoritmo 
de Simon. Finalmente en el capítulo 6 se analizan los resultados de la 
simulación y se contrastan con los teóricos.

\section{Estado del arte}
% Comentar que otros simuladores hay, y que resultados obtienen
Actualmente existe una larga lista de simuladores y herramientas para la 
simulación, que son recopilados en la web \cite{quantiki}.





\chapter{Computación cuántica}
% Introducir suavemente el tema de la computación cuántica. Como surge, por que, 
% y para que sirve

%\section{Introducción}



%\subsection{Mecánica cuántica}
La naturaleza se comporta de una forma sorprendentemente inesperada a medida que 
la escala a la que se observan los detalles se hace cada vez más pequeña.  
Resulta muy extraño describir dicho comportamiento en comparación con las cosas 
que estamos acostumbrados a ver en el día a día. Sin embargo, existe una serie 
de reglas que se han ido descubriendo, y que siempre se cumplen (hasta ahora lo 
han hecho). Además estas reglas son excepcionalmente simples, y pueden ser 
descritas mediante las ecuaciones matemáticas que hasta ahora conocemos. Este 
conjunto de reglas se conocen como los postulados de la mecánica cuántica.

Una de las ventajas de poder describir el comportamiento de algún suceso, es la 
posibilidad de emplearlo de forma provechosa. Por ejemplo, saber que una 
substancia impide el crecimiento de una bacteria infecciosa permite erradicar 
una enfermedad---es el caso de la penicilina.

La mecánica cuántica puede emplearse para describir sistemas cuidadosamente 
diseñados para que se comporten como un mecanismo de cálculo. Actualmente los 
ordenadores emplean nuestro conocimiento de las leyes físicas como el 
electromagnetismo para realizar cómputos. Del mismo modo, se pueden emplear 
sistemas cuánticos dando lugar al término ordenador cuántico. La idea de 
construir tal dispositivo ha sido propuesta por primera vez por Richard Feynman 
en 1982 \cite{feynman-sim}, con el objetivo de simular sistemas cuánticos.

%\subsection{Computación cuántica}
La computación cuántica es un nuevo paradigma de computación, que busca la
resolución de problemas empleando las propiedades de la mecánica cuántica.
Fundamenta su utilidad en el hecho de que algunos problemas que tienen una alta
complejidad en los ordenadores clásicos, y son intratables, pasan a ser
tratables en un ordenador cuántico.

Peter Shor descubrió en 1994 \cite{shor97} como resolver el problema de 
factorización de un número compuesto en el producto de dos primos, como $15 = 
3\cdot5$. La complejidad de este cómputo es tan alta para números elevados, que 
se considera como base para algunos métodos de criptografía de como RSA. Se 
supone que nadie sería capaz de encontrar esos dos números primos en mucho 
tiempo. Pero empleando un ordenador cuántico, el tiempo se reduce lo suficiente 
como para encontrarlos. De modo que la criptografía tendría que buscar otros 
métodos.

La factorización de números mediante algoritmos cuánticos se ha llevado a cabo 
de forma experimental, siendo el número 56153 el más grande hasta la fecha 
\cite{factor}. 

Otro algoritmo cuántico fue descubierto en 1996 por Lov Grover \cite{grover96} 
para encontrar un elemento en una lista desordenada de tamaño $n$ con una 
probabilidad de $1/2$.  Mejorando el tiempo empleado por el mejor algoritmo 
probabilístico conocido de $O(n)$ a $O(\sqrt{n})$.

Una de las dificultades a la hora de diseñar un algoritmo cuántico es la gran 
diferencia de comportamiento, comparado con un algoritmo convencional o clásico.  
Nuestra intuición y formas actuales para diseñar algoritmos, fallan al tratar de 
comprender el proceso. Este es quizás uno de los motivos por los cuales no se 
han descubierto muchos algoritmos cuánticos \cite{shor03}.

\section{El bit y el qubit}
La forma de almacenar la información en un ordenador cuántico difiere en algunos 
aspectos a la forma en la que se almacena actualmente en los ordenadores 
convencionales.

\subsection{Bit clásico}
%Explicar lo que es un bit, y diferenciar entre el estado interno y lo que 
%representa
En los ordenadores actuales, la unidad básica de representación de información 
es el bit. La palabra bit significa dígito binario, que puede ser 0 o 1. Es el 
nexo de unión entre los sistemas físicos que implementan un bit, y una lógica 
binaria, que se abstrae de su implementación.

La forma en la que se implementa un bit depende de la arquitectura: la posición 
de una leva mecánica en la máquina, la existencia de una presilla en una cinta 
de vídeo (o también de cinta adhesiva), el estado de un interruptor, la 
presencia o ausencia de un agujero en una tarjeta perforada, anillos de ferrita 
que se magnetizan en un sentido o en otro, dos niveles de voltaje diferentes...

Todas estas representaciones, diferentes en su naturaleza física, comparten dos 
propiedades en común; es posible representar dos estados diferentes, y además es 
posible modificar y leer el estado en el que se encuentra el sistema.

De esta forma es importante distinguir entre la \textit{representación física} y 
el \textit{significado lógico}. En el caso de la tarjeta perforada, esta 
relación puede darse de la siguiente forma: Si hay un agujero, entonces 
representa un 0; en caso contrario, un 1. Para representar un bit, se empleará 
un estado $\V x$, de forma que los bits 0 y 1 se definen respectivamente como
$$ \V x^0 = \mat{1 \\ 0}, \quad \V x^1 = \mat{0 \\ 1}, \quad
\V x \in \{\V x^0, \V x^1\} $$


\subsection{Bit probabilístico}
%Introducir la analogía de una moneda para diferenciar entre el estado de la 
%moneda que es 0.5/0.5 mientras no se mide, y una vez "colapsa". Expresar la 
%probabilidad en forma de vector

Una moneda $A$ perfecta que se lanza al aire, y que cae sobre una superficie 
lisa, saca cara con igual probabilidad que cruz. Mientras se encuentra en el 
aire dando vueltas, su estado no es ni cara ni cruz. Es un estado diferente.  
Este estado puede representarse mediante un bit probabilístico \cite{watrous}, 
de forma que la cara es el 0, la cruz es el 1, y ambos tienen una probabilidad 
de 1/2 de ocurrir.  Escribiendo las probabilidades en un vector, se define
%
$$ \V y = \mat{a \\ b} = \mat{1/2 \\ 1/2} $$
%
Comenzando la numeración en 0, el primer elemento $\V y [0]$ representa la 
probabilidad de que el sistema saque un 0 (de que salga cara). Y $\V y [1]$ de 
que saque un 1 (que salga cruz). El hecho de que el índice coincida con el 
estado que representa, será útil en el futuro.
%
Además, debido a que se trata de un sistema probabilístico, ambas probabilidades 
han de sumar la unidad.
%
$$ \V y [0] + \V y [1] = a + b = 1 $$
%
Una vez que la moneda cae, su estado deja de ser un bit probabilístico $\V y$, y 
se convierte en un bit clásico $\V x$. Esta operación es la \textit{medición} de 
un sistema probabilístico.

\subsection{Bit cuántico o qubit}
%Extender el ejemplo de la moneda con la amplitud en vez de la probabilidad, y 
%explicar la notación en forma de vector

Hasta ahora, la clase de sistemas que se han descrito, son conocidos por 
experiencias en la vida cotidiana o profesional, y sirven de analogías. Sin 
embargo, el funcionamiento del sistema que se describirá a continuación no tiene 
un ejemplo conocido, tan sólo la imaginación será capaz de construir dicho 
sistema.
%
Un sistema cuántico puede encontrarse en un estado $\V z$, definido como
%
$$ \V z = \mat{\alpha \\ \beta}$$
%
Además, tanto $\alpha$ como $\beta$ son números complejos. Este estado se 
denomina bit cuántico o qubit.
%
Tras medir un qubit, el sistema se convertirá en un bit clásico $\V x$, con una 
probabilidad de que salga 0 de $|\alpha|^2$ y de que salga 1 de $|\beta|^2$.
Es decir, que se comportará como un bit probabilístico con las probabilidades
%
$$ \V y = \mat{a \\ b} = \mat{|\alpha|^2 \\ |\beta|^2}$$
%
Y de igual modo, cumple la restricción de que ambas probabilidades suman la 
unidad:
%
$$ a + b = |\alpha|^2 + |\beta|^2 = 1 $$
%
Los números $\alpha$ y $\beta$ se denominan \textit{amplitudes}, y tienen 
asociada una probabilidad que es $|\alpha|^2$ y $|\beta|^2$ respectivamente.

\section{Múltiples qubits}
% Combinar dos bits para representar 4 estados, dos monedas para 4 estados
% probabilísticos, y 2 qubits para un 2-qubit.
Un sistema de dos monedas perfectas $A$ y $B$ tiene exactamente cuatro posibles 
resultados tras lanzarlas al aire $\{00, 01, 10, 11\}$. Siendo cara el 0, cruz 
el 1, y la cadena 01 que $A$ sale cara y $B$ sale cruz.

El número de resultados posibles aumenta con el número de monedas $n$ de la 
forma $2^n$. Si ambas monedas se describen como un bit probabilístico, se 
obtiene
$$ \V y_A = \mat{1/2 \\ 1/2} \quad \V y_B = \mat{1/2 \\ 1/2} $$
Siendo $\V y_m$ la descripción del estado de la moneda $m$. Entonces, la 
probabilidad de obtener cada uno de los cuatro diferentes estados al lanzar 
ambas, se puede describir en forma de vector.
Sea $\V y_m[k]$ la probabilidad de que la moneda $m$ salga $k$, entonces la 
probabilidad de que salga $k_A$ y luego $k_B$ será:
$$ \V y_A[k_A] \cdot \V y_B[k_B] $$
De modo que se puede construir un vector $\V y$ que describa todas las 
posibilidades:
$$ \V y = \mat{
	\V y_A[0] \cdot \V y_B[0] \\ \V y_A[0] \cdot \V y_B[1] \\
	\V y_A[1] \cdot \V y_B[0] \\ \V y_A[1] \cdot \V y_B[1]
}
= \mat{
	\V y_A[0] \cdot \V y_B \\
	\V y_A[1] \cdot \V y_B
} $$
Esta operación se conoce como el producto tensorial y conocer sus propiedades
será fundamental.

\subsection{Producto tensorial}
\label{sss:producto-tensorial}

La operación $\otimes$ se define como el producto tensorial.  Si $A$ y $B$ 
son dos matrices de $n \times m$ y $k \times l$ respectivamente:
%
$$
\begin{array}{c c}
	A=\mat{
		a_{11} & \cdots & a_{1m} \\
		\vdots &        & \vdots \\
		a_{n1} & \cdots & a_{nm}
	}
	,\quad
	&
	B=\mat{
		b_{11} & \cdots & b_{1l} \\
		\vdots &        & \vdots \\
		b_{k1} & \cdots & b_{kl}
	}
\end{array}
$$
%
Entonces $C = A \otimes B$ es la matriz $C$ de $nk \times ml$, 
definida como
%
$$
C = A \otimes B = \mat{
	a_{11} B & \cdots & a_{1m} B \\
	\vdots   &        & \vdots   \\
	a_{n1} B & \cdots & a_{nm} B
}
$$
%
El producto tensorial cumple varias propiedades importantes.
%
\begin{itemize}
\item Es asociativo $(A \otimes B) \otimes C = A \otimes (B \otimes C)$

\item Es distributivo respecto a la suma $A \otimes (B + C) = (A \otimes B) + (A
\otimes C) $ y $(A + B) \otimes C = (A \otimes C) + (B \otimes C)$

\item Es distributivo respecto a la multiplicación de matrices $A \otimes (B 
\cdot C) = (A \otimes B) \cdot (A \otimes C) $ y $(A \cdot B) \otimes C = (A 
\otimes C) \cdot (B \otimes C)$

\item Para un escalar $k$, cumple $(kA) \otimes B = A \otimes (kB) = k (A 
\otimes B)$
\end{itemize}
%
Pero en general, no es conmutativo: $A \otimes B \neq B \otimes A$.

\subsection{Monedas y qubits}

Empleando la notación de producto tensorial, ahora calcular el estado global de 
un sistema de dos monedas, se simplifica a
%
$$ \V y = \V y_A \otimes \V y_B $$
%
Este mismo procedimiento se emplea en la descripción de qubits. Un sistema $\V 
z$ de dos qubits $\V z_A$ y $\V z_B$ se puede describir como:
%
$$
\V z_A = \mat{\alpha_1 \\ \beta_1}, \quad
\V z_B = \mat{\alpha_2 \\ \beta_2}, \quad
\V z = z_A \otimes z_B $$
%
De forma que se obtiene:
%
$$
\V z = \mat{\alpha_1 \alpha_2 & \alpha_1 \beta_2
	& \beta_1 \alpha_2 & \beta_1 \beta_2}^T $$
%
Por ahora no hay ninguna diferencia apreciable en la forma en la que comporta un 
sistema cuántico frente a uno probabilístico. Pero no será durante mucho tiempo.
El sistema de dos monedas $\V y$ está sujeto a la restricción de que al lanzar 
una moneda la suma de las probabilidades de los posibles resultados debe ser la 
unidad, esto es
%
$$ \V y_A[0] + \V y_A[1] = 1, \quad \V y_B[0] + \V y_B[1] = 1$$
%
Cuando dos qubits se tratan por separado, esta restricción es análoga:
%
$$|\alpha_1|^2 + |\beta_1|^2 = 1, \quad |\alpha_2|^2 + |\beta_2|^2 = 1$$
%
Pero cuando se combinan en un sólo sistema cuántico, pasa a denominarse 
\textit{registro cuántico}, y la restricción se convierte en:
\begin{equation}
\label{eq:qbits-unidad}
\sum_{\alpha \in \V z} |\alpha|^2 = 1
\end{equation}
%
Permitiendo el caso en el que puede existir un registro de dos qubits con el 
estado:
$$ \V z = \mat{\frac{1}{\sqrt{2}} & 0 & 0 & \frac{1}{\sqrt{2}}}^T $$
Que cumple la restricción \eqref{eq:qbits-unidad} puesto que:
$$ \abs{\frac{1}{\sqrt{2}}}^2 + \abs{\frac{1}{\sqrt{2}}}^2 = \frac{1}{2} + 
\frac{1}{2} = 
1 $$
Y sin embargo, $\V z$ no puede ser expresado como el producto tensorial de dos 
qubits $\V z_A$ y $\V z_B$:
$$ \V z = \V z_A \otimes \V z_B = \mat{\alpha_1 \alpha_2 & \alpha_1 \beta_2
	& \beta_1 \alpha_2 & \beta_1 \beta_2}^T = \mat{\frac{1}{\sqrt{2}} & 0 & 0 & 
\frac{1}{\sqrt{2}}}^T
$$
Se obtiene que $\alpha_1 \alpha_2 = \frac{1}{\sqrt{2}}$, por lo que $\alpha_1 
\neq 0$, y $\beta_1 \beta_2 = \frac{1}{\sqrt{2}}$ entonces $\beta_2 \neq 0$ 
pero sin embargo $\alpha_1 \beta_2 = 0$, lo cual es imposible.

Este extraño suceso no existe en el mundo clásico, y no se puede realizar con 
las monedas. Es único del comportamiento cuántico. Se denomina 
\textit{entrelazamiento cuántico} cuando un sistema cuántico compuesto no puede 
ser descrito mediante el producto de sus constituyentes. Se dice que se 
encuentra entrelazado.

\section{Notación de Dirac o bra-ket}
A medida que aumenta el número de qubits de un registro cuántico, se hace cada 
vez más difícil describir los $2^n$ componentes que requiere dicho vector.

El matemático Hermann Grassmann empleó la notación $[u|v]$ en 1862 para indicar 
el producto interno de dos vectores \cite{cajori-grassmann}. Posteriormente, en 
1939 el físico Paul Dirac empleó $\braket{u|v}$ para describir estados 
cuánticos, y dicha notación se ha extendido debido a su utilidad. Se denomina 
braket a $\braket{u|v}$, formado por el bra $\bra{u}$ y el ket $\ket{v}$.
%
De esta forma, se definen los kets
$$ \ket{0} = \mat{1 \\ 0}, \quad \ket{1} = \mat{0 \\ 1} $$
%
Y de forma correspondiente, los bras, como duales de los kets
$$ \bra{0} = \ket{0}^\dagger = \mat{1 & 0}, \quad
\bra 1 = \ket{1}^\dagger = \mat{0 & 1} $$
%
Entonces un qubit $\V z$ puede describirse como una combinación lineal de kets 
$\ket{0}$ y $\ket{1}$
$$ \V z = \mat{\alpha \\ \beta} = \alpha \ket{0} + \beta \ket{1}$$
Dado que cualquier qubit puede escribirse partiendo de los kets $\ket{0}$ y 
$\ket{1}$, forman una base vectorial y se les denomina vectores base, 
\textit{kets base} o también \textit{estados puros}.
%
De forma general, se define un estado cuántico o registro de $n$ qubits como el 
ket $\ket{\psi}$ con $X = \{0,1\}^n$:
$$ \ket{\psi} = \sum_{x \in X} \alpha_x \ket{x} $$
Por ejemplo para $n = 2$, se tiene $X = \{00, 01, 10, 11\}$, y el ket 
$\ket{\psi}$:
$$ \ket{\psi} = \alpha_{00} \ket{00} + \alpha_{01} \ket{01}
+ \alpha_{10} \ket{10} + \alpha_{11} \ket{11} =
\mat{\alpha_{00} & \alpha_{01} & \alpha_{10} & \alpha_{11}}^T $$
El conjunto de kets $\ket{x}$ para todo $x \in X$ forma de nuevo una base de 
vectores, para el registro $\ket{\psi}$. Al escribir un ket como $\ket{01}$ se 
hace referencia al producto tensorial:
$$ \ket{01} = \ket{0} \otimes \ket{1} $$
Y también se puede escribir sin el símbolo $\otimes$, como $\ket{0}\ket{1}$. De 
igual forma, para estados más grandes como $\ket{0110} = 
\ket{0}\ket{1}\ket{1}\ket{0}$.

\subsection{Superposición}
% Que es y que utilidad tiene

Cuando un qubit o un registro se encuentra en un estado diferente a un estado 
puro, por ejemplo el ket
$$\ket{\psi} = \frac{1}{\sqrt{2}} \ket{0} + \frac{1}{\sqrt{2}} \ket{1} $$
se dice que se encuentra en \textit{superposición}. El qubit se comporta como si 
se encontrase en ambos estados $\ket{0}$ y $\ket{1}$ a la vez, y cuando se 
observa, colapsa en un estado puro. A diferencia de una moneda que se lanza, 
existe la forma de determinar el estado de la moneda a medida que se mueve por 
el aire. Con una cámara que grabe a una alta velocidad, se puede observar como 
la moneda da vueltas oscilando entre cara y cruz. Sin embargo no existe ninguna 
forma actual para determinar el estado interno de un qubit o registro.

Adicionalmente, uno de los puntos clave de la computación cuántica, se basa en 
que es posible operar con ambos estados a la vez, obteniendo un paralelismo que 
crece de forma exponencial con el número de qubits. Un sistema de $n$ qubits 
puede realizar cálculos operando directamente con las $2^n$ amplitudes asociadas 
a cada estado puro. En un sistema clásico, se requieren $2^n$ registros de bits 
para alcanzar un cálculo similar.

\section{Operaciones}

Las operaciones que se pueden realizar sobre un qubit se representan como una 
matriz $A$ de $2 \times 2$ elementos complejos y se denominan 
\textit{operadores}.

Además, todos los operadores cuánticos deben ser \textit{reversibles}, es decir 
que la matriz $A$ debe ser \textit{unitaria}, cumpliendo que $A$ por su 
transpuesta conjugada $A^\dagger$ sea la identidad:
$$ A A^\dagger = A^\dagger A = I$$
%
Cuando se aplica un operador $A$ sobre un qbit $\ket{\psi}$, se obtiene un 
estado resultante $\ket{\psi'}$ obtenido por el producto matricial
$$ \ket{\psi'} = A\ket{\psi} $$
%
A los operadores cuánticos también se les denomina \textit{puertas cuánticas}, 
por su parecido con las puertas lógicas. Por ejemplo la puerta lógica NOT, 
invierte el estado de un bit, y su correspondiente cuántica es
%
$$ NOT = \mat{0 & 1\\1 & 0} $$
%
Que aplicada sobre los estados puros los invierte.
%
$$ NOT\ket{0} = \mat{0 & 1\\1 & 0} \mat{1\\0} = \mat{0\\1} = \ket{1} $$
$$ NOT\ket{1} = \mat{0 & 1\\1 & 0} \mat{0\\1} = \mat{1\\0} = \ket{0} $$
%
Y sobre un estado en superposición $\ket{\psi}$, invierte sus amplitudes
%
$$ NOT\ket{\psi} = \mat{0 & 1\\1 & 0} \mat{\alpha \\ \beta} =
\mat{\beta\\\alpha} $$
%

\subsection{Operadores de varios qubits}

Además de poder aplicar un operador sobre un qubit, es posible hacerlo sobre un 
registro de varios qubits. En concreto, para $n$ qubits, la matriz $A$ que 
describe al operador contiene $2^n \times 2^n$ filas y columnas.

Una forma de construir operadores grandes es aplicando el producto tensorial de 
varios operadores simples. Por ejemplo, el operador que invierte de igual forma 
que el $NOT$ un registro de dos qubits, se calcula como
%
$$ NOT \otimes NOT = \mat{0&0&0&1\\ 0&0&1&0 \\ 0&1&0&0 \\ 1&0&0&0}$$
%
Que al aplicarse sobre el estado $\ket{01}$ se convierte en:
%
\begin{equation*}
\begin{split}
(NOT \otimes NOT) \ket{01} &= (NOT \otimes NOT) (\ket{0} \otimes \ket{1}) =\\
& = NOT \ket{0} \otimes NOT \ket{1} = \ket{1} \otimes \ket{0} = \ket{10}
\end{split}
\end{equation*}
%
Obteniéndose el estado invertido $\ket{10}$. La operación $NOT \otimes NOT$ que 
consiste en multiplicar tensorialmente un operador por si mismo, es equivalente 
a la potencia tensorial de operadores, y se indica como
%
$$ NOT \otimes NOT = NOT^{\otimes 2} $$
%
De forma que se puede extender para cualquier exponente.
%
$$ NOT^{\otimes n} = \underbrace{NOT\otimes NOT\otimes \ldots \otimes NOT}_{n 
\text{ veces}}$$
%

\subsubsection{Operadores comunes}

Existen una serie de operadores que se emplean con mucha frecuencia en la 
computación cuántica. Uno de los más importantes es el operador de 
\textit{Hadamard}, representado por la letra $H$.
%
$$ H = \frac{1}{\sqrt 2} \mat{1 & 1 \\ 1 & -1} $$
%
Tiene la peliculiaridad de descomponer el estado puro $\ket{0}$ en un estado 
superpuesto:
%
$$ H \ket{0} = \frac{1}{\sqrt 2}\ket{0} + \frac{1}{\sqrt 2}\ket{1} $$
%
% TODO Explicar la fórmula general del operador de Hadamard
El operador de Hadamard para $n$ qubits $H^{\otimes n}$, aplicado sobre un 
estado $\ket{x}$, se puede generalizar de la forma
%
$$ H^{\otimes n} \ket{x} = \frac 1 {\sqrt {2^n}} \sum_z (-1)^{\braket{x|z}} 
\ket{z}$$
%
Donde $\braket{x|z}$ representa el producto interno de $x$ y $z$, bit a bit 
módulo dos:
%
$$ \braket{x|z} = x_1 z_1 + x_2 z_2 + \ldots + x_n z_n \mod 2$$
%
El operador identidad $I$ deja un estado tal y como estaba, no realiza ninguna 
acción, y se define
%
$$ I = \mat{1&0 \\ 0&1}$$
%
Otros son el desplazamiento de fase $R_\theta$, la puerta SWAP que intercambia 
dos qubits, la puerta not controlada CNOT que realiza la negación del segundo 
qubit cuando el primero es $\ket{1}$.
$$ R_\theta = \mat{1 & 0 \\ 0 & e^{i\theta}}, \quad
SWAP = \mat{1&0&0&0 \\ 0&0&1&0 \\ 0&1&0&0 \\ 0&0&0&1}, \quad
CNOT = \mat{1&0&0&0 \\ 0&1&0&0 \\ 0&0&0&1 \\ 0&0&1&0}
$$

%Medición

\subsection{Varias operaciones}

Cuando se realizan varias operaciones seguidas sobre un estado $\ket{\psi}$, el 
resultado es el producto de las matrices de los operadores por el estado.  
Siguiendo el orden en el que fueron aplicados. Por ejemplo, aplicar el operador 
de Hadamard sobre el estado $\ket{0}$ y posteriormente el NOT, se representa
%
$$ NOT(H\ket{0}) = NOT\; H \ket{0} $$
%
El orden viene dado por el producto de matrices, de derecha a izquierda. Y se 
calcula:
%
$$ NOT\; H \ket{0} = NOT\left(\frac{1}{\sqrt 2}\ket{0} + \frac{1}{\sqrt 
2}\ket{1}\right) =
\frac{1}{\sqrt 2}\ket{1} + \frac{1}{\sqrt 2}\ket{0} = \frac{1}{\sqrt 2}(\ket{0} 
+ \ket{1}) $$

\subsubsection{Circuitos cuánticos}

Los \textit{circuitos cuánticos} son agrupaciones de operadores aplicados en un 
orden determinado, que tienen un propósito. Se pueden describir como una matriz 
resultante de multiplicar todas las matrices que operan de forma sucesiva.

A medida que se combinan los operadores, es posible crear complejas estructuras 
que son difíciles de comprender empleando la notación de producto de matrices.  
Los circuitos cuánticos se describen habitualmente con diagramas que incluyen 
varios elementos.

Las líneas horizontales representan qubits o registros de qubits. Las puertas 
cuánticas se colocan intersecando las líneas que representan los qubits en los 
que actúan.

Por ejemplo el operador de Hadamard, aplicado sobre el qubit $\ket{0}$, se 
representa con el siguiente circuito.
%
\begin{center}
	\begin{tikzpicture}%[thick]
	% `operator' will only be used by Hadamard (H) gates here.
	\tikzstyle{operator} = [draw,fill=white,minimum size=1.5em] 
	%
	\matrix[row sep=0.4cm, column sep=1cm] (circuit) {
		% First row
		\node (q1) {$\ket{0}$}; &
		\node[operator] (H11) {$H$}; &
		\coordinate (end1); \\
		% Second row.
	};

	\begin{pgfonlayer}{background}
		% Draw lines.
		\draw[thick] (q1) -- (end1);
	\end{pgfonlayer}
	%
\end{tikzpicture}
\end{center}
%
Si se aplica primero el operador de Hadamard y luego la puerta NOT, se obtiene
%
\begin{center}
	\begin{tikzpicture}%[thick]
	% `operator' will only be used by Hadamard (H) gates here.
	\tikzstyle{operator} = [draw,fill=white,minimum size=1.5em] 
	%
	\matrix[row sep=0.4cm, column sep=1cm] (circuit) {
		% First row
		\node (q1) {$\ket{0}$}; &
		\node[operator] (H11) {$H$}; &
		\node[operator] (H11) {$NOT$}; &
		\coordinate (end1); \\
		% Second row.
	};

	\begin{pgfonlayer}{background}
		% Draw lines.
		\draw[thick] (q1) -- (end1);
	\end{pgfonlayer}
	%
\end{tikzpicture}
\end{center}
%
También se pueden realizar operaciones en \textit{paralelo}. En un sistema de 
dos qubits, en el que el que se aplica un operador de forma simultánea en dos 
líneas, corresponde con el producto tensorial.

En el caso de aplicar el operador de Hadamard sobre los dos qubits del registro 
$\ket{00}$ se representa
%
\begin{center}
	\begin{tikzpicture}%[thick]
	% `operator' will only be used by Hadamard (H) gates here.
	\tikzstyle{operator} = [draw,fill=white,minimum size=1.5em] 
	%
	\matrix[row sep=0.4cm, column sep=1cm] (circuit) {
		% First row
		\node (q1) {$\ket{0}$}; &
		\node[operator] (H11) {$H$}; &
		\coordinate (end1); \\
		% Second row.
		\node (q2) {$\ket{0}$}; &
		\node[operator] (H21) {$H$}; &
		\coordinate (end2); \\
	};

	\begin{pgfonlayer}{background}
		% Draw lines.
		\draw[thick] (q1) -- (end1);
		\draw[thick] (q2) -- (end2);
	\end{pgfonlayer}
	%
\end{tikzpicture}
\end{center}
%
Y corresponde con el resultado de calcular $(H \otimes H) \ket{00}$. De la misma 
forma, combinando operaciones secuenciales y en paralelo, se pueden construir 
circuitos complejos, que se representan de forma sencilla.
%
\begin{center}
	\begin{tikzpicture}%[thick]
	% `operator' will only be used by Hadamard (H) gates here.
	\tikzstyle{operator} = [draw,fill=white,minimum size=1.5em] 
	%
	\matrix[row sep=0.4cm, column sep=1cm] (circuit) {
		% First row
		\node (q1) {$\ket{0}$}; &
		\node[operator] (H11) {$H$}; &
		\node[operator] (NOT) {$NOT$}; &
		\node[operator] (H13) {$H$}; &
		\coordinate (end1); \\
		% Second row.
		\node (q2) {$\ket{0}$}; &
		\node[operator] (H21) {$H$}; &
		&
		&
		\coordinate (end2); \\
	};

	\begin{pgfonlayer}{background}
		% Draw lines.
		\draw[thick] (q1) -- (end1);
		\draw[thick] (q2) -- (end2);
	\end{pgfonlayer}
	%
\end{tikzpicture}
\end{center}
%
Que es equivalente a $(H \otimes H)(NOT\otimes I)(H \otimes I) \ket{00} $. Se 
observa como las líneas son equivalentes a aplicar el operador identidad $I$.
%
\begin{center}
	\begin{tikzpicture}%[thick]
	% `operator' will only be used by Hadamard (H) gates here.
	\tikzstyle{operator} = [draw,fill=white,minimum size=1.5em] 
	%
	\matrix[row sep=0.4cm, column sep=1cm] (circuit) {
		% First row
		\node (q1) {$\ket{0}$}; &
		\node[operator] (H11) {$H$}; &
		\node[operator] (NOT) {$NOT$}; &
		\node[operator] (H13) {$H$}; &
		\coordinate (end1); \\
		% Second row.
		\node (q2) {$\ket{0}$}; &
		\node[operator] (H21) {$H$}; &
		\node[operator] (I22) {$I$}; &
		\node[operator] (I23) {$I$}; &
		\coordinate (end2); \\
	};

	\begin{pgfonlayer}{background}
		% Draw lines.
		\draw[thick] (q1) -- (end1);
		\draw[thick] (q2) -- (end2);
	\end{pgfonlayer}
	%
\end{tikzpicture}
\end{center}
%

\subsubsection{Operadores de medición}

Existe una operación que es diferente de todas las demás puertas cuánticas. Se 
trata de los operadores de medición, representados por la letra $M$, o otras 
veces por el dibujo de un aparato de medida analógico.
%
\begin{center}
	\begin{tikzpicture}%[thick]
	% `operator' will only be used by Hadamard (H) gates here.
	\tikzstyle{operator} = [draw,fill=white,minimum size=1.5em] 
	%
	\matrix[row sep=0.4cm, column sep=1cm] (circuit) {
		% First row
		\node (q1) {}; &
		\node[operator] (M11) {$M$}; &
		\coordinate (end1); \\
		% Second row.
	};
	\node[fill=white, fit=(end1)] (cover) {};

	\begin{pgfonlayer}{background}
		% Draw lines.
		\draw[thick] (q1) -- (M11);
		\draw[thick, double, double distance=2pt] (M11) -- (end1);
	\end{pgfonlayer}
	%
\end{tikzpicture}
\end{center}
%
Este operador realiza la medición de una línea, y produce un resultado binario.  
Destruyendo el estado cuántico en el proceso de medida. El resultado de medir un 
qubit será un bit. En el caso de medir un registro, será un resultado binario 
correspondiente a sus estados puros. Las dos líneas paralelas que salen del 
circuito tras el operador $M$ representan una línea de información clásica.

No puede ser representado por una matriz unitaria, ya que el resultado no es 
reversible.  Una vez realizada una medición, el estado cuántico se colapsa. Una 
de las interesantes cuestiones de los estados entrelazados como el estado de 
Bell denominado por $\ket{\beta_{00}}$:
%
$$ \ket{\beta_{00}} = \frac{1}{\sqrt{2}}(\ket{00} + \ket{11}) $$
%
Es que es posible realizar una medida sobre uno de sus qubits. El resultado será 
0 o 1 con una probabilidad del 50\%. Inmediatamente al conocer el estado de uno 
de los qubits, el otro también se conoce. Este hecho, que intrigó a Einstein, 
Podolsky y Rosen \cite{EPR}, produjo la formulación de la paradoja EPR, en la 
que se discutía como era posible que al realizar la medición de un qubit, el 
otro inmediatamente se alterase aunque ambos qubits estuvieran separados.

Este hecho, ha sido comprobado de forma experimental con resultados positivos, 
contradiciendo totalmente la intuición. Además es posible aprovechar sus 
consecuencias para ``teletransportar'' información cuántica a través de un canal 
clásico de bits.
