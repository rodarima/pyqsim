\chapter{Análisis de la simulación}
% Comentar que es lo que se va a medir. Como se realiza el proceso de medición, 
% y finalmente cuales son los resultados.

El proceso de la simulación se analiza exhaustivamente para determinar por una 
parte la \textit{eficiencia} del simulador, midiendo el tiempo empleado y la 
memoria. Además también se observa el número de ejecuciones del algoritmo 
cuántico, para calcular su \textit{complejidad}.

%%
%\begin{center}
%% Define block styles
%\tikzstyle{decision} = [diamond, draw, text width=4.5em, text badly centered, 
%inner sep=0pt]
%\tikzstyle{block} = [rectangle, draw, text width=5em, text centered, minimum 
%height=2cm]
%\tikzstyle{line} = [draw, thick, decoration={markings,mark=at position 
%1 with {\arrow[scale=1.5]{latex'}}}, postaction={decorate}]
%%
%\begin{tikzpicture}[node distance = 3cm, auto]
%	% Place nodes
%	\node[block] (qc0) {Simulación cuántica inicial};
%	\node[block, right of=qc0] (qcf) {Finalizar simulación cuántica};
%	\node[block, right of=qcf] (measure) {Medición};
%	\node[block, right of=measure] (classic) {Procesado clásico};
%	% Draw edges
%	\draw [line] (qc0.west)+(-1cm,0) -- (qc0.west);
%	\draw [line] (qc0) -> (qcf);
%	\draw [line] (qcf) -- (measure);
%	\draw [line] (measure) -- (classic);
%	\draw [line] (qc0.-30) -| +(+1em,-1cm) -| node [near start, above] {$M_1$} 
%(qc0)+(-2em,-1em);
%	\draw [line] (qcf.-30) -| +(+1em,-1cm) -| node [near start, above] {$M_2$} 
%(qcf)+(-2em,-1em);
%	\draw [line] (measure.-30) -| +(+1em,-1cm) -| node [near start, above] {$M_2$} 
%(measure)+(-2em,-1em);
%	\draw [line] (classic.-30) -| +(+1em,-1cm) -| node [near start, above] {$M_2$} 
%(classic)+(-2em,-1em);
%	\draw [line] (classic.east) -- +(+1cm,0);
%
%\end{tikzpicture}
%\end{center}
%%

\section{Análisis del tiempo de CPU}
\textsl{``En casi todo cómputo son posibles una gran variedad de configuraciones 
para la sucesión de un proceso, y varias consideraciones pueden influir en la 
selección de estas según el propósito de un motor de cálculos. Un objetivo 
esencial es escoger la configuración que tienda a minimizar el tiempo necesario 
para completar el cálculo.''}---Augusta Ada Lovelace.
\footnote{Firmó las notas sobre la máquina analítica de Babbage en 1843 como 
A.A.L. para evitar la censura por ser una mujer. Fue la primera persona de la 
historia que creó un programa para ser ejecutado en una máquina (pese a que aún 
no había sido construida).}
%Augusta Ada Byron, Condesa de Lovelace.
\newline

La simulación debe ser realizada teniendo en cuenta el tiempo de procesamiento 
requerido por la CPU. Es importante investigar cómo reducirlo para conseguir que 
la simulación sea eficiente.

Para calcular con precisión los datos medidos sobre cada simulación, se realizan 
varias ejecuciones, y posteriormente se analiza la media y varianza de las 
medidas obtenidas. El análisis divide la simulación en cuatro etapas. Las dos 
primeras, $QC_0$ y $QC_f$, realizan la simulación del circuito cuántico. A 
continuación se analiza el proceso de medición $M$ que provee un nexo entre la 
parte cuántica y la etapa final de procesado clásico $CC$.

Cada etapa se mide repetidamente y de forma independiente, permitiendo una mayor 
precisión en la medida de la complejidad. El esquema se muestra a continuación.
%
\begin{center}
% Define block styles
\tikzstyle{decision} = [diamond, draw, text width=4.5em, text badly centered, 
inner sep=0pt]
\tikzstyle{block} = [rectangle, draw, text width=2em, text centered, minimum 
height=1cm]
\tikzstyle{line} = [draw, thick, decoration={markings,mark=at position 
1 with {\arrow[scale=1.5]{latex'}}}, postaction={decorate}]
%
\begin{tikzpicture}[node distance = 2.5cm, auto]
	% Place nodes
	\node[block] (qc0) {$QC_0$};
	\node[block, right of=qc0] (qcf) {$QC_f$};
	\node[block, right of=qcf] (measure) {$M$};
	\node[block, right of=measure] (classic) {$CC$};
	% Draw edges
	\draw [line] (qc0.west)+(-1cm,0) -- (qc0.west);
	\draw [line] (qc0) -> (qcf);
	\draw [line] (qcf) -- (measure);
	\draw [line] (measure) -- (classic);
	\draw [line] (qc0.-30) -| +(+1em,-1cm) -| node [near start, above] {} 
	(qc0)+(-2em,-1em);
	\draw [line] (qcf.-30) -| +(+1em,-1cm) -| node [near start, above] {} 
	(qcf)+(-2em,-1em);
	\draw [line] (measure.-30) -| +(+1em,-1cm) -| node [near start, above] 
	{} (measure)+(-2em,-1em);
	\draw [line] (classic.-30) -| +(+1em,-1cm) -| node [near start, above] 
	{} (classic)+(-2em,-1em);
	\draw [line] (classic.east) -- +(+1cm,0);

\end{tikzpicture}
\end{center}
%

\subsection{Simulación cuántica}
El análisis de la simulación cuántica, se divide en dos procesos. La parte 
inicial $QC_0$, calcula el estado intermedio $\ketp 1$ que es independiente de 
la función $f$ del problema. Permitiendo la reutilización de los cálculos en las 
etapas posteriores.

La parte final del análisis $QC_f$, toma el estado $\ketp 1$ previamente 
calculado, y continúa la simulación del circuito hasta el estado final $\ketp 
3$. Para medir el tiempo que toma una etapa se denotará como $T_\mu(x)$ el 
tiempo medio en segundos de $r$ ejecuciones, y $T_{\sigma^2}(x)$ la varianza del 
proceso $x$. Por lo general se realizarán $r = 100$ repeticiones, excepto si el 
proceso se demora demasiado, entonces se reducirán las iteraciones para mantener 
un tiempo de simulación razonable.

\begin{table}[!htb]
\centering
\pgfplotstabletypeset[
	col sep=comma,
	columns={n, N, R, qc0-mean, qc0-var,qcf-mean, qcf-var},
	columns/n/.style={column name=$n$, column type={r}},
	columns/N/.style={column name=$N$, column type={r}},
	columns/R/.style={column name=$r$, column type={r}},
	columns/qc0-mean/.style={
		column name=$T_\mu(QC_0)$,
		sci,sci zerofill,precision=2,dec sep align
	},
	columns/qc0-var/.style={
		column name=$T_{\sigma^2}(QC_0)$,
		sci,sci zerofill,precision=2,dec sep align
	},
	columns/qcf-mean/.style={
		column name=$T_\mu(QC_f)$,
		sci,sci zerofill,precision=2,dec sep align
	},
	columns/qcf-var/.style={
		column name=$T_{\sigma^2}(QC_f)$,
		sci,sci zerofill,precision=2,dec sep align
	},
	every head row/.style={before row=\toprule,after row=\midrule},
	every last row/.style={after row=\bottomrule},
]{csv/table_cpu.csv}
\caption{Tiempo empleado por $QC_0$ y $QC_f$ en segundos.}
\label{tab:cpu-qc}
\end{table}

Se observa en la tabla~\ref{tab:cpu-qc} como se incrementa el tiempo a medida 
que crece en número de qubits. La parte final $QC_f$ requiere un orden de 
magnitud superior que $QC_0$. Ambas requieren un tiempo superior a $O(2^N)$.

\subsection{Tiempo empleado en $M$ y $CC$}
Tras calcular el estado final del circuito, se analiza el proceso de medición 
$M$. Debe calcularse la distribución de probabilidad que asocia a cada posible 
valor tras medir una línea, la probabilidad de que ocurra. Dado que cada línea 
tiene $n$ qubits, serán necesarias $O(2^n)$ operaciones. El proceso de cálculo 
clásico, resuelve el sistema de ecuaciones, una vez se han obtenido $n-1$ 
vectores.
El tiempo obtenido en estas dos últimas etapas, pese a ser exponencial, es mucho 
menor que en la simulación del circuito cuántico, como se observa en la 
tabla~\ref{tab:cpu-m-cc}.

\begin{table}[!htb]
\centering
\pgfplotstabletypeset[
	col sep=comma,
	columns={n, N, R,m-mean, m-var, cc-mean, cc-var},
	columns/n/.style={column name=$n$, column type={r}},
	columns/N/.style={column name=$N$, column type={r}},
	columns/R/.style={column name=$r$, column type={r}},
	columns/m-mean/.style={
		column name=$T_\mu(M)$,
		sci,sci zerofill,precision=2,dec sep align
	},
	columns/m-var/.style={
		column name=$T_{\sigma^2}(M)$,
		sci,sci zerofill,precision=2,dec sep align
	},
	columns/cc-mean/.style={
		column name=$T_\mu(CC)$,
		sci,sci zerofill,precision=2,dec sep align
	},
	columns/cc-var/.style={
		column name=$T_{\sigma^2}(CC)$,
		sci,sci zerofill,precision=2,dec sep align
	},
	every head row/.style={before row=\toprule,after row=\midrule},
	every last row/.style={after row=\bottomrule},
]{csv/table_cpu.csv}
\caption{Tiempo empleado por $M$ y $CC$ en segundos.}
\label{tab:cpu-m-cc}
\end{table}

\subsection{Tiempo total de la simulación}
Debido al carácter exponencial que presenta la simulación, es conveniente 
emplear una representación con escala logarítmica. En la 
figura~\ref{fig:tiempo-qc} se observa como crecen los tiempos de ejecución con 
el número de qubits. Para $N = 20$ es necesario un tiempo de simulación 
aproximado de un minuto. La simulación, pese a ser extremadamente costosa, se 
puede realizar hasta $N \leq 20$.

%
\begin{figure}[!htb]
\centering
\begin{tikzpicture}
\begin{semilogyaxis}[
	name=cpu,
	width=0.9\linewidth,
	height=8cm,
%	no marks,
	mark options={mark size=1},
	thick,
	grid=both,
	xtick={4,...,20},
	xmin=3.5,xmax=20.5,
	legend style={at={(0.0,1.0)},anchor=north west},
	xlabel={Número de qubits $N$},
	ylabel={Tiempo empleado (segundos)},
]
\addplot table [x=N, y=qc0-mean, col sep=comma] {csv/table_cpu.csv};
\addplot table [x=N, y=qcf-mean, col sep=comma] {csv/table_cpu.csv};
\addplot table [x=N, y=m-mean, col sep=comma] {csv/table_cpu.csv};
\addplot table [x=N, y=cc-mean, col sep=comma] {csv/table_cpu.csv};
\addplot [dashed] table [x=N, y=all-mean, col sep=comma] {csv/table_cpu.csv};
\legend{
	$T_\mu(QC_0)$,
	$T_\mu(QC_f)$,
	$T_\mu(M)$,
	$T_\mu(CC)$,
	$T_\mu(SIM)$
};
\end{semilogyaxis}
\end{tikzpicture}
\caption{Tiempo de simulación en escala logarítmica.}
\label{fig:tiempo-qc}
\end{figure}
%

\section{Análisis de espacio}
Para observar el comportamiento de un algoritmo es importante tener en cuenta 
además del tiempo que requiere su ejecución, el espacio que emplea. El simulador 
será analizado paso a paso mostrando como varía la memoria empleada a medida que 
crece el número de qubits del circuito.

\subsection{Simulación cuántica}
La simulación del circuito cuántico se divide en dos etapas, la parte inicial, 
hasta el estado $\ketp 1$ denominada $QC_0$, y la parte final, hasta el estado 
$\ketp 3$, denominada $QC_f$. El esquema del circuito cuántico muestra el orden 
de los operadores, y la posición de los estados.
%
%
\begin{center}
	\begin{tikzpicture}%[thick]
	% `operator' will only be used by Hadamard (H) gates here.
	\tikzstyle{operator} = [draw,fill=white,minimum size=1.5em] 
	%
	\matrix[row sep=0.4cm, column sep=1cm] (circuit) {
		% First row
		\node (q1) {$\ket{0^n}$}; &
		\node[operator] (H11) {$H^{\otimes n}$}; &
		\node[operator] (P13) {}; &
		\node[operator] (H11) {$H^{\otimes n}$}; &
		\node[operator] (M11) {$M$};&
		\coordinate (end1); \\
		% Second row.
		\node (q2) {$\ket{0^n}$}; & & \node[operator] (P23) {}; & & & \coordinate (end2);\\
		% Third row
		\node (q31) {}; & \node (q32) {}; & \node (q33) {}; &
		\node (q34) {}; & \node (q35) {}; & \node (q36) {}; \\
		\node (q41) {}; & \node (q42) {}; & \node (q43) {}; &
		\node (q44) {}; & \node (q45) {}; & \node (q46) {}; \\
	};
	\node[operator] (Uf) [fit = (P13) (P23), minimum width=1cm] {$U_f$};

	\node (arr0) [fit = (q31) (q32)] {$\uparrow$};
	\node (arr1) [fit = (q32) (q33)] {$\uparrow$};
	\node (arr2) [fit = (q33) (q34)] {$\uparrow$};
	\node (arr3) [fit = (q34) (q35)] {$\uparrow$};

	\node (psi0) [fit = (q41) (q42)] {$\ket{\psi_0}$};
	\node (psi1) [fit = (q42) (q43)] {$\ket{\psi_1}$};
	\node (psi2) [fit = (q43) (q44)] {$\ket{\psi_2}$};
	\node (psi3) [fit = (q44) (q45)] {$\ket{\psi_3}$};

	\node[fill=white, fit=(end1) (end2)] (cover) {};

	\begin{pgfonlayer}{background}
		% Draw lines.
		\draw[thick] (q1) -- (M11)  (q2) -- (end2);
		\draw[thick, double, double distance=2pt] (M11) -- (end1);
	\end{pgfonlayer}
	%
\end{tikzpicture}
\end{center}

%
Esquema de análisis de memoria de la simulación completa.
%
\begin{center}
% Define block styles
\tikzstyle{decision} = [diamond, draw, text width=4.5em, text badly centered, 
inner sep=0pt]
\tikzstyle{block} = [rectangle, draw, text width=2em, text centered, minimum 
height=1cm]
\tikzstyle{line} = [draw, thick, decoration={markings,mark=at position 
1 with {\arrow[scale=1.5]{latex'}}}, postaction={decorate}]
%
\begin{tikzpicture}[node distance = 2.5cm, auto]
	% Place nodes
	\node[block] (qc0) {$QC_0$};
	\node[block, right of=qc0] (qcf) {$QC_f$};
	\node[block, right of=qcf] (measure) {$M$};
	\node[block, right of=measure] (classic) {$CC$};
	% Draw edges
	\draw [line] (qc0.west)+(-1cm,0) -- (qc0.west);
	\draw [line] (qc0) -> (qcf);
	\draw [line] (qcf) -- (measure);
	\draw [line] (measure) -- (classic);
	\draw [line] (classic.east) -- +(+1cm,0);
\end{tikzpicture}
\end{center}
%
\subsubsection{Análisis de la primera parte $QC_0$}
% TODO mencionar la optimización de semi-operador.

En $QC_0$, será necesario almacenar en la memoria el operador de Hadamard 
$H^{\otimes n}$, además de un estado cuántico $\ket{\psi}$ de $N = 2n$ qubits.  
Al analizar el comportamiento del circuito, se obtiene la tabla~\ref{tab:qc0}.
%
\begin{table}[!htb]
\centering
\begin{tabular}{rrrrrr}
\toprule
   $n$ &   $N$ &   $\log_2 S(H)$ &   $\log_2 S(\ket{\phi_1})$ &   $\log_2 S_T$ &   $\log_2 S_T'$ \\
\midrule
     2 &     4 &            7.21 &                       6.64 &           7.95 &            7.58 \\
     3 &     6 &            9.10 &                       8.34 &           9.77 &            9.58 \\
     4 &     8 &           11.05 &                      10.17 &          11.68 &           11.58 \\
     5 &    10 &           13.02 &                      12.09 &          13.63 &           13.58 \\
     6 &    12 &           15.01 &                      14.04 &          15.61 &           15.58 \\
     7 &    14 &           17.01 &                      16.02 &          17.60 &           17.58 \\
     8 &    16 &           19.00 &                      18.01 &          19.59 &           19.58 \\
     9 &    18 &           21.00 &                      20.01 &          21.59 &           21.58 \\
    10 &    20 &           23.00 &                      22.00 &          23.59 &           23.58 \\
\bottomrule
\end{tabular}
\caption{Espacio empleado por $QC_0$ en escala logarítmica (bytes).}
\label{tab:qc0}
\end{table}
%
Sea $S(x)$ el tamaño del objeto $x$ en bytes, y $S_T$ el tamaño total requerido 
por la simulación. Entonces el espacio $S_T'$ para $QC_0$ será una aproximación 
al espacio real $S_T$ determinado como
$$ S_T = S(H) + S(\ketp 1) $$
Y aproximado mediante
\begin{equation*}
\begin{split}
	&\log_2 S(H) \approx N + 3 \\
	&\log_2 S(\ketp 1)  \approx N + 2 \\
	&\log_2 S_T \approx N+\log_2 (2^2+2^3) = N+3.58 = \log_2 S_T' \\
	& S_T' = 2^{N} \log_2 12
\end{split}
\end{equation*}
Se observa como el espacio requerido aumenta de forma exponencial a medida que 
aumenta el número de qubits del sistema y se encuentra en torno a $O(2^N)$.



\subsubsection{Análisis de la parte final $QC_f$}
En la etapa final de la simulación del circuito, se sobreescribe $\ketp 1$ con 
el estado final $\ketp 3$, de modo que sólo será necesario un espacio 
equivalente al del más grande.
Además del estado, se necesita el operador $U_f$, que se calcula a partir de la 
función dada $f$. Los tamaños se muestran en la tabla \ref{tab:qcf}.
%
\begin{table}[!htb]
\centering
\begin{tabular}{rrrrrrr}
\toprule
   $n$ &   $N$ &   $\log_2 S(H)$ &   $\log_2 S(U)$ &   $\log_2 S(\ket{\phi_3})$ &   $\log_2 S_T$ &   $\log_2 S_T'$ \\
\midrule
     2 &     4 &            7.21 &            7.61 &                       6.64 &           8.79 &            8.70 \\
     3 &     6 &            9.10 &            9.59 &                       8.60 &          10.74 &           10.70 \\
     4 &     8 &           11.05 &           11.59 &                      10.59 &          12.72 &           12.70 \\
     5 &    10 &           13.02 &           13.59 &                      12.59 &          14.71 &           14.70 \\
     6 &    12 &           15.01 &           15.59 &                      14.59 &          16.70 &           16.70 \\
     7 &    14 &           17.01 &           17.58 &                      16.59 &          18.70 &           18.70 \\
     8 &    16 &           19.00 &           19.58 &                      18.58 &          20.70 &           20.70 \\
     9 &    18 &           21.00 &           21.58 &                      20.58 &          22.70 &           22.70 \\
    10 &    20 &           23.00 &           23.58 &                      22.58 &          24.70 &           24.70 \\
\bottomrule
\end{tabular}
\caption{Espacio empleado por $QC_f$ en escala logarítmica (bytes).}
\label{tab:qcf}
\end{table}

El tamaño requerido en esta etapa de la simulación será $S_T$, aproximado a 
$S_T'$. Además el operador $H$ se reutiliza de la etapa inicial, de modo que ese 
espacio debe tenerse en cuenta. El tamaño del estado $\ketp 3$ es siempre un 
poco más grande que $\ketp 1$. Dado que $\ketp 1$ se reemplazará por $\ketp 3$, 
será necesario el tamaño del más grande, siendo este $S(\ketp 3)$. El tamaño 
total $S_T$ será
$$ S_T = S(H) + S(U) + S(\ketp 3)$$
Que de forma aproximada $S_T'$, se determina como
%
\begin{equation*}
\begin{split}
&\log_2 S(H) \approx N + \log_2 8 = N+3 \\
&\log_2 S(U) \approx N + \log_2 12 \approx N+3.58\\
&\log_2 S(\ketp 3) \approx N + \log_2 6 \approx N+2.58 \\
&\log_2 S_T \approx \log_2 S_T' = N+\log_2 (8 + 12 + 6) \approx N+4.70 \\
&S_T' = 2^N \log_226
\end{split}
\end{equation*}
%
Mostrando de nuevo el carácter exponencial del espacio requerido, en torno a 
$O(2^N)$. Dado que el espacio empleado para $QC_f$ es superior a $QC_0$, se 
usará $S_T' = 2^N \log_226$ como aproximación a la memoria requerida por la 
simulación de todo el circuito.

\subsubsection{Limitaciones de la simulación}

El espacio de simulación tiene una memoria finita, fijada en 512MB, un total de 
$2^{29}$ bytes. En la figura~\ref{fig:espacio-qc} se observa como crece el 
espacio necesario a medida que aumentan los qubits.
%
\begin{figure}[!htb]
\centering
\begin{tikzpicture}
\begin{axis}[
	name=qc0,
	width=0.9\linewidth,
	height=8cm,
%	no marks,
	mark options={mark size=1},
	thick,
	grid=both,
	xtick={4,...,20},
	xmin=3.5,xmax=20.5,
	legend style={at={(1.0,0.0)},anchor=south east},
	xlabel={Número de qubits $N$},
	ylabel={Espacio ocupado $\log_2 S$},
]
\addplot [red] coordinates {(0,29) (20.5,29)};
\addplot [blue,mark=*] table [x=N, y=log2_all_qcf, col sep=comma] 
{csv/table_qcf.csv};
\addplot [black,mark=*] table [x=N, y=log2_all_qc0, col sep=comma] 
{csv/table_qc0.csv};
\addplot [blue, dashed] table [x=N, y=log2_approx_qcf, col sep=comma] 
{csv/table_qcf.csv};
\addplot [black, dashed] table [x=N, y=log2_approx_qc0, col sep=comma] 
{csv/table_qc0.csv};
\addplot [purple, dashed, domain=4:6.25]{(4*x+4)};
\legend{
	$S_{MAX}$,
	$S_T (QC_0)$,
	$S_T (QC_f)$,
	$S_T' (QC_0)$,
	$S_T' (QC_f)$,
	$S_D$};
\end{axis}
\end{tikzpicture}
\caption{Espacio necesario para la simulación en bytes (escala logarítmica). Se
muestra en línea continua el espacio real, y en discontinua el aproximado. El 
espacio requerido sin emplear matrices huecas, usando matrices densas se muestra 
como $S_D$. La memoria disponible para la simulación es $S_{MAX} = 2^{29}$.}
\label{fig:espacio-qc}
\end{figure}
%
Dada la complejidad de carácter exponencial, la cantidad de qubits simulados por 
el circuito se ve severamente limitada por $S_T \leq 2^{29}$. Que de forma 
aproximada, se obtiene
\begin{equation*}
\begin{split}
&S_T' = 2^N \log_226 \leq 2^{29} \\
&N + \log_2 26 \leq 29 \\
&N \leq 29 - \log_2 26 \approx 24.30 \\
&N \leq 24 \\
\end{split}
\end{equation*}
Por lo tanto, sólo es posible simular cicuitos de hasta un máximo de 24 qubits, 
en el entorno de simulación actual. Dado que el problema requiere dos líneas de 
$n$ bits, son necesarios $N = 2n$ qubits. Obteniendo la limitación $n \leq 12$.


\subsection{Medición y computación clásica}
% El tamaño necesario para el resto de etapas es insignificante
Una vez obtenido el estado $\ketp 3$, tras la costosa simulación, los demás 
componentes del simulador apenas requieren memoria.
El proceso de medición, denominado $M$, toma el vector $\ketp 3$ y calcula la 
distribución de probabilidad resultante de medir la línea superior. Esta 
distribución tiene $2^n$ posibles resultados, que se almacenan en un vector de 
probabilidades $v_p$ junto con otro vector del resultado correspondiente $v_n$.  
Con un tamaño
$$ S(v_p) = S(v_n) = 2^{n} \cdot 4 = 2^{N/2} \cdot 4 $$
Obteniéndose un espacio requerido para la medición
$$ S_T(M) = S(v_p) + S(v_n) = 2^{N/2} \cdot 8 $$
Debido a que tras el proceso de simulación, al finalizar la etapa $QC_f$, ya no 
es necesario mantener los operadores $U_f$ y $H$ almacenados, se libera un 
espacio
$$ S(U_f) + S(H) = 2^N \log_2 20 \approx 2^N \cdot 4.32$$
Dicho espacio es más que suficiente para los vectores $v_p$ y $v_n$ con $N \geq 
2$.

El proceso de cómputo clásico posterior, denotado $CC$, requiere que $M$ 
mantenga las distribuciones de probabilidad en memoria. Además, calcula para 
cada medición, el conjunto de combinaciones lineales de vectores acumulados.  
Como se requieren $n-1$ vectores, será necesario un espacio para los vectores 
acumulados $v_a$ dado por
$$ S_T(CC) = S(v_a) = (2^{n-1} - 1) \cdot 4 = 2^{N/2} \cdot 2 - 4$$
Conjuntamente con el requerido por $M$
$$ S_T(M) + S_T(CC) = 2^{N/2} \cdot 10 - 4 $$
Que continúa siendo inferior al liberado por los operadores tras terminar $QC_f$ 
para $N \geq 2$. Por tanto, la etapa final de la simulación cuántica $QC_f$ es 
el cuello de botella que limita el espacio que empleará toda la simulación.

%Partiendo de que sólo es posible simular circuitos de hasta $n = 12$.

\section{Complejidad del circuito cuántico}
Tras analizar la eficiencia midiendo el tiempo y la memoria de la simulación, el 
último paso consiste en examinar la complejidad del circuito cuántico. Esta 
complejidad debe ser la observada en un ordenador cuántico real, y debe 
coincidir también con los cálculos teóricos.

El escenario de análisis consistirá en una serie de simulaciones completas, 
observando el número de ejecuciones necesarias del circuito cuántico para 
completar el problema.

Para evitar la costosa simulación sucesiva de la parte cuántica, se empleará el 
diseño optimizado que permite reutilizar los cálculos, almacenando el estado 
final. De esta forma se consigue que la simulación cuántica sólo sea necesaria 
realizarla una única vez. El esquema de la simulación se presenta a 
continuación.

\begin{center}
% Define block styles
\tikzstyle{decision} = [diamond, draw, text width=4em, text badly centered, 
inner sep=0pt]
\tikzstyle{block} = [rectangle, draw, text width=2em, text centered, minimum 
height=1cm]
\tikzstyle{line} = [draw, decoration={markings,mark=at position 
1 with {\arrow[scale=1.5]{latex'}}}, postaction={decorate}]
%
\begin{tikzpicture}[node distance = 2.5cm, auto, thick]
	% Place nodes
	\node[block] (quantum) {$QC$};
	\node[block, right of=quantum] (measure) {$M$};
	\node[block, right of=measure] (classic) {$CC$};
	\node[decision, right of=classic] (end) {Fin?};
	% Draw edges
	\draw [line] (quantum.west)+(-1cm,0) -- (quantum.west);
	\draw [line] (quantum) -> (measure);
	\draw [line] (measure) -- (classic);
	\draw [line] (classic) -- (end);
	\draw [line] (end.south) |-+(0,-1em)-| node [near start, above] {No} 
(measure.south);
	\draw [line] (end.east) -- node [near start] {Sí} +(1cm, 0);

\end{tikzpicture}
\end{center}

Tras repetir la simulación una cantidad $r = 10000$ veces, se analiza la media y 
varianza, y se comparan con las calculadas previamente de forma teórica. En la 
tabla~\ref{tab:comp-qc} se puede observar como ambos resultados se acercan 
mucho. En la figura~\ref{fig:comp-qc} se observa como el crecimiento es lineal, 
de modo que la complejidad se aproxima a $O(n)$, tal y como se había calculado 
teóricamente.

%TODO Mencionar la ecuación de ejecuciones medias E[R].

\begin{table}[!htb]
\centering
\pgfplotstabletypeset[
	col sep=comma,
	columns={n, N, R-mean, R-var, RT-mean, RT-var},
	columns/n/.style={column name=$n$, column type={r}},
	columns/N/.style={column name=$N$, column type={r}},
	%columns/r/.style={column name=$r$, column type={r}},
	columns/R-mean/.style={
		column name=$R_\mu$,
		fixed zerofill,precision=3,dec sep align
	},
	columns/R-var/.style={
		column name=$R_{\sigma^2}$,
		fixed zerofill,precision=3,dec sep align
	},
	columns/RT-mean/.style={
		column name=$R_\mu'$,
		fixed zerofill,precision=3,dec sep align
	},
	columns/RT-var/.style={
		column name=$R_{\sigma^2}'$,
		fixed zerofill,precision=3,dec sep align
	},
	every head row/.style={before row=\toprule,after row=\midrule},
	every last row/.style={after row=\bottomrule},
]{csv/com_qc.csv}
\caption{Ejecuciones del algoritmo cuántico $R$ experimentales, $R'$ teóricas.}
\label{tab:comp-qc}
\end{table}

%
\begin{figure}[!htb]
\centering
\begin{tikzpicture}
\begin{axis}[
	name=qc0,
	width=0.9\linewidth,
	height=8cm,
%	no marks,
	mark options={mark size=1},
	thick,
	grid=both,
	xtick={4,...,20},
	ytick={2,3,...,20},
	xmin=3.5,xmax=20.5,
	legend style={at={(1.0,0.0)},anchor=south east},
	xlabel={Número de qubits $N$},
	ylabel={Número esperado de ejecuciones R},
]
\addplot [red,mark=*,thick] table [x=N, y=R-mean, col sep=comma] 
{csv/com_qc.csv};
\addplot [black, dashed, mark=none, very thick] table [x=N, y=RT-mean, col 
sep=comma] {csv/com_qc.csv};
\legend{
	$R_\mu$,
	$R_\mu'$};
\end{axis}
\end{tikzpicture}
\caption{Número esperado de iteraciones a medida que aumentan los qubits. Se 
observa el valor experimental $R_\mu$ comparado con el teórico $R_\mu'$.}
\label{fig:comp-qc}
\end{figure}
%

%\cite{nielsen00}

%\bibliography{bibliography}{}
%\bibliographystyle{plain}


\chapter{Conclusiones y trabajo futuro}

En los capítulos anteriores se ha presentado el diseño y la implementación de un 
simulador capaz de llevar a cabo la ejecución de un algoritmo cuántico.  La 
complejidad del problema es analizada primero de forma teórica, calculando el 
número esperado de ejecuciones $E[R]$. Posteriormente se analiza la complejidad 
del algoritmo de forma experimental a través de la simulación.
Ambos resultados indican que se trata de un algoritmo con una complejidad de 
$O(n)$, confirmando que un algoritmo cuántico puede resolver un problema mucho 
más eficientemente que uno clásico, que requeriría un tiempo de $O(2^{n-1})$. De 
esta forma se satisface el objetivo principal de determinar la complejidad de un 
algoritmo cuántico.

También se aprecia la dificultad que presenta realizar la simulación de un 
sistema cuántico dado que los recursos tanto de espacio como de tiempo crecen de 
forma exponencial con el número de qubits. Richard Feynman realizó la 
observación de que un sistema clásico no podría simular eficientemente uno 
cuántico, y este simulador presenta un ejemplo en el que se aprecia dicha 
dificultad. Tan sólo es posible realizar una simulación hasta los 20 qubits, 
limitando el uso de algoritmos más complejos. Una posible mejora de cara al 
futuro sería la investigación de técnicas para reducir los recursos necesarios 
por los circuitos y permitir simulaciones con más qubits.

Por otra parte, para ayudar en la tarea de determinar la complejidad de los 
algoritmos, el simulador podría incluír técnicas automatizadas de análisis 
mediante ejecuciones reiteradas, con el objetivo de presentar los resultados de 
forma gráfica. Por ahora estas técnicas se realizan de forma manual en la 
implementación de los algoritmos cuánticos

El código fuente del proyecto se encuentra en GitHub \cite{pyqsim}, y permite 
trabajar de forma colaborativa en la construcción de mejoras y nuevas 
características.
