\documentclass[12pt,a4paper]{article}
\usepackage{amsfonts}
\usepackage{amsmath}
\usepackage{amsthm}
\usepackage[utf8]{inputenc}
\usepackage{braket}

% Romper líneas por sílabas en español
\usepackage[spanish,es-nosectiondot]{babel}

% Dimensiones de los márgenes.
%\usepackage[margin=2.5cm]{geometry}

% Vector
\newcommand*\mat[1]{ \begin{pmatrix} #1 \end{pmatrix}}
\newcommand*\arr[1]{ \begin{bmatrix} #1 \end{bmatrix}}

\newcommand*\V[1]{ \boldsymbol{#1}}

% Apartado de un ejemplo
\theoremstyle{definition}
\newtheorem{ejemplo}{Ejemplo}[section]

% Gráficos con tikz
\usepackage{tikz}

\usetikzlibrary{calc,fadings,decorations.pathreplacing}
\usetikzlibrary{backgrounds,fit}
\newcommand\pgfmathsinandcos[3]{%
  \pgfmathsetmacro#1{sin(#3)}%
  \pgfmathsetmacro#2{cos(#3)}%
}
\newcommand\LongitudePlane[3][current plane]{%
  \pgfmathsinandcos\sinEl\cosEl{#2} % elevation
  \pgfmathsinandcos\sint\cost{#3} % azimuth
  \tikzset{#1/.estyle={cm={\cost,\sint*\sinEl,0,\cosEl,(0,0)}}}
}
\newcommand\LatitudePlane[3][current plane]{%
  \pgfmathsinandcos\sinEl\cosEl{#2} % elevation
  \pgfmathsinandcos\sint\cost{#3} % latitude
  \pgfmathsetmacro\yshift{\cosEl*\sint}
  \tikzset{#1/.estyle={cm={\cost,0,0,\cost*\sinEl,(0,\yshift)}}} %
}
\newcommand\DrawLongitudeCircle[2][1]{
  \LongitudePlane{\angEl}{#2}
%  \tikzset{current plane/.prefix style={scale=#1}}
   % angle of "visibility"
  \pgfmathsetmacro\angVis{atan(sin(#2)*cos(\angEl)/sin(\angEl))} %
  \draw[current plane] (\angVis:1) arc (\angVis:\angVis+180:1);
  \draw[current plane,dashed] (\angVis-180:1) arc (\angVis-180:\angVis:1);
}
\newcommand\DrawLatitudeCircle[2][1]{
  \LatitudePlane{\angEl}{#2}
%	\tikzset{current plane/.prefix style={scale=#1}}
  \pgfmathsetmacro\sinVis{sin(#2)/cos(#2)*sin(\angEl)/cos(\angEl)}
  % angle of "visibility"
  \pgfmathsetmacro\angVis{asin(min(1,max(\sinVis,-1)))}
  \draw[current plane] (\angVis:1) arc (\angVis:-\angVis-180:1);
  \draw[current plane,dashed] (180-\angVis:1) arc (180-\angVis:\angVis:1);
}


\setcounter{section}{4}
\usetikzlibrary{shapes,arrows,chains}
\usetikzlibrary{decorations.markings}

\begin{document}

\section{Implementación}
La simulación se realizará en el lenguaje de programación \texttt{python}, 
empleando paquetes externos que aportarán las partes comunes del simulador. De 
esta forma se realiza una reutilización del software ya existente.

Para los cálculos numéricos se emplea \texttt{numpy} y \texttt{scipy}.

\subsection{Estructuras de datos}
El paquete \texttt{qutip} permite representar los estados cuánticos, así como 
los operadores. Esta representación, se realiza mediante matrices huecas. La 
clase \texttt{qutip.Qobj} permite emplear de forma implícita las matrices huecas 
de \texttt{scipy.sparse}.

De esta forma, un estado $\ket{\phi_0} = \ket{000} $


\section{Análisis de la simulación}
% Comentar que es lo que se va a medir. Como se realiza el proceso de medición, 
% y finalmente cuales son los resultados.

El proceso de la simulación se analiza exhaustivamente para determinar por una 
parte la \textit{eficiencia} del simulador, midiendo el tiempo empleado y la 
memoria. Además también se observa el número de ejecuciones del algoritmo 
cuántico, para calcular su \textit{complejidad}.

%
\begin{center}
% Define block styles
\tikzstyle{decision} = [diamond, draw, text width=4.5em, text badly centered, 
inner sep=0pt]
\tikzstyle{block} = [rectangle, draw, text width=5em, text centered, minimum 
height=2cm]
\tikzstyle{line} = [draw, thick, decoration={markings,mark=at position 
1 with {\arrow[scale=1.5]{latex'}}}, postaction={decorate}]
%
\begin{tikzpicture}[node distance = 3cm, auto]
	% Place nodes
	\node[block] (qc0) {Simulación cuántica inicial};
	\node[block, right of=qc0] (qcf) {Finalizar simulación cuántica};
	\node[block, right of=qcf] (measure) {Medición};
	\node[block, right of=measure] (classic) {Procesado clásico};
	% Draw edges
	\draw [line] (qc0.west)+(-1cm,0) -- (qc0.west);
	\draw [line] (qc0) -> (qcf);
	\draw [line] (qcf) -- (measure);
	\draw [line] (measure) -- (classic);
	\draw [line] (qc0.-30) -| +(+1em,-1cm) -| node [near start, above] {$M_1$} 
(qc0)+(-2em,-1em);
	\draw [line] (qcf.-30) -| +(+1em,-1cm) -| node [near start, above] {$M_2$} 
(qcf)+(-2em,-1em);
	\draw [line] (measure.-30) -| +(+1em,-1cm) -| node [near start, above] {$M_2$} 
(measure)+(-2em,-1em);
	\draw [line] (classic.-30) -| +(+1em,-1cm) -| node [near start, above] {$M_2$} 
(classic)+(-2em,-1em);
	\draw [line] (classic.east) -- +(+1cm,0);

\end{tikzpicture}
\end{center}
%

\subsection{Detalles del análisis}
Para calcular con precisión los datos medidos sobre cada simulación, se realizan 
varias ejecuciones, y posteriormente se analiza la media y varianza de las 
medidas obtenidas.

Cada etapa se mide de forma independiente, permitiendo una mayor precisión en la 
medida de la complejidad.

El análisis de la simulación cuántica, se divide en dos procesos. La parte 
inicial calcula el estado intermedio $\ketp 1$ que es independiente de la 
función $f$ del problema. Permitiendo la reutilización de los cálculo previos.

La parte final del análisis, toma el estado $\ketp 1$ previamente calculado, y 
continúa la simulación del circuito hasta el estado final.

De esta forma, al determinar cómo se comporta el circuito con diferentes 
funciones $f$, la parte inicial no requiere ser computada en cada simulación.


\section{Análisis de espacio}
Para observar el comportamiento de un algoritmo es importante tener en cuenta 
además del tiempo que requiere su ejecución, el espacio que emplea. El simulador 
será analizado paso a paso mostrando como varía la memoria empleada a medida que 
crece el tamaño de la entrada.

\subsection{Simulación cuántica}
La simulación del circuito cuántico se divide en dos etapas, la parte inicial, 
hasta el estado $\ketp 1$ denominada $QC_0$, y la parte final, hasta el estado 
$\ketp 3$, denominada $QC_f$. El esquema del circuito cuántico muestra el orden 
de los operadores, y la posición de los estados.
%
%
\begin{center}
	\begin{tikzpicture}%[thick]
	% `operator' will only be used by Hadamard (H) gates here.
	\tikzstyle{operator} = [draw,fill=white,minimum size=1.5em] 
	%
	\matrix[row sep=0.4cm, column sep=1cm] (circuit) {
		% First row
		\node (q1) {$\ket{0^n}$}; &
		\node[operator] (H11) {$H^{\otimes n}$}; &
		\node[operator] (P13) {}; &
		\node[operator] (H11) {$H^{\otimes n}$}; &
		\node[operator] (M11) {$M$};&
		\coordinate (end1); \\
		% Second row.
		\node (q2) {$\ket{0^n}$}; & & \node[operator] (P23) {}; & & & \coordinate (end2);\\
		% Third row
		\node (q31) {}; & \node (q32) {}; & \node (q33) {}; &
		\node (q34) {}; & \node (q35) {}; & \node (q36) {}; \\
		\node (q41) {}; & \node (q42) {}; & \node (q43) {}; &
		\node (q44) {}; & \node (q45) {}; & \node (q46) {}; \\
	};
	\node[operator] (Uf) [fit = (P13) (P23), minimum width=1cm] {$U_f$};

	\node (arr0) [fit = (q31) (q32)] {$\uparrow$};
	\node (arr1) [fit = (q32) (q33)] {$\uparrow$};
	\node (arr2) [fit = (q33) (q34)] {$\uparrow$};
	\node (arr3) [fit = (q34) (q35)] {$\uparrow$};

	\node (psi0) [fit = (q41) (q42)] {$\ket{\psi_0}$};
	\node (psi1) [fit = (q42) (q43)] {$\ket{\psi_1}$};
	\node (psi2) [fit = (q43) (q44)] {$\ket{\psi_2}$};
	\node (psi3) [fit = (q44) (q45)] {$\ket{\psi_3}$};

	\node[fill=white, fit=(end1) (end2)] (cover) {};

	\begin{pgfonlayer}{background}
		% Draw lines.
		\draw[thick] (q1) -- (M11)  (q2) -- (end2);
		\draw[thick, double, double distance=2pt] (M11) -- (end1);
	\end{pgfonlayer}
	%
\end{tikzpicture}
\end{center}

%
Esquema de la simulación cuántica
%
\begin{center}
% Define block styles
\tikzstyle{decision} = [diamond, draw, text width=4.5em, text badly centered, 
inner sep=0pt]
\tikzstyle{block} = [rectangle, draw, text width=2em, text centered, minimum 
height=1cm]
\tikzstyle{line} = [draw, thick, decoration={markings,mark=at position 
1 with {\arrow[scale=1.5]{latex'}}}, postaction={decorate}]
%
\begin{tikzpicture}[node distance = 2.5cm, auto]
	% Place nodes
	\node[block] (qc0) {$QC_0$};
	\node[block, right of=qc0] (qcf) {$QC_f$};
	\node[block, right of=qcf] (measure) {$M$};
	\node[block, right of=measure] (classic) {$CC$};
	% Draw edges
	\draw [line] (qc0.west)+(-1cm,0) -- (qc0.west);
	\draw [line] (qc0) -> (qcf);
	\draw [line] (qcf) -- (measure);
	\draw [line] (measure) -- (classic);
	\draw [line] (classic.east) -- +(+1cm,0);
\end{tikzpicture}
\end{center}
%
\subsubsection{Análisis de la primera parte $QC_0$}
En $QC_0$, será necesario almacenar en la memoria el operador de Hadamard 
$H^{\otimes n}$, además de un estado cuántico $\ket{\psi}$ de $N = 2n$ qubits.  
Al analizar el comportamiento del circuito, se obtiene la tabla~\ref{tab:qc0}.
%
\begin{table}[!htb]
\centering
\begin{tabular}{rrrrrr}
\toprule
   $n$ &   $N$ &   $\log_2 S(H)$ &   $\log_2 S(\ket{\phi_1})$ &   $\log_2 S_T$ &   $\log_2 S_T'$ \\
\midrule
     2 &     4 &            7.21 &                       6.64 &           7.95 &            7.58 \\
     3 &     6 &            9.10 &                       8.34 &           9.77 &            9.58 \\
     4 &     8 &           11.05 &                      10.17 &          11.68 &           11.58 \\
     5 &    10 &           13.02 &                      12.09 &          13.63 &           13.58 \\
     6 &    12 &           15.01 &                      14.04 &          15.61 &           15.58 \\
     7 &    14 &           17.01 &                      16.02 &          17.60 &           17.58 \\
     8 &    16 &           19.00 &                      18.01 &          19.59 &           19.58 \\
     9 &    18 &           21.00 &                      20.01 &          21.59 &           21.58 \\
    10 &    20 &           23.00 &                      22.00 &          23.59 &           23.58 \\
\bottomrule
\end{tabular}
\caption{Espacio empleado por $QC_0$ en escala logarítmica (bytes).}
\label{tab:qc0}
\end{table}
%
Sea $S(x)$ el tamaño del objeto $x$ en bytes, y $S_T$ el tamaño total requerido 
por la simulación. Entonces el espacio $S_T'$ para $QC_0$ será una aproximación 
al espacio real $S_T$ determinado como
\begin{equation*}
\begin{split}
	&\log_2 S(H) \approx N + 3 \\
	&\log_2 S(\ketp 1)  \approx N + 2 \\
	&\log_2 S_T \approx N+\log_2 (2^2+2^3) = N+3.58 = \log_2 S_T' \\
	& S_T' = 2^{N} \log_2 12
\end{split}
\end{equation*}
Se observa como el espacio requerido aumenta de forma exponencial a medida que 
aumenta el número de qubits del sistema y se encuentra en torno a $O(2^n)$.



\subsubsection{Análisis de la parte final $QC_f$}
En la etapa final de la simulación del circuito, se sobreescribe $\ketp 1$ con 
el estado final $\ketp 3$, de modo que sólo será necesario un espacio 
equivalente al del más grande.
Además del estado, se necesita el operador $U_f$, que se calcula a partir de la 
función dada $f$. Los tamaños se muestran en la tabla \ref{tab:qcf}.
%
\begin{table}[!htb]
\centering
\begin{tabular}{rrrrrrr}
\toprule
   $n$ &   $N$ &   $\log_2 S(H)$ &   $\log_2 S(U)$ &   $\log_2 S(\ket{\phi_3})$ &   $\log_2 S_T$ &   $\log_2 S_T'$ \\
\midrule
     2 &     4 &            7.21 &            7.61 &                       6.64 &           8.79 &            8.70 \\
     3 &     6 &            9.10 &            9.59 &                       8.60 &          10.74 &           10.70 \\
     4 &     8 &           11.05 &           11.59 &                      10.59 &          12.72 &           12.70 \\
     5 &    10 &           13.02 &           13.59 &                      12.59 &          14.71 &           14.70 \\
     6 &    12 &           15.01 &           15.59 &                      14.59 &          16.70 &           16.70 \\
     7 &    14 &           17.01 &           17.58 &                      16.59 &          18.70 &           18.70 \\
     8 &    16 &           19.00 &           19.58 &                      18.58 &          20.70 &           20.70 \\
     9 &    18 &           21.00 &           21.58 &                      20.58 &          22.70 &           22.70 \\
    10 &    20 &           23.00 &           23.58 &                      22.58 &          24.70 &           24.70 \\
\bottomrule
\end{tabular}
\caption{Espacio empleado por $QC_f$ en escala logarítmica (bytes).}
\label{tab:qcf}
\end{table}

El tamaño requerido en esta etapa de la simulación será $S_T$, aproximado a 
$S_T'$. Además el operador $H$ se reutiliza de la etapa inicial, de modo que ese 
espacio debe tenerse en cuenta. El tamaño del estado $\ketp 3$ es siempre un 
poco más grande que $\ketp 1$. Dado que $\ketp 1$ se reemplazará por $\ketp 3$, 
será necesario el tamaño del más grande, siendo este $S(\ketp 3)$. El tamaño 
total $S_T$ será
$$ S_T = S(H) + S(U) + S(\ketp 3)$$
Que de forma aproximada $S_T'$, se determina como
%
\begin{equation*}
\begin{split}
&\log_2 S(H) \approx N + \log_2 8 = N+3 \\
&\log_2 S(U) \approx N + \log_2 12 \approx N+3.58\\
&\log_2 S(\ketp 3) \approx N + \log_2 6 \approx N+2.58 \\
&\log_2 S_T \approx \log_2 S_T' = N+\log_2 (8 + 12 + 6) \approx N+4.70 \\
&S_T' = 2^N \log_226
\end{split}
\end{equation*}

\subsubsection{Limitaciones de la simulación}

El espacio de simulación tiene una memoria finita, fijada en 512MB, un total de 
$2^{29}$ bytes. En la figura~\ref{fig:espacio-qc} se observa como crece el 
espacio necesario a medida que aumentan los qubits.

\begin{figure}[!htb]
\centering
\begin{tikzpicture}
\begin{axis}[
	name=qc0,
	width=0.9\linewidth,
	height=8cm,
	no marks,
	thick,
	grid=both,
	xtick={4,...,20},
	xmin=3.5,xmax=20.5,
	legend style={at={(1.0,0.0)},anchor=south east},
	xlabel={Número de qubits $N$},
	ylabel={Espacio ocupado $\log_2 S$},
]
\addplot [red] coordinates {(0,29) (20.5,29)};
\addplot [blue] table [x=N, y=log2_all_qcf, col sep=comma] {csv/table_qcf.csv};
\addplot [black] table [x=N, y=log2_all_qc0, col sep=comma] {csv/table_qc0.csv};
\addplot [blue, dashed] table [x=N, y=log2_approx_qcf, col sep=comma] 
{csv/table_qcf.csv};
\addplot [black, dashed] table [x=N, y=log2_approx_qc0, col sep=comma] 
{csv/table_qc0.csv};
\addplot[purple, dashed, domain=4:6.25]{(4*x+4)};
\legend{
	$S_{MAX}$,
	$S_T (QC_0)$,
	$S_T (QC_f)$,
	$S_T' (QC_0)$,
	$S_T' (QC_f)$,
	$S_D$};
\end{axis}
\end{tikzpicture}
\caption{Espacio necesario para la simulación. Se muestra en línea continua el 
espacio real, y en discontinua el aproximado. El espacio requerido sin emplear 
matrices huecas, usando matrices densas se muestra como $S_D$.}
\label{fig:espacio-qc}
\end{figure}
%
Dada la complejidad de carácter exponencial, la cantidad de qubits simulados por 
el circuito se ve severamente limitada por $S_T \leq 2^{29}$. Que de forma 
aproximada, se obtiene
\begin{equation*}
\begin{split}
&S_T' = 2^N \log_226 \leq 2^{29} \\
&N + \log_2 26 \leq 29 \\
&N \leq 29 - \log_2 26 \approx 24.30 \\
&N \leq 24 \\
\end{split}
\end{equation*}
Por lo tanto, sólo es posible simular cicuitos de hasta un máximo de 24 qubits, 
en el entorno de simulación actual. Dado que el problema requiere dos líneas de 
$n$ bits, son necesarios $N = 2n$ qubits. Obteniendo la limitación $n \leq 12$.


\subsection{Medición y computación clásica}
% El tamaño necesario para el resto de etapas es insignificante
Una vez obtenido el estado $\ketp 3$, tras la costosa simulación, el resto de 
componentes del simulador apenas requieren espacio. Partiendo de que sólo es 
posible simular circuitos de hasta $n = 12$.

\end{document}
