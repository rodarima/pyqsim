\documentclass[12pt,a4paper]{article}
\usepackage{amsfonts}
\usepackage{amsmath}
\usepackage{amsthm}
\usepackage[utf8]{inputenc}
\usepackage{braket}

% Romper líneas por sílabas en español
\usepackage[spanish,es-nosectiondot]{babel}

% Dimensiones de los márgenes.
%\usepackage[margin=2.5cm]{geometry}

% Vector
\newcommand*\mat[1]{ \begin{pmatrix} #1 \end{pmatrix}}
\newcommand*\arr[1]{ \begin{bmatrix} #1 \end{bmatrix}}

\newcommand*\V[1]{ \boldsymbol{#1}}

% Apartado de un ejemplo
\theoremstyle{definition}
\newtheorem{ejemplo}{Ejemplo}[section]

% Gráficos con tikz
\usepackage{tikz}

\usetikzlibrary{calc,fadings,decorations.pathreplacing}
\usetikzlibrary{backgrounds,fit}
\newcommand\pgfmathsinandcos[3]{%
  \pgfmathsetmacro#1{sin(#3)}%
  \pgfmathsetmacro#2{cos(#3)}%
}
\newcommand\LongitudePlane[3][current plane]{%
  \pgfmathsinandcos\sinEl\cosEl{#2} % elevation
  \pgfmathsinandcos\sint\cost{#3} % azimuth
  \tikzset{#1/.estyle={cm={\cost,\sint*\sinEl,0,\cosEl,(0,0)}}}
}
\newcommand\LatitudePlane[3][current plane]{%
  \pgfmathsinandcos\sinEl\cosEl{#2} % elevation
  \pgfmathsinandcos\sint\cost{#3} % latitude
  \pgfmathsetmacro\yshift{\cosEl*\sint}
  \tikzset{#1/.estyle={cm={\cost,0,0,\cost*\sinEl,(0,\yshift)}}} %
}
\newcommand\DrawLongitudeCircle[2][1]{
  \LongitudePlane{\angEl}{#2}
%  \tikzset{current plane/.prefix style={scale=#1}}
   % angle of "visibility"
  \pgfmathsetmacro\angVis{atan(sin(#2)*cos(\angEl)/sin(\angEl))} %
  \draw[current plane] (\angVis:1) arc (\angVis:\angVis+180:1);
  \draw[current plane,dashed] (\angVis-180:1) arc (\angVis-180:\angVis:1);
}
\newcommand\DrawLatitudeCircle[2][1]{
  \LatitudePlane{\angEl}{#2}
%	\tikzset{current plane/.prefix style={scale=#1}}
  \pgfmathsetmacro\sinVis{sin(#2)/cos(#2)*sin(\angEl)/cos(\angEl)}
  % angle of "visibility"
  \pgfmathsetmacro\angVis{asin(min(1,max(\sinVis,-1)))}
  \draw[current plane] (\angVis:1) arc (\angVis:-\angVis-180:1);
  \draw[current plane,dashed] (180-\angVis:1) arc (180-\angVis:\angVis:1);
}


\setcounter{section}{4}
\usetikzlibrary{shapes,arrows,chains}
\usetikzlibrary{decorations.markings}

\begin{document}

\section{Implementación}
La simulación se realizará en el lenguaje de programación \texttt{python}, 
empleando paquetes externos que aportarán las partes comunes del simulador. De 
esta forma se realiza una reutilización del software ya existente.

Para los cálculos numéricos se emplea \texttt{numpy} y \texttt{scipy}.

\subsection{Estructuras de datos}
El paquete \texttt{qutip} permite representar los estados cuánticos, así como 
los operadores. Esta representación, se realiza mediante matrices huecas. La 
clase \texttt{qutip.Qobj} permite emplear de forma implícita las matrices huecas 
de \texttt{scipy.sparse}.

De esta forma, un estado $\ket{\phi_0} = \ket{000} $


\section{Análisis de la simulación}
% Comentar que es lo que se va a medir. Como se realiza el proceso de medición, 
% y finalmente cuales son los resultados.

El proceso de la simulación se analiza exhaustivamente para determinar por una 
parte la \textit{eficiencia} del simulador, midiendo el tiempo empleado y la 
memoria. Además también se observa el número de ejecuciones del algoritmo 
cuántico, para calcular su \textit{complejidad}.

%
\begin{center}
% Define block styles
\tikzstyle{decision} = [diamond, draw, text width=4.5em, text badly centered, 
inner sep=0pt]
\tikzstyle{block} = [rectangle, draw, text width=5em, text centered, minimum 
height=2cm]
\tikzstyle{line} = [draw, thick, decoration={markings,mark=at position 
1 with {\arrow[scale=1.5]{latex'}}}, postaction={decorate}]
%
\begin{tikzpicture}[node distance = 3cm, auto]
	% Place nodes
	\node[block] (qc0) {Simulación cuántica inicial};
	\node[block, right of=qc0] (qcf) {Finalizar simulación cuántica};
	\node[block, right of=qcf] (measure) {Medición};
	\node[block, right of=measure] (classic) {Procesado clásico};
	% Draw edges
	\draw [line] (qc0.west)+(-1cm,0) -- (qc0.west);
	\draw [line] (qc0) -> (qcf);
	\draw [line] (qcf) -- (measure);
	\draw [line] (measure) -- (classic);
	\draw [line] (qc0.-30) -| +(+1em,-1cm) -| node [near start, above] {$M_1$} 
(qc0)+(-2em,-1em);
	\draw [line] (qcf.-30) -| +(+1em,-1cm) -| node [near start, above] {$M_2$} 
(qcf)+(-2em,-1em);
	\draw [line] (measure.-30) -| +(+1em,-1cm) -| node [near start, above] {$M_2$} 
(measure)+(-2em,-1em);
	\draw [line] (classic.-30) -| +(+1em,-1cm) -| node [near start, above] {$M_2$} 
(classic)+(-2em,-1em);
	\draw [line] (classic.east) -- +(+1cm,0);

\end{tikzpicture}
\end{center}
%

\subsection{Detalles del análisis}
Para calcular con precisión los datos medidos sobre cada simulación, se realizan 
varias ejecuciones, y posteriormente se analiza la media y varianza de las 
medidas obtenidas.

Cada etapa se mide de forma independiente, permitiendo una mayor precisión en la 
medida de la complejidad.

El análisis de la simulación cuántica, se divide en dos procesos. La parte 
inicial calcula el estado intermedio $\ket{psi_1}$ que es independiente de la 
función $f$ del problema. Permitiendo la reutilización de los cálculo previos.

La parte final del análisis, toma el estado $\ket{psi_1}$ previamente calculado, 
y continúa la simulación del circuito hasta el estado final.

De esta forma, al determinar cómo se comporta el circuito con diferentes 
funciones $f$, la parte inicial no requiere ser computada en cada simulación.

El número de ejecuciones es independiente

\end{document}
