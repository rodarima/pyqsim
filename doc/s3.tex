\chapter{Algoritmo de Simon}
\section{Algoritmo de Simon}
\subsection{Descripción del problema}

Daniel R. Simon encontró un problema que presentaba una complejidad exponencial 
al tratar de resolverse mediante métodos de computación tradicionales. Sin 
embargo, al abordarlo mediante una solución cuántica, dicha complejidad 
disminuía de orden, a una lineal.

El problema consiste en encontrar una cadena binaria desconocida, denominada 
$s$, que contiene $n$ bits, relacionada con una función $f$. Sea $V$ el conjunto 
de cadenas de $n$ bits, $V = \{0,1\}^n$, entonces la función $f$ asigna a cada 
cadena de $V$ otra cadena del mismo conjunto, $f:V \rightarrow V$.

Además la función cumple la restricción
%
\begin{equation}
	\label{eq:simon_f}
	f(x_1) = f(x_2) \iff x_1 = x_2 \oplus s
\end{equation}
%
Por una parte, si el resultado de dos salidas es idéntico, es decir $f(x_1) = 
f(x_2)$ entonces entradas deben cumplir que $x_1 = x_2 \oplus s$:
%
$$ f(x_1) = f(x_2) \implies x_1 = x_2 \oplus s $$
%
Y por otra parte, si dos pares de entradas cumplen $x_1 = x_2 \oplus s$ entonces 
sus salidas serán idénticas:
%
$$ x_1 = x_2 \oplus s \implies f(x_1) = f(x_2) $$
%
Con un ejemplo de 2 bits, y una cadena $s = 01$, se define $f$:
%
\begin{center}
\begin{tabular}{|c|c||c|c|}
	\hline
	$x$ & $f(x)$ & $x \oplus s$ & $f(x \oplus s)$\\
	\hline
	00 & 00 & 01 & 00\\
	01 & 00 & 00 & 00\\
	10 & 01 & 11 & 01\\
	11 & 01 & 10 & 01\\
	\hline
\end{tabular}
\end{center}
%
Se observa que $f(x)$ es idéntico a $f(x \oplus s)$, de modo que esta función 
cumple los requisitos para el problema.

\subsubsection{Solución clásica}

Una solución clásica consistiría en evaluar $f$ con entradas diferentes hasta 
encontrar una entrada $x_1$ que produzca una salida que ya se hubiera observado 
con otra entrada $x_2$. Entonces, debido a la restricción \ref{eq:simon_f} se 
tiene que $x_2 = x_1 \oplus s$, y por lo tanto se puede calcular $s$ como $s = 
x_1 \oplus x_2$.

Debido a que dos entradas producen la misma salida, y que todas las posibles 
entradas forman pares de la forma $x_2 = x_1 \oplus s$, la función $f$ tendrá la 
mitad de sus salidas iguales. Es decir, que sólo existen $2^{n/2}$ salidas 
diferentes, por lo que al evaluar una más, $2^{n/2} + 1$ se asegura encontrar 
dos entradas que produzcan la misma salida. De esta forma la complejidad se 
sitúa en $\mathcal{O}(2^{n/2})$.

Esta solución tiene el inconveniente de ser muy costosa computacionalmente, dado 
que para encontrar una coincidencia en la salida es necesario evaluar la función 
un número exponencial de veces con respecto al número de bits $n$.

Empleando la función del ejemplo, bastaría con evaluarla $k$ veces con
$$k = 2^{2/2}+1 = 3$$
Entonces si $y_i = f(x_i)$, se obtiene $\V x = \mat{00 & 01 & 10}$ y $ \V y = 
\mat{00 &00 & 01}$.  Como se puede observar $y_1 = y_2$, de modo que $x_1 = x_2 
\oplus s$, y por lo tanto $s = x_1 \oplus x_2 = 00 \oplus 01 = 01$

\subsection{Solución cuántica}
%
El algoritmo de Simon consiste en un circuito cuántico con dos líneas, cada una
de $n$ qubits.
%
\begin{center}
	\begin{tikzpicture}%[thick]
	% `operator' will only be used by Hadamard (H) gates here.
	\tikzstyle{operator} = [draw,fill=white,minimum size=1.5em] 
	%
	\matrix[row sep=0.4cm, column sep=1cm] (circuit) {
		% First row
		\node (q1) {$\ket{0^n}$}; &
		\node[operator] (H11) {$H^{\otimes n}$}; &
		\node[operator] (P13) {}; &
		\node[operator] (H11) {$H^{\otimes n}$}; &
		\node[operator] (M11) {$M$};&
		\coordinate (end1); \\
		% Second row.
		\node (q2) {$\ket{0^n}$}; & & \node[operator] (P23) {}; & & & \coordinate (end2);\\
		% Third row
		\node (q31) {}; & \node (q32) {}; & \node (q33) {}; &
		\node (q34) {}; & \node (q35) {}; & \node (q36) {}; \\
		\node (q41) {}; & \node (q42) {}; & \node (q43) {}; &
		\node (q44) {}; & \node (q45) {}; & \node (q46) {}; \\
	};
	\node[operator] (Us) [fit = (P13) (P23), minimum width=1cm] {$U_s$};

	\node (arr0) [fit = (q31) (q32)] {$\uparrow$};
	\node (arr1) [fit = (q32) (q33)] {$\uparrow$};
	\node (arr2) [fit = (q33) (q34)] {$\uparrow$};
	\node (arr3) [fit = (q34) (q35)] {$\uparrow$};

	\node (psi0) [fit = (q41) (q42)] {$\ket{\psi_0}$};
	\node (psi1) [fit = (q42) (q43)] {$\ket{\psi_1}$};
	\node (psi2) [fit = (q43) (q44)] {$\ket{\psi_2}$};
	\node (psi3) [fit = (q44) (q45)] {$\ket{\psi_3}$};

	\node[fill=white, fit=(end1) (end2)] (cover) {};

	\begin{pgfonlayer}{background}
		% Draw lines.
		\draw (q1) -- (M11)  (q2) -- (end2);
		\draw[double, double distance=2pt] (M11) -- (end1);
	\end{pgfonlayer}

	%
	\end{tikzpicture}
\end{center}
%
Primero el sistema se inicia con ambas líneas a $\ket{0^n}$. Luego se aplica un 
operador de Hadamard sobre la primera, que creará una superposición de estados.  
Posteriormente, el operador $U_s$ computa los resultados de la función $f$ de 
forma simultánea. Para terminar, se revierte el operador de Hadamard sobre la 
primera línea, que finalmente se mide.

Los estados asociados a cada paso de la computación, han sido etiquetados como 
$\ket{\psi_i}$, de forma que en todo momento se pueda determinar el lugar al que 
corresponden en el circuito.

\subsection{Funcionamiento}
Para comprender el funcionamiento del algoritmo, se empleará un ejemplo sencillo 
con sólo dos bits, $n=2$ y de período $s=01$:
%
\begin{center}
\begin{tabular}{|c|c|}
	\hline
	$x$ & $f(x)$ \\
	\hline
	00 & 00 \\
	01 & 00 \\
	10 & 01 \\
	11 & 01 \\
	\hline
\end{tabular}
\end{center}
%
Sea $V = \{0,1\}^n$ entonces $f:V\rightarrow V$ y $f(x) = f(x\oplus s)$. De esta 
forma, $f(00) = f(00 \oplus s) = f(01)$ y también $f(10) = f(10 \oplus s) = 
f(11)$.

El sistema ha de iniciarse con ambas líneas a $\ket{0^n}$, de modo que:
%
$$ \ket{\psi_0} = \ket{0^n} \otimes \ket{0^n} $$
%
A continuación, el operador de Hadamard es aplicado sobre la línea superior.
%
\begin{equation}
\begin{split}
\ket{\psi_1} & = (H^{\otimes n} \otimes I^{\otimes n}) \ket{\phi_0} \\
	& = (H^{\otimes n} \otimes I^{\otimes n}) (\ket{0^n} \otimes \ket{0^n}) \\
	& = (H^{\otimes n} \ket{0^n}) \otimes (I^{\otimes n} \ket{0^n}) \\
	& = \frac{1}{\sqrt{2^n}} \sum_{x \in V} \ket{x, 0^n}
\end{split}
\end{equation}
%
Produciendo el estado entrelazado
%
\begin{equation}
\ket{\psi_1} = \frac{1}{\sqrt{2^2}} (\ket{00,00} + \ket{01,00} + \ket{10,00} + 
\ket{11,00})
\end{equation}
%
Posteriormente, la puerta $U_s$ actúa sobre $\ket{\psi_1}$ transformándolo en
%
\begin{equation}
\ket{\psi_2} = \frac{1}{\sqrt{2^n}} \sum_{x \in V} \ket{x, f(x) \oplus 00} = 
\frac{1}{\sqrt{2^n}} \sum_{x \in V} \ket{x, f(x)}
\end{equation}
%
En este estado, la función $f$ está siendo evaluada simultáneamente en todo $V$, 
y este efecto será clave para la reducción de complejidad. Con la función $f$ 
previamente descrita, dicho estado es
%
\begin{equation}
\ket{\psi_2} = \frac{1}{\sqrt{2^2}} (\ket{00,00} + \ket{01,00} + \ket{10,01} + 
\ket{11,01})
\end{equation}
%
Finalmente, el operador de Hadamard es nuevamente aplicado sobre la línea 
superior, de este modo se obtiene
% TODO: Explicar el origen de esta expresion
\begin{equation}
\ket{\psi_3} = \frac{1}{2^n} \sum_{x \in V} \sum_{z \in V}
	(-1)^{\braket{z|x}} \ket{z, f(x)}
\end{equation}
%
Por consiguiente, el coeficiente de cada ket será
%
\begin{equation}
c_k = \frac{1}{2^n} (-1)^{\braket{z|x}}
\end{equation}
%
Pero dado que $f(x) = f(x \oplus s)$, los kets $\ket{z, f(x)}$ y $\ket{z, f(x 
\oplus s)}$ serán el mismo, por lo que el coeficiente para este ket será
%
\begin{equation}
c_k = \frac{1}{2^n} \left((-1)^{\braket{z|x}} + (-1)^{\braket{z|x \oplus 
s}}\right)
\end{equation}
%
El producto interno $\braket{z|x \oplus s}$ puede descomponerse en $\braket{z|x} 
\oplus \braket{z|s}$, y substituyendo
%
\begin{equation}
\begin{split}
c_k & = \frac{1}{2^n} \left((-1)^{\braket{z|x}} + (-1)^{\braket{z|x} \oplus 
\braket{z|s}} \right) \\
	& = \frac{1}{2^n} \left((-1)^{\braket{z|x}} + (-1)^{\braket{z|x}}
	(-1)^{\braket{z|s}} \right)
\end{split}
\end{equation}
%
Cuando $\braket{z|s} = 1$, el coeficiente $c_k$ será
%
\begin{equation}
\begin{split}
c_k = \frac{1}{2^n} \left((-1)^{\braket{z|x}} - (-1)^{\braket{z|x}} \right) =
	\frac{1}{2^n} (0) = 0
\end{split}
\end{equation}
%
Y cuando $\braket{z|s} = 0$, será
%
\begin{equation}
c_k = \frac{1}{2^n} \left((-1)^{\braket{z|x}} + (-1)^{\braket{z|x}} \right)
	= \frac{1}{2^n} (\pm 2) = \pm 2^{1-n}
\end{equation}
%
De modo que $c_k \neq 0$ si en el ket $\ket{z, f(x)}$ se cumple que 
$\braket{z|s} = 0$. Así que sólo los estados de esta forma tendrán coeficientes 
no nulos. En el caso del ejemplo será
%
\begin{equation}
\begin{split}
\ket{\psi_3} = 2^{-n} \big( &
		+ \ket{00,00} + \ket{01,00} + \ket{10,00} + \ket{11,00} \\
	& + \ket{00,00} - \ket{01,00} + \ket{10,00} - \ket{11,00} \\
	& + \ket{00,01} + \ket{01,01} - \ket{10,01} - \ket{11,01} \\
	& + \ket{00,01} - \ket{01,01} - \ket{10,01} + \ket{11,01}
	\big)
\end{split}
\end{equation}
%
Finalmente, tras anular los términos en los que $\braket{z|01} = 1$, es decir 
$z=01$ y $z=11$, queda
%
\begin{equation}
	\ket{\psi_3} = 2^{1-n} \left( \ket{00,00} + \ket{10,00} + \ket{00,01} - 
\ket{10,01} \right)
\end{equation}
%
En este estado, si se realiza una medición de $z$, el resultado siempre cumplirá 
la restricción $\braket{z|s} = 0$. Una vez que se obtengan $n-1$ vectores 
linealmente independientes, será posible construir un sistema de ecuaciones y 
calcular $s$.

En este ejemplo, después de medir la primera línea, el conjunto de vectores 
posibles resulta $z \in \{00, 10\}$. En todo caso, se tiene que $\braket{z|s} = 
0$ de modo que $\braket{00|s} = 0$ y $\braket{10|s} = 0$, produciendo las 
ecuaciones:
%
\begin{equation}
\begin{split}
	0 s_0 \oplus 0 s_1 = 0 \\
	1 s_0 \oplus 0 s_1 = 0
\end{split}
\end{equation}
%
Cuyas soluciones son $s \in \{00, 01\}$, pero dado que $s \neq 00$, se obtiene 
$s = 01$.

\subsection{Análisis de complejidad}

El algoritmo de Simon produce en la salida tras la medición, un vector $z$ que 
pertenece a un conjunto $W$, de forma que $W = \{z / \braket{z|s} = 0\}$ para 
todo $z \in V$.

En cuanto se obtienen $n-1$ vectores independientes, el proceso concluye, ya que 
es posible averiguar $s$. Sin embargo, todos los vectores de $W$ pueden salir 
con igual probabilidad.

Sea $R$ una variable aleatoria discreta que mide el número de ejecuciones 
necesarias para completar los $n-1$ vectores independientes. El valor medio de 
ejecuciones $E[R]$, puede medir cual es la media de ejecuciones que son 
necesarias a la larga. Para calcularlo:
%
$$ E[R] = \sum^{\infty}_{x=1} x \cdot p(R=x) $$
%
Es necesario determinar cual es la probabilidad de terminar en $x$ ejecuciones 
consecutivas del algoritmo, o pasos.

\subsubsection{Probabilidad de terminar}

La probabilidad de terminar en $p$ pasos depende del número de bits $n$ y del 
número de vectores linealmente independientes $i$ ya obtenidos previamente.  

Sea $T^i_p$ la probabilidad de terminar en $p$ pasos, cuando ya se han obtenido 
$i$ vectores independientes de los $n-1$ necesarios. La probabilidad de que el 
algoritmo termine desde el comienzo en $p$ pasos será $T^0_p$, puesto que al 
comienzo no se tiene ningún vector.

Tras un paso de la ejecución, será posible encontrar un nuevo vector 
independiente con una probabilidad $I^i$, partiendo de que ya se tenían $i$ 
previamente. La probabilidad de no obtenerlo será entonces representada por $\NI 
i$, siendo $\NI i = 1 - I^i$.

De este modo, la probabilidad $T^i_p$ se puede expresar como una recurrencia, 
dividida en dos casos:

Cuando al obtener un nuevo vector, éste no es linealmente independiente, con 
probabilidad $\NI i$, de forma que será necesario resolver el problema con un 
paso menos, y los mismos vectores $i$, representado dicha probabilidad como 
$T^i_{p-1}$.

Y cuando se obtiene un vector linealmente independiente, con probabilidad $I^i$, 
que entonces existirá un nuevo vector, pero con un paso menos, con la 
probabilidad $T^{i+1}_{p-1}$ de terminar.

En forma general, se tiene:
%
\begin{equation}
	\label{eq:rec_T}
	T^i_p = \NI i T^i_{p-1} + I^i \, T^{i+1}_{p-1}
\end{equation}
%
Adicionalmente, cuando el número de pasos $p$ restantes es 0, y se han 
conseguido exactamente $n-1$ vectores, la probabilidad de acabar es total:
%
$$ T^{n-1}_0 = 1 $$
%
En cualquier otro caso, cuando $i \neq n-1$, será nula, debido a que o bien se 
ha resuelto el problema antes de lo previsto, o no se han conseguido los 
vectores necesarios.
%
$$ T^i_0 = 0 $$
%
Finalmente, las probabilidades de obtener y no obtener un vector linealmente 
independiente, partiendo de $i$ previamente, en un problema de $n$ bits son, 
respectivamente:
% TODO: Esxplicar de donde sale esto
\begin{equation}
\begin{split}
	I^i & = 1 - 2^{i-n+1}
\\
	\NI i & = 2^{i-n+1}
\end{split}
\end{equation}
%
Al poder determinar la probabilidad con la que el algoritmo resuelve el problema 
en $p$ pasos, es posible determinar el valor estimado de pasos a la larga.

\subsubsection{Valor esperado de ejecuciones}
%
El valor esperado de ejecuciones mide el número promedio de veces que es 
necesario repetir el algoritmo cuántico, a la larga, para resolver el problema.  
El número de ejecuciones $R$ es una variable aleatoria discreta, y el valor 
esperado $E[R]$ puede calcularse partiendo de las probabilidades de cada valor 
de $R$. Si $p(R=r) = T^0_r$, entonces se define el valor esperado de $R$ como
%
\begin{equation}
	E[R] = \sum^{\infty}_{p=1} p \cdot T^0_p
\end{equation}
%
\subsubsection{Resolución de la recurrencia $T^0_p$}
Para resolver la recurrencia \eqref{eq:rec_T} se empleará un método de prueba y 
simplificación hasta dar con un patrón. Se comprobarán los primeros casos para 
$3 \leq n \leq 5$, ya que crecen muy rápidamente.
%
Caso para $n=3$
%
\begin{equation}
\label{eq:rec_T3}
\begin{split}
	T^0_0 &= T^0_1 = 0 \\
	T^0_2 &= I^0I^1 \\
	T^0_3 &= I^0I^1(\NI 0 + \NI 1 ) \\
	T^0_4 &= I^0I^1(\NI 0 \NI 0 + \NI 0 \NI 1  + \NI 1 \NI 1 ) \\
	T^0_5 &= I^0I^1(\NI 0 \NI 0 \NI 0 + \NI 0 \NI 0 \NI 1  + \NI 0 \NI 1 \NI 1 + 
\NI 1 \NI 1 \NI 1) \\
\end{split}
\end{equation}
%
Para $n=4$:
%
\begin{equation}
\begin{split}
	T^0_0 &= T^0_1 = T^0_2 = 0 \\
	T^0_3 &= I^0I^1I^2 \\
	T^0_4 &= I^0I^1I^2 (\NI 0 + \NI 1 + \NI 2) \\
	T^0_5 &= I^0I^1I^2 (
		\NI 0 \NI 0 +
		\NI 0 \NI 1 +
		\NI 0 \NI 2 +
		\NI 1 \NI 1 +
		\NI 1 \NI 2 +
		\NI 2 \NI 2) \\
	T^0_6 &= I^0I^1I^2 (
		\NI 0 \NI 0 \NI 0 +
		\NI 0 \NI 0 \NI 1 +
		\NI 0 \NI 0 \NI 2 +
		\NI 0 \NI 1 \NI 1 +
		\NI 0 \NI 1 \NI 2 + \\
		& \NI 0 \NI 2 \NI 2 +
		\NI 1 \NI 1 \NI 1 +
		\NI 1 \NI 1 \NI 2 +
		\NI 1 \NI 2 \NI 2 +
		\NI 2 \NI 2 \NI 2) \\
\end{split}
\end{equation}
%
Para $n=5$:
%
\begin{equation}
\begin{split}
	T^0_0 &= T^0_1 = T^0_2 = T^0_3 = 0 \\
%
	T^0_4 &= I^0I^1I^2I^3 \\
%
	T^0_5 &= I^0I^1I^2I^3 (\NI 0 + \NI 1 + \NI 2 + \NI 3) \\
%
	T^0_6 &= I^0I^1I^2I^3 (
		\NI 0 \NI 0 +
		\NI 0 \NI 1 +
		\NI 0 \NI 2 +
		\NI 0 \NI 3 +
		\NI 1 \NI 1 +
		\NI 1 \NI 2 +\\
&		\NI	1 \NI 3 +
		\NI 2 \NI 2 +
		\NI 2 \NI 3 +
		\NI 3 \NI 3) \\
%
	T^0_7 &= I^0I^1I^2I^3 (
		\NI 0 \NI 0 \NI 0 +
		\NI 0 \NI 0 \NI 1 +
		\NI 0 \NI 0 \NI 2 +
		\NI 0 \NI 0 \NI 3 +
		\NI 0 \NI 1 \NI 1 +
\\&	\NI 0 \NI 1 \NI 2 +
		\NI 0 \NI 1 \NI 3 +
		\NI 0 \NI 2 \NI 2 +
		\NI 0 \NI 2 \NI 3 +
		\NI 0 \NI 3 \NI 3 +
		\NI 1 \NI 1 \NI 1 +
\\&	\NI 1 \NI 1 \NI 2 +
		\NI 1 \NI 1 \NI 3 +
		\NI 1 \NI 2 \NI 2 +
		\NI 1 \NI 2 \NI 3 +
		\NI 1 \NI 3 \NI 3 +
		\NI 2 \NI 2 \NI 2 +
\\&	\NI 2 \NI 2 \NI 3 +
		\NI 2 \NI 3 \NI 3 +
		\NI 3 \NI 3 \NI 3) \\
\end{split}
\end{equation}
%
Se observa que en cada expresión $T_p^0$ se encuentra un producto de $I^j$. Dado 
que es necesario encontrar exactamente $n-1$ vectores linealmente 
independientes, este producto se corresponde con las probabilidades de encontrar 
los $n-1$ vectores uno detrás de otro. Primero partiendo de $0$ vectores, luego 
de $1$ y así sucesivamente hasta $n-2$. Dado que se conoce $n-1$ este producto 
se puede expresar como
$$
	J =\prod^{n-2}_{j=0} I^j
$$
%
El resto de la ecuación se denotará como $S_p$, de forma que se obtiene
$$
	T^0_p = J \cdot S_p
$$
%
La expresión $S_p$ es una suma de probabilidades, y representan las diferentes 
combinaciones de obtener los $n-1$ vectores. Por ejemplo, para $n=3$, es 
necesario obtener 2 vectores linealmente independientes. Sea el número de pasos 
$p = 4$. Se empleará una cadena binaria de $p$ bits para representar si en una 
ejecución se obtiene un nuevo vector independiente indicado con un 1, o con un 0 
si no se obtiene.

Dado que son necesarios exactamente $n-1$ vectores, todas las ejecuciones 
contendrán exactamente dos unos.  Además no puede haber ejecuciones que consigan 
haber conseguido 2 unos antes de terminar, dado que entonces habrían resuelto el 
problema en menos pasos de los necesarios.  Las 
3 posibles ejecuciones son:
\begin{itemize}
\item Conseguirlos al final de todo: 0011
\item Conseguir uno en la segunda ejecución y otro al final: 0101
\item Conseguir uno al principio y otro al final: 1001
\end{itemize}
La probabilidad de que no se obtenga un vector linealmente independiente, 
expresada como $\NI i$, tan sólo depende del número de vectores independientes 
de los que ya se dispone. De esta forma, las probabilidades de obtener el primer 
resultado 0011 serán:
$$ \NI 0 \NI 0 I^0 I^1 $$
Para el segundo, 0101:
$$ \NI 0 I^0 \NI 1 I^1 $$
Y para el último, 1001:
$$ I^0 \NI 1 \NI 1 I^1 $$
Estas probabilidades son las tres posibles opciones que resuelven el problema en 
exactamente 4 pasos. Al sumarlas se obtiene la probabilidad de que el problema 
sea resuelto empleando alguna posible ejecución correcta. Dado que cada 
ejecución es independiente se tiene:
\begin{equation}
\begin{split}
\NI 0 \NI 0 I^0 I^1 + \NI 0 I^0 \NI 1 I^1 + I^0 \NI 1 \NI 1 I^1 =
I^0I^1(\NI 0 \NI 0 + \NI 0 \NI 1  + \NI 1 \NI 1 )
\end{split}
\end{equation}
Que es exactamente la misma expresión obtenida por la recurrencia 
\eqref{eq:rec_T3} con $T^0_4$ para $n=3$:
%
$$
	T^0_4 = \underbrace{I^0I^1}_J
	(\underbrace{\NI 0 \NI 0 + \NI 0 \NI 1  + \NI 1 \NI 1 }_{S_p}) $$
%

El siguiente paso, será determinar como se forman las secuencias $S_p$.
Dado que son necesarios $n-1$ vectores linealmente independientes, el resto de 
ejecuciones $p - (n-1)$ proporcionan vectores dependientes o repetidos. Es decir 
que se obtendrán secuencias de $\NI i$ con $p - (n-1)$ elementos.

Sea $K$ una sucesión de $m$ productos $K = \NI{i_1} \NI{i_2} \ldots \NI{i_m}$, 
partiendo de la definición de $\NI i$, se tiene:
%
\begin{equation}
\label{NI_seq}
\begin{split}
	K &= 2^{i_1-n+1} \, 2^{i_2-n+1} \ldots 2^{i_m-n+1} \\
		&= 2^{(i_1-n+1) + (i_2-n+1) + \ldots + (i_m-n+1)} \\
		&= 2^{m(-n+1) + i_1 + i_2 + \ldots + i_m} \\
		& = 2^{m(-n+1)} \, 2^{i_1 + i_2 + \ldots + {i_m}}
\end{split}
\end{equation}
%
Examinando las secuencias de productos para $n = 3$, con $p \geq n$:
%
\begin{equation}
\begin{split}
	S_3 &= \NI 0 + \NI 1\\
	S_4 &= \NI 0 \NI 0 + \NI 0 \NI 1 + \NI 1 \NI 1 \\
	S_5 &= \NI 0 \NI 0 \NI 0 + \NI 0 \NI 0 \NI 1  + \NI 0 \NI 1 \NI 1 + \NI 1 \NI 
1 \NI 1 \\
\end{split}
\end{equation}
%
Reemplazando las secuencias con la ecuación \eqref{NI_seq}:
%
\begin{equation}
\begin{split}
	S_3 &= 2^{1(-n+1)} (2^0+2^1) = 1/4^1 \cdot 3\\
	S_4 &= 2^{2(-n+1)} (2^{0+0} + 2^{0+1} + 2^{1+1}) = 1/4^2 \cdot 7\\
	S_5 &= 2^{3(-n+1)} (2^{0+0+0}+2^{0+0+1}+2^{0+1+1}+2^{1+1+1}) = 1/4^3 \cdot 
15\\
\end{split}
\end{equation}
%
Para $n = 4$:
%
\begin{equation}
\begin{split}
	S_4 &= 1/8^1 \cdot 7\\
	S_5 &= 1/8^2 \cdot 35\\
	S_6 &= 1/8^3 \cdot 155\\
\end{split}
\end{equation}
%
Para $n = 5$:
%
\begin{equation}
\begin{split}
	S_5 &= 1/16^1 \cdot 15 \\
	S_6 &= 1/16^2 \cdot 155 \\
	S_7 &= \underbrace{1/16^3}_{P_p} \cdot
		\underbrace{\vphantom{1/16^3}1395}_{Q_p} \\
\end{split}
\end{equation}
%
Por una parte, $S_p$ contiene la potencia $P_p = 2^{(p-n+1)(-n+1)}$, y por otra, 
un número $Q_p$, de modo que
$$ S_p = P_p \cdot Q_p$$
%
Al examinar las secuencias $Q_p$, se obtienen los valores:
\begin{center}
	\begin{tabular}{|c|c *{4}{|c}|}
		\hline
		$n$ & $Q_{n}$ & $Q_{n+1}$ & $Q_{n+2}$ & $Q_{n+3}$ & $Q_{n+4}$ \\ \hline
		3   & 3       & 7         & 15        & 31        & 63        \\
		4   & 7       & 35        & 155       & 651       & 2667      \\
		5   & 31      & 651       & 11811     & 2007878   & 3309747   \\
		\hline
	\end{tabular}
\end{center}
%
Estas secuencias se corresponden con los coeficientes binomiales de Gauss con el 
parámetro $q=2$, definidos para $r\leq m$, como:
$$
	{m \choose r}_{q=2} = \frac
		{\prod_{i=m}^{m-r+1}{(1-2^i)}}
		{\prod_{i=1}^{r}{(1-2^i)}} =
	\prod_{i = 0}^{r-1}{\frac
		{(2^m-2^i)}
		{(2^r-2^i)}}
$$
De esta forma, el valor de $Q_p$ se determina como:
$$
Q_p = {p-1 \choose n-2}_{q=2}
$$
%
Reunificando todos los componentes que forman la recurrencia, se obtiene
\begin{equation}
\label{eq_T}
\begin{split}
	T^0_p &= J \cdot S_p = J \cdot P_p \cdot Q_p = \\
	&= \prod^{n-2}_{j=0} I^j \cdot P_p \cdot Q_p = \\
	&= \prod^{n-2}_{j=0} I^j \cdot 2^{(p-n+1)(-n+1)} \cdot Q_p = \\
	&= \prod^{n-2}_{j=0} I^j \cdot 2^{(p-n+1)(-n+1)} \cdot {p-1 \choose n-2}_{q=2}
\end{split}
\end{equation}
%
\subsubsection{Cálculo del valor medio}
Una vez obtenida la ecuación \eqref{eq_T}, se puede reemplazar la probabilidad 
$T_p^0$ en $E[R]$, quedando:
%
\begin{equation}
\begin{split}
	E[R] &= \sum_{p=1}^\infty p T^0_p =\\
	&= \sum_{p=1}^\infty p \cdot \prod^{n-2}_{j=0} I^j \cdot 2^{(p-n+1)(-n+1)} 
\cdot {p-1 \choose n-2}_{q=2} = \\
	&= \prod^{n-2}_{j=0} I^j \cdot \sum_{p=1}^\infty p \cdot 2^{(p-n+1)(-n+1)} 
\cdot {p-1 \choose n-2}_{q=2}
\end{split}
\end{equation}
%
De este modo, se pueden computar los valores teóricos esperados de ejecuciones 
que cabría esperar a la larga.
\begin{table}[h]
	\begin{tabular}{|c|*{9}{|c}|}
		\hline
		$n$      & 2 & 3 & 4 & 5 & 6 & 7 & 8 & 9 & 10\\ \hline
		$E[R]$   &
2.000& 3.333& 4.476& 5.542& 6.575& 7.591& 8.598& 9.602& 10.60
\\
		$\frac{E[R]}{n}$ &
1.000& 1.111& 1.119& 1.109& 1.096& 1.084& 1.075& 1.067& 1.060\\
		\hline
	\end{tabular}
\caption{Valor esperado de ejecuciones del algoritmo de Simon}
\label{tabla_E}
\end{table}
%

Se observa como $E[R]$ se va aproximando cada vez más a $n$. De esta forma, el 
algoritmo se comporta con una complejidad a la larga de $O(n)$.
